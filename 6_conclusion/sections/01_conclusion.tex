\section{Conclusion}
\label{sec:conclusion:conclusion}

In this work, we address three main research questions related to the performance evaluation of \gls{sm} systems. The approach used to answer these research questions follows the vision of \gls{mcs}, in which we heavily emphasize the design, implementation, deployment, analysis, and benchmarking of distributed systems. This approach divided our research on \gls{sm} systems in three major chapters that each aim to answer a main research question. In \cref{chap:survey} we performed an extensive system survey, to analyse the state-of-the-art \gls{sm} systems. Then, in  \cref{chap:system-design}, we design \textit{Mesh Bench}, a benchmark for \gls{sm} systems based on our previous learnings. Last, in \cref{chap:experimental-evaluation}, we take an experimental approach to evaluating \gls{sm} systems, by conducting performance related experiments and performing an extensive analysis on the results.

We now address the main research questions individually, to answer our research questions and conclude our work.


\begin{enumerate}[label=\textbf{RQ\arabic*}, leftmargin=3\parindent]
    \item \textbf{How to compare, and evaluate \gls{sm} systems?}
    \label{rq-1:ans}
    
    We presented an extensive system survey in \cref{chap:survey}, in which we identified several state-of-the-art \gls{sm} systems. During this survey, we identified several domain-specific characteristics and features that these systems had and categorized them under functional and non-functional requirements of a \gls{sm} system. Based on these findings we created a comparison framework, which allowed us to compare various \gls{sm} systems based on these attributes. Furthermore, we identified the primary components that could impact the performance of these systems. We analysed the proxy components that make up the actual mesh, the layer of dedicated infrastructure often referred to as the data plane, and discovered various approaches and architectural designs that could heavily influence the performance of these systems. This analysis laid the groundwork for the work presented in this thesis, and established the direction we take to evaluate the performance of \gls{sm} systems.

    
    \item \textbf{How to design and implement a benchmark that evaluates the performance of \gls{sm} systems?}
    \label{rq-2:ans}
    
    Based on the results of our system survey, we designed and implemented a prototype of \textit{Mesh Bench} in \cref{chap:system-design}. To guide our design process, we established a set of benchmarking objectives that capture the performance related characteristics of distributed systems which are applicable to \gls{sm} systems. We then used an iterative design process guided by a set of established best practices to design our benchmark. We bootstrapped this process by properly defining the \gls{sut}, which consists of the \gls{sm} components that we aim to evaluate. Following that, we performed an extensive requirements analysis that identifies the stakeholders of such a benchmark system and their concerns and use cases. Based on this information we constructed a detailed design of Mesh Bench, our benchmark system for \gls{sm} systems. We then concluded this chapter of our research with a prototype implementation, in which we discussed each implementation detail and their potential alternatives. The resulting prototype then served as the base for our experimental evaluation.
    
    \item \textbf{What are the differences between current \gls{sm} systems in terms of overhead, throughput, and latency?}
    \label{rq-3:ans}
    
    The experimental evaluation as presented in \cref{chap:experimental-evaluation} was set out to uncover the performance related intricacies of \gls{sm} systems and provide an answer to our final main research question. It builds upon the obtained knowledge from the system survey, and used the prototype of Mesh Bench to perform various performance related experiments. We designed several performance related experiments that evaluate \gls{sm} systems under various conditions by changing three primary dimensions. We found that \gls{sm} systems can have a significant impact on maximum sustained throughput, e.g. an ~80\% reduction in one of the most commonly used \gls{sm} systems. Additionally, we provided evidence that tail latencies could be massively affected by the network proxies in certain cases, and heavily emphasized the impact tail latencies can have in a service-oriented architecture. Finally, we have shown that there is a huge discrepancy between common \gls{sm} systems in terms of resource utilization, where competing systems can utilize 2x the amount of CPU resources under similar levels of load. Finally, in our analysis we relate back to our architectural findings of the system survey, and identify promising future directions for \gls{ebpf}-based networking solutions.

\end{enumerate}