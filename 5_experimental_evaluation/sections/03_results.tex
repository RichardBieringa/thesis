\section{Experiment Results}
\label{sec:experiments:results}

In this section, we present the results obtained from the experiments as defined in \cref{sec:experiments:design:overview}. We first present the main findings in \cref{sec:experiments:main-findings}. After this, we present and explain how to interpret the results per experiment and discuss their findings in greater detail.


\todo{Feedback: dodge mr bar chart, if possible}
\todo{Feedback: line plot -> change to something}
\todo{Feedback: line plot -> microbench bar chart -> line plot}
\todo{Feedback: line plot -> cpu line}


\subsection{Main Findings}
\label{sec:experiments:main-findings}
% Significant decrease in throughput
% Additional avg latency, esp tail ends. p99+
% Traefik experiences bottlenecks, bimodal distribution
% Istio consumes most CPU resources
% Applications Payloads seem to have little effect on performance
% Application protocol seems to have little impact, although linkerd performs a bit worse on gRPC throughput
% Traefik failed to handle gRPC worklaods
% Cilium best allround performer

We now present the main findings of the conducted experiments.

\begin{enumerate}[label=\textbf{MF\arabic*}, leftmargin=3\parindent]
    \item \textbf{There can be a significant drop in maximum throughput when using a service mesh}
    \label{exp:mf1}
    Every service mesh suffers from a decrease in maximum throughput. The observed decrease in throughput varies greatly, with a best-case scenario having a  $-16.8\%$ decrease while the worst performing configuration has a significant  $-97.38\%$ decrease in maximum throughput.
    
    \item \textbf{There can be a significant increase in request latencies when using a service mesh}
    \label{exp:mf2}
    A service mesh introduces additional request latency by introducing a network proxy in front of the service. This additional overhead, however, can have a large impact on average and tail end request latencies as observed. With the best performing mesh increasing the average request latency with $20.27\%$ whereas the worst performing configuration saw a staggering $3720.95\%$ uplift.
    
    \item \textbf{Traefik mesh experiences bottlenecks after it reaches a certain throughput threshold}
    \label{exp:mf3}
    We observed that \textit{Traefik} cannot perform above a certain throughput level, and will throttle requests, resulting in a bimodal distribution of request latencies and a significant loss of performance for those requests.
    
    \item \textbf{Istio is the most demanding in terms of CPU utilization}
    \label{exp:mf4}
    We observe that \textit{Istio} consumes significantly more CPU resources compared to other configurations. We also observe that this is due to its data plane proxy which does not scale as well as the proxies in other configurations.
    
    \item \textbf{Application payloads have little to no effect on the observed performance}
    \label{exp:mf5}
    We observed that increasing the application payload, has little to no effect on the performance of meshed configurations.
    
    \item \textbf{Application level protocols have little to no effect on the observed performance}
    \label{exp:mf6}
    We observed that different layer 7 protocols, HTTP vs gRPC, had little to no impact on the decrease in throughput observed.
    
    \item \textbf{Traefik was unable to handle gRPC workloads}
    \label{exp:mf7}
    We observed that \textit{Traefik} was unable to handle and execute gRPC workloads in our experiments even though it listed explicit support for gRPC.
    
    \item \textbf{Cilium was the best all-round performing configuration, with a significant margin}
    \label{exp:mf8}
    We observed that the configuration using \textit{Cilium} outperformed any other configuration using a \gls{sm} with significant margins in all the experiments that were conducted.
\end{enumerate}


% Results per experimetn
\subsection{\ref{exp:design:1} - HTTP Maximum Throughput}
\label{sec:experiments:results:per-experiment:01}
% Goal:
% To find the maximum throughput each of the meshed configurations can achieve.

\begin{figure}
\centering
\makebox[\linewidth][c]{
    \begin{minipage}{.6\textwidth}
      \centering
      \includegraphics[width=\linewidth]{5_experimental_evaluation/figures/exp-01-max-throughput.pdf}
      \caption[Average throughputs of \gls{sm} systems under maximum load.]{Average throughputs of \gls{sm} systems under maximum load.}
      \label{fig:exp:01:maximum-throughput}
    \end{minipage}
    
    
    \begin{minipage}{.6\textwidth}
      \centering
      \includegraphics[width=\linewidth]{5_experimental_evaluation/figures/exp-01-tail-latencies.pdf}
      \caption[Tail end latencies of \gls{sm} systems under maximum load.]{Tail end latencies of \gls{sm} systems under maximum load.}
      \label{fig:exp:01:tail-end-latencies}
    \end{minipage}
}
\end{figure}



The first experiment evaluates the \gls{sm} systems when they are fully satiated, i.e. they are at full capacity and handling the maximum amount of load that they can process. This amount of load, or throughput, is measured by the number of requests per second the system can process.

\subsubsection{Maximum Sustained Throughput Analysis}
\label{sec:experiments:results:per-experiment:01:throughput}

In \cref{fig:exp:01:maximum-throughput} we present a bar chart which depicts the average, sustained throughput that a \gls{sm} system was able to process throughout the duration of the experiment. On the y-axis we present the different \gls{sm} configurations, whereas the x-axis represents the throughput in requests per second. Next to each bar we present the actual observed value and the percentage of throughput a system was able to process compared to the  best performing configuration, which in this case is the baseline.

From these results, we can derive that all the \gls{sm} systems experience a loss of throughput compared to the baseline configuration. We can relate this observation to the fact that the \gls{sm} systems introduce additional components in the form of proxies in the critical data path of requests. Every request has to be processed by these proxies, which in turn leads to the decrease in maximum sustained throughput. However, even through the decrease in throughput is expected, the amount of this decrease is rather significant. Cilium is the best performing \gls{sm} configuration in this experiment in terms of throughput. However, it still experienced a massive $16.83\%$ reduction compared to the baseline configuration. This observation leads to our first main finding: 

\begin{shaded*}
    \noindent
    \ref{exp:mf1}: 
    Using a \gls{sm} can lead to a significant decrease in sustained throughput.
\end{shaded*}

Another observation we can make from these results is the amount of variance there is among the evaluated \gls{sm} systems. We already established that Cilium is the best performing \gls{sm} system in terms of throughput, even though it had a significant reduction compared to the baseline configuration. This reduction in throughput, however, is little compared to the performance of other configurations. The configuration using Linkerd led to a $54.94\%$ reduction and Istio saw a staggering $80.09\%$ reduction in throughput. The worst performing configuration was using Traefik. This configuration only managed to serve a tiny fraction of the requests and experienced a massive $97.39\%$ reduction in throughput compared to the baseline configuration. From these observations, we can conclude that there is a large variance in the observed throughputs among \gls{sm} systems, where the reductions in throughput range from $16.83\%$ to $97.39\%$. This leads to our second main finding:

\begin{shaded*}
    \noindent
    \ref{exp:mf2}: 
    There is a significant variance in the amount of sustained throughput that \gls{sm} systems can handle.
\end{shaded*}

\subsubsection{Latency Analysis}
\label{sec:experiments:results:per-experiment:01:latency}

In our previous analysis we looked at the amount of requests a configuration can handle. In this part of the analysis we take a look at the durations of the requests themselves. We measure the latency of each request. These latencies represent the time it takes for a request to complete. This is measured from the moment the workload generator sends a request until it successfully receives a response from the target service.

To fully understand the impact that the values and distributions in this analysis can have on real-world application performance, we have to refer to the manner in which applications are often constructed. When using a service-oriented architecture, and namely one using the granularity of microservices, you create an application of many interconnected services. Applications can consist of thousands of microservices \cite{design-example-microservices, netflix-microservices-cost} and a user can indirectly use hundreds of backing services. To evaluate latencies in such a model, we follow the practices as described in the works of Gil Tene \cite{Tene2015-measure-latency}. Because of this model, the tail end latencies are more common than one might expect due to the nature of probabilities. As an example, the ~99.995th percentile of latencies will be experienced in a procedure involving 200 microservices in more than 99\% of the time. 


\begin{figure}
\centering
\makebox[\linewidth][c]{
    \begin{subfigure}{.6\textwidth}
      \centering
      \includegraphics[width=\textwidth]{5_experimental_evaluation/figures/exp-01-latency-log.pdf}
      \caption{Logarithmic scale}
      \label{fig:exp:01:boxplots-latency:log}
    \end{subfigure}
    
    
    \begin{subfigure}{.6\textwidth}
      \centering
      \includegraphics[width=\textwidth]{5_experimental_evaluation/figures/exp-01-latency-no-traefik.pdf}
      \caption{Linear scale, excluding Traefik}
      \label{fig:exp:01:boxplots-latency:linear}
    \end{subfigure}
}
\caption[Latency distributions of \gls{sm} systems under maximum load]{Latency distributions of \gls{sm} systems under maximum load.}
\label{fig:exp:01:boxplots-latency}
\end{figure}


In \cref{fig:exp:01:boxplots-latency} we present a pair of box and whiskers plots that depict the locality, spread and skewness of latencies per \gls{sm} configuration. The lines at the borders of the box represent the 25th and 75h percentiles of data. The line in the middle of the box depicts the median observed value and the whiskers of the box represent the minimum and maximum values. It is important to note that the outliers are omitted from these figures, as we cover these in greater detail later on. The plots depicted in the figure share the meaning of their axes, where the y-axis represents the \gls{sm} configuration and the x-axis represent the observed latencies in milliseconds. The first plot (\cref{fig:exp:01:boxplots-latency:log}), shows all the \gls{sm} configurations and displays the box plots on a logarithmic scale. The second plot (\cref{fig:exp:01:boxplots-latency:linear}), displays the data on a linear scale but excludes Traefik.

In \cref{fig:exp:01:boxplots-latency:log} we present the observed latencies on a logarithmic scale. We do this because it shows that the observed latencies for Traefik surpass the other configurations by an order of magnitude. In \cref{fig:exp:01:boxplots-latency:linear} we compare the \gls{sm} configurations on a linear scale and exclude Traefik. From these results, we can observe that the distribution of observed latencies for Cilium is very similar to that of the baseline configuration. This indicates that the network overhead caused by the in-kernel proxy of Cilium is kept to a minimum which leads us to the third main finding:

\begin{shaded*}
    \noindent
    \ref{exp:mf3}: 
    The network latency overhead caused by the proxy of Cilium is minimal.
\end{shaded*}

Another observation we can make is that Linkerd has a slightly higher median and spread, but is still performing relatively well as it does not deviate too much from the baseline configuration. Istio, on the other hand, appears to suffer significantly more. Not only is the median affected, the spread of observed latencies is significantly greater. We can relate this behaviour to the type of proxy used in their data planes and their design decisions (see \cref{sec:survey:results} \cref{tab:result-proxy}). Whereas Istio uses \textit{Envoy}, a complex and feature rich general purpose proxy, Linkerd uses its own lightweight proxy \cite{linkerd-no-envoy}.


In \cref{fig:exp:01:tail-end-latencies} we present the tail end of latencies observed (above the 99th percentile) for the \gls{sm} configurations excluding Traefik. On the y-axis we represent the latencies observed in millisecond whereas the x-axis represents the various \gls{sm} configurations. The colours of the bars represent a certain percentile of latencies observed.

From the chart in \cref{fig:exp:01:tail-end-latencies} we can observe that Istio suffers the most in the observed tail end latencies. We observed a latency values of 40, 141 and 358 milliseconds for the 99th, 99.9th and 99.99th percentiles respectively. This observation is a continuation from the behaviour depicted in the box and whiskers plot (\cref{fig:exp:01:boxplots-latency:linear}), where the results tied to the configuration using Istio also depicted a large spread in observed latency values. This leads to a fourth main finding:

\begin{shaded*}
    \noindent
    \ref{exp:mf4}: 
    The latencies observed from Istio under load have a large spread, and especially suffer in their tail end.
\end{shaded*}

\begin{figure}
\centering
\makebox[\linewidth][c]{
    \begin{subfigure}{.6\textwidth}
      \centering
      \includegraphics[width=\textwidth]{5_experimental_evaluation/figures/exp-01-latency-distribution-baseline.pdf}
      \caption{Baseline}
      \label{fig:exp:01:histogram:baseline}
    \end{subfigure}
    
    
    \begin{subfigure}{.6\textwidth}
      \centering
      \includegraphics[width=\textwidth]{5_experimental_evaluation/figures/exp-01-latency-distribution-traefik.pdf}
      \caption{Traefik}
      \label{fig:exp:01:histogram:traefik}
    \end{subfigure}
}
\caption[Histogram of observed latencies under maximum load]{Histogram of observed latencies under maximum load per \gls{sm} configuration.}
\label{fig:exp:01:histograms-latency}
\end{figure}

To evaluate the behaviour of Traefik in more detail, we depict the latency distributions in a histogram. In \cref{fig:exp:01:histograms-latency} we depict two plots where each plot contains a histogram of latency distributions. The first plot contains the histogram of the baseline configuration (\cref{{fig:exp:01:histogram:baseline}}) and the second plot contains the histogram of Traefik  (\cref{{fig:exp:01:histogram:traefik}}). Each histogram has a y-axis that represents the number of requests for a specific bin and an x-axis that represents the latency in milliseconds. In addition to the bins, we depict the cumulative density function that shows the proportion of requests smaller than the latency depicted on the x-axis. 

From the results depicted in \cref{fig:exp:01:histograms-latency}, we can observe the distribution of latencies for the baseline configuration and that of Traefik. Note that the values next to the y-axis depict different values, there are more latencies observed for the baseline configuration, as the throughput was not fixed in this experiment. The shape of the distribution for the baseline configuration is similar to that of other configurations except for Traefik (see Appendix for a full comparison). From the shape of the histogram distribution of Traefik we can derive a peculiar finding, it has a bimodal distribution. The first mode is close to 8ms, whereas the second mode is around the 90 milliseconds. These observations lead to our fifth  finding of the experiment: 

\begin{shaded*}
    \noindent
    \ref{exp:mf5}: 
    Traefik mesh performs an order of magnitude worse than any other evaluated service mesh in terms of request latencies when the system is under full load.
\end{shaded*}

\subsubsection{Resource Utilization Analysis}
\label{sec:experiments:results:per-experiment:01:throughput}


\begin{figure}
\centering
\makebox[\linewidth][c]{
    \begin{subfigure}{.6\textwidth}
      \centering
      \includegraphics[width=\textwidth]{5_experimental_evaluation/figures/exp-01-cpu-utilization.pdf}
      \caption{CPU Utilization}
      \label{fig:exp:01:resource:cpu}
    \end{subfigure}
    
    
    \begin{subfigure}{.6\textwidth}
      \centering
      \includegraphics[width=\textwidth]{5_experimental_evaluation/figures/exp-01-memory-utilization.pdf}
      \caption{Memory Utilization}
      \label{fig:exp:01:resource:mem}
    \end{subfigure}
}
\caption[Resource utilization for \gls{sm} systems under load]{Resource utilization for \gls{sm} systems under load}
\label{fig:exp:01:resource-utilization}
\end{figure}


The final type of analysis that we perform is related to the utilization of system resources. In \cref{fig:exp:01:resource-utilization}, we present two line graphs that depict resource usage over time. The first plot (\cref{fig:exp:01:resource:cpu}) is related to the CPU utilization of \gls{sm} proxies. The y-axis represents the fraction of CPU cores used for the \gls{sm} proxy (on a two core system), and the x-axis represents the time delta of the experiment. The second plot depicts \cref{fig:exp:01:resource:mem} the memory utilization. For this, plot the y-axis represents the memory utilization in kilobytes, whereas the x-axis once again represents the time delta of the experiment. In these plots, we display three out of four \gls{sm} systems. This is because we were only able to capture the user-level application containers related to Cilium, and not the kernel-level proxy. For this reason, we omit Cilium from the resource graphs, as the data gathered was inconclusive for that system.

The results as depicted in \cref{fig:exp:01:resource-utilization} show the various configurations under varying levels of load. This means that the comparison cannot be considered fair. With that said, we can still observe some interesting behaviour. In \cref{fig:exp:01:resource:cpu} we depict the CPU utilization for the three observed systems. We can see that the proxy of Istio is the largest consumer of the CPU, even though as previously observed, it had a significantly fewer requests to process compared to Linkerd. This can be related to the aforementioned design decisions of the proxy implementations \cite{linkerd-no-envoy}. Another thing to note is that Traefik, the worst performing \gls{sm} system consumes the least amount of CPU resources. This at the very least proves that the poor performance is not related to any CPU related bottlenecks.

From the results depicting memory utilization (\cref{fig:exp:01:resource:cpu}) we can derive that the memory utilization for the data plane proxies is very minimal, even under maximum load. The highest values observed as depicted by the spikes in the graph, are less than 1500Kb and pose no significant bottleneck for most systems and environments.


\subsection{\ref{exp:design:2} - HTTP Constant Throughput}
\label{sec:experiments:results:per-experiment:02}
% Goal: To evaluate how \gls{sm} configurations behave under varying levels of load.

In the previous experiment (\cref{sec:experiments:results:per-experiment:01}), we evaluated \gls{sm} systems under maximum load. In this experiment, we evaluate these systems under varying, pre-defined levels of constant throughput. The results of these experiments aim to show how the \gls{sm} systems scale, across varying levels of load.

\subsection{Latency Analysis}
\label{sec:experiments:results:per-experiment:02:latency}

\begin{figure}[!t]
    \centering
    
    \includegraphics[width=1\linewidth]{5_experimental_evaluation/figures/exp-02-latency-log.pdf}
    
    \caption[Box plot of observed latencies under various levels of constant throughput]{Box plot of observed latencies per service mesh configuration under various levels of constant throughput on a logarithmic scale.}
    
    \label{fig:exp:02:latency-distributions}
\end{figure}


In \label{fig:exp:02:latency-distributions} we present the latency distributions of the evaluated \gls{sm} configurations under various levels of constant throughput.  On the y-axis we present the \gls{sm} configurations and on the x-axis the latency expressed in milliseconds on a logarithmic scale. The legend on the plots display the various levels of throughput depicted on this graph and the colours that represent them.

From the figure we can derive that the latency distributions are generally close for each of the defined levels of throughput for most of the systems. This is an indication that the various levels of throughput are manageable by the evaluated configurations. This is exemplified by the average levels of throughput that these configurations were able to process as observed in our previous experiment in \cref{sec:experiments:results:per-experiment:01}. 

One exception to this, however, is that the configuration using Traefik is experiencing a significant increase in observed latencies when experiencing a constant load of 500 requests per second. Additionally, it is important to note that the results regarding 1000 requests per second have not been included in this figure, as it was unable to manage that level of throughput. As a matter of fact, it was unable to even fully sustain the constant throughput of 500 requests per second, as it only managed to process 419 requests per second in this particular experiment. This result falls in line with the observed behaviour from our previously conducted experiment, in which we evaluated these systems under maximum load. The reason that the observed throughput is even lower in this experiment compared to the previous one, is that the previous experiment allowed for dynamic levels of throughput. This experiment on the other hand, uses constant levels of throughput as generated by the workload generator. The workload generator, however, does not make up or increase the level of throughput to compensate if the system was unable to previously process the results in time. 

Additionally, we analysed the tail end latencies of the evaluated configurations under these levels of constant throughput. The results that depict these high percentile latencies can be found in the Appendix. However, at these levels of load, we did not observe any increase in high percentile latencies as we previously did in our first experiment. The values we observed for the constant levels of throughput align with the expected values of a normal distribution.

\subsection{Traefik Bottleneck Analysis}
\label{sec:experiments:results:per-experiment:02:traefik-bottleneck}

To provide a more extensive analysis into the observed behaviour of Traefik, we visualized the latency distributions of the configuration using Traefik in violin plot as depicted in \cref{fig:exp:02:traefik-bottleneck}. The y-axis of this plot represents the various levels of throughput, and the x-axis present the observed latency values in milliseconds. It is important to note that we did include the results of when we evaluated Traefik under a constant throughput of 1000 requests per second, even if it did not manage to actually sustain this level of throughput as previously discussed. 

\begin{figure}[!t]
    \centering
    
    \includegraphics[width=1\linewidth]{5_experimental_evaluation/figures/exp-02-bottleneck-traefik.pdf}

    \caption[Latency distribution of Traefik under varying levels of constant throughput]{Latency distribution of Traefik under varying levels of constant throughput.}
    
    \label{fig:exp:02:traefik-bottleneck}
\end{figure}

Through the violin plots as depicted in \cref{fig:exp:02:traefik-bottleneck}, we can observe the behaviour and bottlenecks of the system under load in more detail. The first violin plot shows the latency distribution when it is under a constant load of 100 requests per second. From this, we can derive that the system can process this amount and that the distribution of latencies is similar to that of other evaluated \gls{sm} systems as previously seen. The second violin plot, however, shows the system under a constant load of 500 requests per second. This plot shows the first signals of a potential bottleneck in the system. We can observe that the observed latency values have a higher spread and that these values are often an order of magnitude higher, often reaching values well above 50 milliseconds. This provides a sharp contrast with the other systems that we evaluated, which performed significantly better under full or near full load. The third violin plot, however, shows the bottleneck in full effect. This plot depicts the system under a constant throughput of 1000 requests per second, which it was unable to process. This results in the previously observed bimodal distribution of requests latencies. This leads us to our sixth main finding:


\begin{shaded*}
    \noindent
    \ref{exp:mf6}: 
    Traefik experiences bottlenecking behaviour under a load of approximately 500 requests per second resulting in a bimodal distribution of request latencies.
\end{shaded*}


\subsection{Resource Utilization Analysis}
\label{sec:experiments:results:per-experiment:02:resource}

\begin{figure}
\centering
\makebox[\linewidth][c]{
    \begin{subfigure}{.6\textwidth}
      \centering
      \includegraphics[width=\textwidth]{5_experimental_evaluation/figures/exp-02-cpu-utilization.pdf}
      \caption{CPU Utilization}
      \label{fig:exp:02:resource:cpu}
    \end{subfigure}
    
    
    \begin{subfigure}{.6\textwidth}
      \centering
      \includegraphics[width=\textwidth]{5_experimental_evaluation/figures/exp-02-memory-utilization.pdf}
      \caption{Memory Utilization}
      \label{fig:exp:02:resource:mem}
    \end{subfigure}
}
\caption[Resource utilization for \gls{sm} systems under load]{Resource utilization for \gls{sm} systems experiencing various levels of constant throughput. The level of throughput is increased at a 15-minute interval, and changes from  1, 100, 500, and 1000 requests per second respectively.}
\label{fig:exp:02:resource-utilization}
\end{figure}

In our first experiment we already took a look at the resource utilization values of various \gls{sm} systems under maximum load. However, those results could not be compared fairly, as they each experienced various levels of load. In this experiment, however, we can evaluate the resource utilization fairly as they are evaluated under similar circumstances. 

In \cref{fig:exp:02:resource-utilization} we present two plots related to the resource consumption of \gls{sm} systems. Both plots can be interpreted similarly. On the y-axis we have the value of interest, which is the CPU utilization as expressed in fractions of a core for the first plot and is the memory utilization expressed in Kb for the second plot. The x-axis presents the time delta of the experiment since the start. It is important to note the labels on the x-axis as these 15-minute intervals represent the time point in which the level of constant throughput was increased. The four intervals represent the constant levels of throughput from 1, 100, 500, and 1000 requests per second respectively (e.g. the interval from 0:30:00 to 0:45:00 represents the systems under a constant load of 500 requests per second). Additionally, the colours of the line represent the various \gls{sm} systems that we evaluated. As previously noted, we are unable to evaluate the actual resource utilization of Cilium as the actual in-kernel proxy is not exposed as a user-space application in the form of a container.

The first plot (\cref{fig:exp:02:resource:cpu}), shows the CPU utilization throughout the experiment for the various \gls{sm} systems. The first thing we can observe the high values at the start of the experiment, these can be explained by the sampling rate used for the time series data which overlaps with the results from the previous experiment. The second thing we can notice is that the line from Traefik does not increase at the 45:00 minute marker. This behaviour falls in line with the previously discussed bottlenecks in our extensive analysis of the bottlenecks of Traefik. Another observation that we can make is that the CPU utilization is relatively stable for all the evaluated systems throughout the duration of the experiment, we can derive this from the lack of extremes and spikes in the presented plot.

More importantly, however, we can observe how the systems scale when the load on these systems is increased. By looking at the first 15-minute interval, we can observe that the proxies require little to no CPU utilization when they experience next to zero load (1 request per second). Interestingly however, is the behaviour when the load is increased to 100 requests per second as it allows us to fairly compare these systems under a similar load. We can observe that the Traefik proxy utilizes the CPU the most, whereas the proxy in the Istio data plane requires just slightly more than half of Traefik's usage. The Linkerd proxy, however, puts the least amount of strain on the CPU requiring less than 5\% of a CPU core at any given time. To analyse how these systems scale we take a closer look at the third and final time intervals. From this, we can observe that the CPU utilization scales linearly with the levels of throughput that a proxy is facing. We can also observe that the proxy empowering the Linkerd \gls{sm} scales significantly better than the one found in Istio. This is exemplified by the fact that Istio under a load of 500 requests per second uses nearly identical amount of CPU resources when Linkerd serves double the number of requests per second. This leads us to a seventh main finding:

\begin{shaded*}
    \noindent
    \ref{exp:mf7}: 
    There can be a significant difference in the amount of CPU utilization for 
\end{shaded*}

\subsection{\textbf{EXP 3} - HTTP Payload}
\label{sec:experiments:results:per-experiment:03}

In the third experiment we introduced variable application payloads. The goal of this experiment is to analyse the \gls{sm} systems when they have to process various amounts of data as application payloads. During this experiment, we generate a constant throughput of 100 requests per second, however, the application payload of each of the HTTP responses vary between 0, 1000, and 10000 bytes in size. 


\begin{figure}[t]
    \centering
    
    \makebox[\linewidth][c]{
        \includegraphics[width=1.3\textwidth]{5_experimental_evaluation/figures/exp-03-latencies-all.pdf}
    }

   \caption[Distribution of observed latencies per \gls{sm} system with varying application payload sizes]{Distribution of observed latencies per \gls{sm} system with varying application payload sizes.}
    
    \label{fig:exp:result:03:latency}
\end{figure}

In \cref{fig:exp:result:03:latency} we present the observed latency distributions of the various \gls{sm} configurations whilst processing varying levels of application payloads. On the x-axis we present the varying configurations, whilst the y-axis represents the observed latency values in milliseconds. The colours of the bars as indicated by the legend, represent the varying levels of application payload sizes encountered.

From the distributions as presented in \cref{fig:exp:result:03:latency} we can observe that the observed latencies are slightly higher of the requests that returned a larger application payload. However, the difference is very minimal, and the selected payload sizes did not seem to impact the overall observed performance. Furthermore, the application payload size also did not seem to affect the tail end latencies, as can be seen in the comparison presented in the Appendix in which we compare the 99th, 99.9th and 99.00th percentile of latencies observed (\cref{fig:exp:02:tail-latencies}).


\begin{figure}[t]
\centering
\makebox[\linewidth][c]{
    \begin{subfigure}{.65\textwidth}
      \centering
      \includegraphics[width=\textwidth]{5_experimental_evaluation/figures/exp-03-cpu-utilization.pdf}
      \caption{CPU Utilization}
      \label{fig:exp:03:resource:cpu}
    \end{subfigure}
    
    
    \begin{subfigure}{.65\textwidth}
      \centering
      \includegraphics[width=\textwidth]{5_experimental_evaluation/figures/exp-03-memory-utilization.pdf}
      \caption{Memory Utilization}
      \label{fig:exp:03:resource:mem}
    \end{subfigure}
}
\caption[Experiment 3 - Resource utilization for \gls{sm} systems experiencing varying application payload sizes]{Resource utilization for \gls{sm} systems experiencing varying application payload sizes.}
\label{fig:exp:03:resource-utilization}
\end{figure}

In \cref{fig:exp:03:resource-utilization} we present two plots related to the resource consumption of \gls{sm} systems. These plots can be interpreted similarly, where on the y-axis we present the observed values that either represents the fraction of CPU cores used or the amount of memory consumed for a given data plane proxy. The x-axis once again represents the time delta of the experiment, which in total takes 15 minutes. The colours of the lines represent the various \gls{sm} systems which we evaluate.

We observe no significant differences for both of the plots presented in \cref{fig:exp:03:resource-utilization}. The CPU utilization is stable, aside from the initial spiked caused by the manner in which the time series data is aggregated and sampled. Furthermore, both CPU utilization and memory consumption levels for the evaluated systems conform to their expected values. This leads us to our singular conclusion from this experiment:

\begin{shaded*}
    \noindent
    \textbf{MF9}:
    The size of the application payload does not have a significant effect on the performance of data-plane proxies on resource utilization levels.
\end{shaded*}



\subsection{\ref{exp:design:4} - gRPC Maximum Throughput}
\label{sec:experiments:results:per-experiment:04}
% Goal: To evaluate how meshed configurations behave with alternative communication protocols.

