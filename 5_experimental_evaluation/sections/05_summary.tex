\section{Summary}
\label{sec:experiments:summary}




To summarize and conclude this chapter, we return to the initial research question (\ref{rq-3}), \textit{What are the differences between current \gls{sm} systems in terms of overhead, throughput and latency?} Based on the design and prototype implementation of \textit{Mesh Bench}, we designed several performance related experiments that provide an answer to our research question (\cref{sec:experiments:design}). First, we establish an experimental environment, which aims to minimize variance from external, unrelated variables to improve the reproducibility of experiments. We then conducted several micro benchmarks to validate this (\cref{sec:experiments:microbenchmarks}). After this, we conducted the performance related experiments that evaluate \gls{sm} systems under various conditions and performed an extensive analysis \cref{sec:experiments:results}. During our analysis we discuss the significant impact a \gls{sm} system can have in terms of resource overheads, reductions in throughput and increase in (tail) latency to provide a detailed answer to our research question. We heavily emphasize on the costs associated with \gls{sm} systems and clearly show how various systems compare to one another, and to an environment without a \gls{sm} present. Furthermore, we show promising future directions based on the relatively new, in-kernel, \gls{ebpf}-based approach of networking as seen with Cilium, the \gls{sm} that outperformed all other systems in all of our experiments. Finally, in \cref{sec:experiments:threats} we discuss potential threats to the validity of the experimental evaluation.




% \item (\textit{Experimental}, \ref{rq-3}) Design and deployment of performance oriented experiments that evaluate \gls{sm} systems under various environments \cref{chap:experimental-evaluation}.

% \item (\textit{Experimental}, \ref{rq-3}) Quantitative performance results and extensive analysis of \gls{sm} systems \cref{chap:experimental-evaluation}.