\section{Benchmarking Objectives}
\label{sec:system:objectives}

% What are we trying to achieve with this benchmarking instrument
% - Analyize introduced latency
% - Analyze overhead in system resources CPU/mem
% - Analyze the maximum throughput of the system
% - Analyze protocol differences

The goal of a benchmarking instrument is to objectively compare, and decide on the best systems in a specific domain. However, the concrete definition of the best system depends on the underlying benchmarking objective \cite{folkerts2012benchmarking}. Therefore, it is important to first establish a set of goals for the benchmarking before designing or implementing it. In this section, we will introduce the benchmarking object vices, related to the main research question \ref{rq-2}.


To design a benchmark that evaluates key characteristics of a system, we guide the process by following industry best practices for measuring and monitoring distributed systems. Following the best practices and principles of the Google Site Reliability Engineering handbook \cite{google-sre}, we can establish a base set of metrics and signals that are important to measure in distributed systems. More specifically, we can follow the guidelines for \textit{white-box monitoring} and implement the  \textit{Four Golden Signals} in our benchmarking instrument.

The following list presents the objectives that the benchmarking instrument has.

\begin{enumerate}[label=\textbf{O\arabic*}, leftmargin=3\parindent]
    \item \textbf{Support System Resource Measurements}
    \label{o-1}
    
    The first objective of the benchmarking instrument is to measure the amount of system resources a \gls{sm} system uses.
    
    \item \textbf{Support Throughput Measurements}
    \label{o-2}
    
    The second objective is to measure the amount of throughput a system can handle. This is used to see if the \gls{sm} system introduces any bottleneck in terms of throughput.

    \item \textbf{Support Latency Measurements}
    \label{o-3}
    
    The third objective is to measure the amount of latency requests have. This is used to measure how much latency a \gls{sm} system introduces.
    

    \item \textbf{Support Resiliency Measurements}
    \label{o-4}
    The final objective of the instrument is to measure the resiliency of the systems by measuring the error rates within the system.
    
\end{enumerate}
