\section{Summary}
\label{sec:system:summary}

To summarize and conclude this chapter, we return to the initial research question (\ref{rq-1}), \textit{How to design and implement a benchmark that evaluates the performance of \gls{sm} systems?} To design and build a benchmark that properly evaluates the performance characteristics of \gls{sm} systems we first used the learnings from our extensive system analysis as conducted in \cref{chap:survey}. Based on these learnings, we focussed on a sub set of components that were of interest to us and the performance implications they could have. Before starting the design phase, we established a set of benchmarking objectives based on industry best practices (\cref{sec:system:objectives}). After this, we clearly defined our \gls{sut} and separated our components of interest and purely functional components as to help with our design (\cref{sec:system:sut}). Subsequently, we then performed an extensive requirements analysis, in which we clearly defined the stakeholders and use cases (\cref{sec:system:requirements-analysis}). With the components, stakeholders and use cases clearly defined, we started our design process (\cref{sec:system:design}). After many reiterations to make the design more concise and modular, we finished the design of \textit{Mesh Bench}. Finally, after the design was completed, we implemented a prototype and evaluated the working prototype.

% (Conceptual, RQ2) Requirements analysis and design of a service mesh benchmarking instrument Chapter 4.
% 4. (Technical, RQ2) Prototype implementation of the designed benchmarking instrument, Mesh Bench Chapter 4