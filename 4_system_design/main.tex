% Setup
\graphicspath{./figures}

\chapter{Design and Implementation of a Service Mesh Benchmark}
\label{chap:system-design}

% Introduction
% - What will this chapter contain and what resaarch question it tries to answer

% Prev chapter -> General
% This chapter -> Performance oriented + importance
In the previous chapter (\cref{chap:survey}), we conducted a systems survey and identified several state-of-the-art \gls{sm} systems. We examined the architectural differences between these systems and compared them on domain-specific functional and non-functional requirements. We then discussed the theoretical performance implications of the architectural styles that we had identified. These performance implications have an effect on the performance of all software services in an environment and can cause a major bottleneck. With these concerns in mind, we dive deeper into the performance of \gls{sm} systems.


% Which RQ will we answer
% Related work -> Shortcomings
% What we do to solve these shortcomings
In this chapter, we provide an answer to research question \ref{rq-2}. \textit{How to design and implement a \gls{sm} benchmark which evaluates the performance of \gls{sm} systems?} During the systems survey, we observed that there was no relevant work found in the \gls{pl}. Furthermore, preliminary research has shown that previous and current efforts made to evaluate these performance characteristics are either in early stages or have limited functionality. To solve these shortcomings, we present a \gls{sm} benchmark instrument and implement a prototype that supports various workloads, application level protocols and \gls{sm} systems.

% Rest of the chapter is structured as follows
% - The need for these surveys (related work)
% - Methodology used for both surveys
% - Results of 
The remainder of this chapter is structured as follows. First, in \cref{sec:system:objectives} we introduce the goals of the benchmarking instrument. Second, in \cref{sec:system:sut} we describe the components that are part of the \gls{sut}. Afterwards, in \cref{sec:system:requirements-analysis} we present a requirement analysis in which we formalize the stakeholders, use cases and system requirements of the instrument. Next, in \cref{sec:system:design} we present the design of the benchmarking instrument. Following that, in \cref{sec:system:implementation} we discuss the implementation details of the system. Finally, in \cref{sec:system:analysis} we analyse the proposed design and implementation.

\input{4_system_design/sections/01_objectives}
\section{System Under Test}
\label{sec:system:sut}
% Explain SUT
% Components of Interest
% Purely Functional Components
% Present SUT

To design a benchmarking instrument that evaluates \gls{sm} systems, we first have to define the system and the components that are included in the benchmark. The components relevant to the benchmark are described in the \gls{sut}. However, even though the benchmark has to deal with numerous components in the system, not all are relevant to the benchmark.

Therefore, we split the components of the \gls{sut} in two categories according to the practices outlined by Folkerts et al. \cite{folkerts2012benchmarking}. The first group of components are labelled \textit{Components of Interest}. These components are of principal interest to the benchmark and have to be measured. The second group of components is labelled as \textit{Purely Functional Components}. The components in the latter group are present in the \gls{sut} in order for the benchmarking instrument to work correctly. However, measuring these components is not of importance relating to the objectives of the benchmark.

In \cref{fig:sut} we present the \gls{sut} for the benchmarking instrument. In the following sections, we elaborate on the  \textit{Components of Interest} (\cref{sec:system:sut:components-interest}) and \textit{Purely Functional Components} (\cref{sec:system:sut:components-functional})  present in this system.

\begin{figure}[!t]
    \centering
    
    \scalebox{.8}{    
    \includegraphics[width=\linewidth]{4_system_design/figures/system-under-test.pdf}
    }
    
    \caption{The \gls{sut} for benchmarking \gls{sm} systems.}
    
    \label{fig:sut}
\end{figure}
\todo{figure: move service proxy halfway between pod/nothingnes?}

\subsection{Components of Interest}
\label{sec:system:sut:components-interest}
% Elaborate on the components of interest
% Data Path of Traffic
% Pod containing the service
% - Application Pod
% - Service Mesh Proxy (data plane)

The \textit{Components of Interest} are aligned with the objectives as presented in \cref{sec:system:objectives}. In the context of a \gls{sm} system, we aim to measure the components present in the data path of a software service. Therefore, in the context of a \gls{k8s} cluster running a \gls{sm} we evaluate the performance of software services that live in the \gls{k8s} specific abstraction, the \gls{pod} \designref{1}.

To measure the service and the service traffic, we measure the application container \designref{2} and the data plane component of a \gls{sm} \designref{3}.

\subsection{Purely Functional Components}
\label{sec:system:sut:components-functional}
% Elaborate on the functional components
% Metric Services - Prom
% Control Plane components of sm
% - More detail?

To be able to run the benchmark, we require the several components. First, we run the benchmark on a \gls{k8s} cluster \designref{F1}. Furthermore, a \gls{sm} system introduces several control plane components, which are required for the service mesh to correctly function \designref{F2}. However, since our objectives are to measure the data path of a \gls{sm} they are not of importance to the benchmarking instrument. Additionally, we introduce a metrics and monitoring system in the cluster \designref{F3}, which will be used to collect and time series data of the pods and containers residing in the \gls{k8s} cluster.
\input{4_system_design/sections/03_requirements-analysis}
\section{Design of a Service Mesh Benchmarking Instrument}
\label{sec:system:design}

In this section, we present the design for \textit{Mesh Bench}, a benchmarking instrument for \gls{sm} systems. In \cref{fig:benchmark-design} we depict an overview of the benchmark and all of its components. This design shows how a user interacts with the benchmark, and how it can evaluate the components within the \gls{sut} (see \cref{sec:system:sut}). In the remainder of this section we will introduce the annotated components (as indicated by the black circles \designref{x}) and discuss the functionalities of them.

\begin{figure}[!t]
    \centering
     
    \includegraphics[width=0.9\linewidth]{4_system_design/figures/detailed-benchmark-design.pdf}
    
    \caption[Design of \textit{Mesh Bench}.]{A design overview of \textit{Mesh Bench}, a benchmark for \gls{sm} systems.}
    
    \label{fig:benchmark-design}
\end{figure}



\subsection{Benchmark Interface}
The \textit{Benchmark Interface} \designref{1} is the entry point for the end user and enables them to conduct performance experiments on \gls{sm} systems. The main goal of the Benchmark Interface is to manage experiment executions. The lifecycle of a single experiment consists of three phases. The first phase consists of initialization and configuration. This is done by parsing and processing an \textit{Experiment Configuration} \designref{2}. After this, it instructs the \textit{Workload Generator} \designref{3}, to generate experiment-specific workloads based on the supplied configuration. Once an experiment is finished and the workload generator has reported back to the benchmark interface, it finalizes the experiment and stores the obtained results \designref{8}.

\subsection{Experiment Configuration}
The benchmark consists of several performance experiments, as introduced in \cref{chap:experimental-evaluation}. Each of these experiments consists of several configuration parameters, which allows the user to change several aspects of the experiment. Notable configurations include the ability to modify the type of workload, the frequency of the workload, the duration of the experiment and what type of workload to use, and therefore which target service \designref{5}. These configuration parameters of a single experiment, forms the \textit{Experiment Configuration} \designref{2} and is the primary input of the benchmark interface \designref{1}.


\subsection{Workload Generator}
The \textit{Workload Generator} \designref{3} is used to generate synthetic workloads and measure the performance of the components within the \gls{sut}. This is arguably the most important component in the benchmark, and it has to support various modes of operation to match the functional requirements as defined in \cref{sec:system:requirements-analysis:functional}. The workload generator has to be able to perform load test experiments, in which the \textit{Request Generator} generates a certain application workload for a given \textit{Target Service} \designref{5}. It has to be able to do so in a constant throughput fashion, i.e., a fixed number of requests per second while also supporting a mode that enables us to test the maximum throughput of a given \gls{sut}. Another important aspect is that it should be configurable at run-time, so that the benchmark interface \designref{2} can configure it, allowing us to design various performance related experiments. This aspect is done through the \textit{REST API} as depicted in the benchmark design. The workload generator also exposes system utilization metrics through the \textbf{Metrics Exporter}.

\subsection{Proxy}
The \textit{Proxy} \designref{4} is the core component of a \gls{sm} architecture. The entire benchmark is designed to measure the performance implications of the proxy component. This component varies based on the \gls{sm} system that is evaluated (see \cref{sec:survey:analysis:sm-framework} for the proxies used per \gls{sm} system). Furthermore, it is unmodified, which means that it will intercept all the requests from and to the target service \designref{5}, and do the \textit{Request Processing} as originally intended. However, we do export the resource utilization metrics through the \textit{Metrics Exporter}.

\subsection{Target Services}
At the receiving end of the requests as generated by the workload generator are the \textit{Target Services} \designref{5}. The goal of this component is to mimic a generic service as encountered in a production environment. In this benchmark, we have two \textit{Application Services} to mimic different types of workloads. The first type of service accepts workloads through the \textit{HTTP endpoint}. The second type of service accepts workloads through its \textit{GRPC endpoint}. Both of the backing services accept the synthetic workloads, and perform minimal to no computations on it (\textit{Application Logic)}. This minimizes the impact on end-to-end latencies observed and keeps the resource overheads to a minimum. This allows us to focus on the performance implications of the proxy, instead of the backing services. Just like the other components in the benchmarking system, we export resource utilization metrics through the \textit{Metrics Exporter}.


\subsection{Monitoring System}
The \textit{Monitoring System} \designref{6} is responsible for monitoring the systems within the \textit{Kubernetes Cluster}. This is used primarily to collect resource utilization metrics for all the components within the system through the \textit{Metrics Aggregator}. The monitoring system has to support the collection of these metrics at a fine granularity. More specifically, it has to be able to distinguish resource utilization metrics at a pod and container granularity. This allows us to identify resource utilization patterns for the components of interest. A user should be able to query the aggregated metrics through a \textit{Query Engine} and output these results in a common format to stable storage in a common format \designref{7}.


\subsection{Load Test Results}
The \textit{Load Test Results} \designref{8} are the results created by the \textit{Workload Generator} \designref{3}. These results contain the performance results of an experiment. More specifically, it contains the request latencies as generated by the workload generator. Additionally, it comes with metadata such as the experimental configuration and environment data. 

\section{Implementation}
\label{sec:system:implementation}

In this section, we discuss the implementation details of a benchmark prototype that adheres to the design of \textit{Mesh Bench} as presented in \cref{sec:system:design}. The remainder of this section will discuss the design decisions for the core components in more detail. We present our implementation details, and the alternatives we had considered and how they relate to our requirements as discussed in \cref{sec:system:requirements-analysis}.

\subsection{Workload Generator}
\label{sec:system:workload-generator}

The workload generator is arguably the most important component of the benchmark and has to support a large amount of functional requirements as discussed in \cref{sec:system:requirements-analysis:functional}. 

Foremost, the workload generator has to support \textit{varying workloads} (\ref{system:fr-4}). This means that it has to support the two most common service-to-service communication protocols in the form of HTTP and gRPC (as derived from our system survey results in \cref{sec:survey:results:comparison:protocols}). Furthermore, to support a wide variety of experiments the workload generator would have to support varying workload sizes and payloads so that the user can specify and evaluate the impact it would have. In addition to this, it would be beneficial to us to be able to configure this on the fly. This would mean that the workload generator would be deployed as a server-side process, that can receive instructions from a client-side process. 

The final two functional requirements for the workload generator are related to the data it produces. First off, it would have to implement three of the four golden metrics as discussed in \ref{system:fr-3}. It would have to measure the end-to-end \textit{latency} of any request that it generates towards a target service. Additionally, it would have to measure the amount of \textit{traffic} at any given time, measured in the form of \textit{queries per second}. Finally, it would have to check for any \textit{errors}, this would mean it has to evaluate the HTTP and gRPC status codes to check if the request was successful. The last requirement has to deal with the format the data is presented in \ref{system:fr-6}. The resulting data would ultimately be reported back to the user to analyse the traffic and its performance.

\begin{table*}[t]
\centering
% \resizebox{\linewidth}{!}{ %< auto-adjusts font size to fill line

\begin{tabularx}{\textwidth}{ll|cc|cc|c}
\toprule

\multicolumn{2}{c|}{\textbf{Workload Generator}} &
\multicolumn{2}{c|}{\textbf{Workload Support}} &
\multicolumn{2}{c|}{\textbf{Configurability}} &
\multicolumn{1}{c}{\textbf{Other}}  \\

\textbf{ID} &
\textbf{Name} &
\textbf{HTTP/2} &
\textbf{GRPC} &
\textbf{Payload} & 
\textbf{Output} & 
\textbf{Server Process} \\
\midrule

\ref{wg-1} &
\textbf{wrk2} &
\circleE &
\circleE &
\circleH &
\circleH &
\circleE  \\

\ref{wg-2} &
\textbf{k6} &
\circleH &
\circleH &
\circleH &
\circleH &
\circleH \\

\ref{wg-3} &
\textbf{hey} &
\circleF &
\circleE &
\circleF &
\circleF &
\circleE \\

\ref{wg-4} &
\textbf{ghz} &
\circleE &
\circleF &
\circleF &
\circleF & 
\circleE \\

\ref{wg-5} &
\textbf{fortio} & 
\circleF &
\circleF &
\circleF & 
\circleF & 
\circleF \\


\bottomrule
\end{tabularx}

% } %< \resizebox

\caption[Comparing workload generator systems.]{Comparing workload generator systems on relevant properties and features. There are three different symbols in the table, each of them represents how well a requirement is satisfied. \circleF: Fully Satisfied, \circleE: Not Satisfied, \circleH: Partially Satisfied.}
\label{tab:system:implementation:workload-generators}
\end{table*}



To decide on a fitting workload generator we evaluated numerous systems as seen in \cref{tab:system:implementation:workload-generators}. We performed simple load tests with each of these systems and assessed their support for the relevant functional requirements. 


\begin{enumerate}[label=\textbf{WG\arabic*}, leftmargin=3\parindent]
   
    \item \textbf{wrk2}
    \label{wg-1}
    
    \textit{wrk2} was the first workload generator that we had considered as it was a battle-tested solution. It is an evolution of \textit{wrk} that can achieve highly accurate latency details (i.e., 99.9999\%'ile). It also consumed the least amount of system resources which would translate in the ability to evaluate systems under more load. Configurability was done through command line interface options, as most of the evaluated systems did. However, it did not support many modern features, most notably it did not support http/2 and only supported the older protocol versions. Additionally, it did not support reporting in common formats nor was it able to modify request payloads out of the box, but required Lua scripts to modify its behaviour. Furthermore, it would have to be paired with a load generator that would support the gRPC protocol as per the functional requirements, as it did not support that either.    

    \item \textbf{k6}
    \label{wg-2}
    
    \textit{k6} was another promising solution, as the programmable take on load testing would allow us to design and create any form of synthetic workload that our benchmark would require. Although this approach is great for complex load testing environments (i.e. those that requires specific flows such as authentication procedures for instance), it was less useful to perform relatively simple load testing experiments. The solution that required pre-programmed scripts that represent a load test scenario is a bit too convoluted for experiments that changed a singular variable. We also noted that k6 used a lot of system resources whilst generating workloads, that could potentially influence the outcome of load-test experiments if the load generator and target service operate on the same machine.

    \item \textbf{hey}
    \label{wg-3}    
    
    \textit{hey} is another load testing tool for HTTP-based workloads. It is a modern tool that supported all the required functionalities and was a prime candidate to act as a component within the workload generator.
    
    \item \textbf{ghz}
    \label{wg-4}
    
    \textit{ghz} is a benchmarking and load testing tool for gRPC based services. It matched all of our functional requirements and was considered for the implementation of the workload generator.
    
    \item \textbf{fortio}
    \label{wg-5}
    
    \textit{fortio}, however, was the implementation we ultimately decided upon using. As this solution was designed to evaluate the performance of services under various environments and conditions. It was able to generate both HTTP and gRPC based workloads, which allowed us to use a single tool for both types of workloads. Additionally, it supported a wide variety of configuration options, that allows us to modify payloads. A decisive feature, however, was that it was able to operate as a server-side process. This meant that we could run this process in the cluster as depicted in the benchmark design \cref{fig:benchmark-design} and instruct it from a separate component, controlled by the user.
\end{enumerate}



\subsection{Target Service}
\label{sec:system:target-service}

The workload generator generates requests to a target service, which has to process this request and reply afterwards. As previously mentioned, we want to minimize the impact the target service has on the end-to-end latencies as observed in the data path. To accomplish this we first developed simple services, that replied the content of the requests back to the requesting entity. 

However, fortio comes bundled with a set of testing services that we could use for our use case. This meant that we could instrument our benchmarking prototype with fortio as the load generator, and as target services. It comes bundled with an HTTP echo service that acts similarly as described above. In addition to the HTTP echo service, it comes with a simple gRPC based service that implements a common health checking protocol\footnote{\url{https://github.com/grpc/grpc/blob/master/doc/health-checking.md}}. 



\subsection{Monitoring System}
\label{sec:system:monitoring-system}

The workload driver allows us to capture and report metrics related to service requests. However, to measure the amount of system resources the components in our \gls{sut} use we have to implement a system that exports the related metrics in a fine granularity which can differentiate various container processes. In addition to that we need a system that can extract these metrics and aggregate them. Furthermore, the reported metrics should be stored as time-series data, so we can analyse the metrics at a given time in conjuncture with our experiments. Finally, the monitoring system should support extensive querying capabilities and the ability to export data in a common format \ref{system:fr-6}.

For the implementation of our monitoring system we decided to use \textit{Prometheus}\footnote{\url{https://prometheus.io/}}. This tool satisfies all the requirements and is a common cloud-native tool that matches our use case. As a matter of fact, it is so common within the \gls{k8s} ecosystem that most of the \gls{sm} systems as identified, provide support for this tool (see \cref{sec:survey:results:comparison:observability}).

To extract resource utilization metrics, we can make use of the \textit{cAdvisor}\footnote{\url{https://github.com/google/cadvisor}} daemon that is already embedded into a standard \gls{k8s} component (the \textit{Kubelet} process). cAdvisor collects, aggregates and processes resource utilization metrics and is aware of the isolation parameters of a container. By introducing Prometheus into our benchmark prototype, and configure it to scrape and collect the metrics exposed by cAdvisor, we can analyse experiment results and relate it to resource utilization data per pod and container.

Additionally, we use \textit{Grafana}\footnote{\url{https://grafana.com/}} in our prototype. This is another common, open-source tool that acts as an observability platform. It natively supports Prometheus and can be used to analyse and explore the collected data. Although we do not use it for our data visualizations as presented in the experimental evaluation (\cref{chap:experimental-evaluation}), we did use it to quickly validate and explore results of experiments.


% \subsection{Benchmark Interface}
% \label{sec:system:benchmark-interface}

% The prototype of the benchmark interface was constructed through several shell scripts and external programs. 
\section{Analysis}
\label{sec:system:analysis}

A benchmarking instrument is successful if all the stakeholders agree it provides meaningful results. To analyse if the proposed implementation satisfies this we use a set of guidelines established by Folkerts et al. \cite{folkerts2012benchmarking}. In their work they establish three groups of requirements which can be used to evaluate ea benchmarking instrument.

In the remainder of this section we introduce these three groups of requirements and evaluate them against the proposed benchmarking instrument. 

\subsection{General Requirements}
\label{sec:system:analysis:general}
% General Requirements – this group contains generic requirements.

The \textit{General Requirements} group establishes four broad requirements (\ref{gr-1} - \ref{gr-4}). The following list introduces those requirements and discusses how the proposed benchmarking instrument satisfies the conditions.

\begin{enumerate}[label=\textbf{GR\arabic*}, leftmargin=3\parindent]
    \item \textbf{Strong Target Audience}
    \label{gr-1}
    % The target audience must be
    % - considerable size and,
    % - interested to obtain the information.
    

    \item \textbf{Relevant}
    \label{gr-2}
    % The benchmark results have to measure the performance of the typical operation within the problem domain.
    

    \item \textbf{Economical}
    \label{gr-3}
    % The cost of running the benchmark should be affordable.
    

    \item \textbf{Simple}
    \label{gr-4}
    % The benchmark should be understandable, as it creates trust.
    
\end{enumerate}


\subsection{Implementation Requirements}
\label{sec:system:analysis:implementation}
% 2. Implementation Requirements – this group contains requirements regarding

The second group of requirements introduced in the work of Folkerts et al. is the \textit{Implementation Requirements} and establishes requirements (\ref{gr-1} - \ref{gr-4}). The following list introduces those requirements and discusses how the proposed benchmarking instrument satisfies the conditions.


\begin{enumerate}[label=\textbf{IR\arabic*}, leftmargin=3\parindent]
    \item \textbf{Fair and Portable}
    \label{ir-1}
    % All compared systems can participate equally.

    
    \item \textbf{Repeatable}
    \label{ir-2}
    % The benchmark results can be reproduced by rerunning the
    
    \item \textbf{Realistic and Comprehensive}
    \label{ir-3}
    % The workload exercises all SUT features typically used in the major classes of target application

    \item \textbf{Configurable}
    \label{ir-4}
    % a benchmark should provide a flexible performance analysis framework allowing users to configure and customize the workload.
\end{enumerate}


\subsection{Workload Requirements}
\label{sec:system:analysis:workload}
% 3. Workload Requirements – contains requirements regarding the workload definition and its interactions.
The final group of requirements introduced is the \textit{Workload Requirements} and consists of three requirements (\ref{wr-1} - \ref{wr-3}). The following list introduces those requirements and discusses how the proposed benchmarking instrument satisfies the conditions.

\begin{enumerate}[label=\textbf{WR\arabic*}, leftmargin=3\parindent]
    \item \textbf{Representativeness}
    \label{wr-1}
    % The benchmark should be based on a workload scenario that contains a representative set of interactions.
    
    \item \textbf{Scalable}
    \label{wr-2}
    % Scalability should be supported in a manner that preserves
    
    \item \textbf{Metric}
    \label{wr-3}
    % A meaningful and understandable metric is required to report about the SUT reactions to the load

\end{enumerate}
