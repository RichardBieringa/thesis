% Setup
\graphicspath{./figures}

\chapter{Design and Implementation of a Service Mesh Benchmark}
\label{chap:system-design}

% Introduction
% - What will this chapter contain and what resaarch question it tries to answer

% Prev chapter -> General
% This chapter -> Perforormance oriented + importance
In the previous chapter (\cref{chap:survey}), we conducted a systems survey and identified several state-of-the-art \gls{sm} systems. We examined the architectural differences between these systems and compared them on domain-specific functional and non-functional requirements. We observed that architectural differences can have various effects on the theoretical performance implications such systems can have. Furthermore, the performance implications have great effects as they affect all software services in the environment. Therefore, we dive deeper into the performance of \gls{sm} systems.


% Which RQ will we answer
% Related work -> Shortcomings
% What we do to solve these shortcomings
In this chapter, we provide an answer to research question \ref{rq-2}. \textit{How to design and implement a \gls{sm} benchmark which evaluates the performance of \gls{sm} systems?} During the literature survey, we observed that there was no relevant work found in the \gls{pl}. Furthermore, preliminary research has shown that previous and current efforts made to evaluate these performance characteristics are either in early stages or have limited functionality. To solve these shortcomings, we present an extensible \gls{sm} benchmark instrument that aims to support various workloads, protocols and \gls{sm} implementations.

% Rest of the chapter is structured as follows
% - The need for these surveys (related work)
% - Methodology used for both surveys
% - Results of 
The remainder of this chapter is structured as follows. First, in \cref{sec:system:objectives} we introduce the goals of the benchmarking instrument. Second, in \cref{sec:system:sut} we describe the components that are part of the \gls{sut}. Afterwards, in \cref{sec:system:requirements-analysis} we present a requirement analysis in which we formalize the stakeholders, use cases and system requirements of the instrument. Next, in \cref{sec:system:design} we present the design of the benchmarking instrument. Following that, in \cref{sec:system:analysis} we analyse the proposed design.  Finally, in \cref{sec:system:implementation} we discuss the implementation details of the system.

\section{Benchmarking Objectives}
\label{sec:system:objectives}

% What are we trying to achieve with this benchmarking instrument
% - Analyize introduced latency
% - Analyze overhead in system resources CPU/mem
% - Analyze the maximum throughput of the system
% - Analyze protocol differences

The goal of a benchmarking instrument is to objectively compare, and decide on the best systems in a specific domain. However, the concrete definition of the best system depends on the underlying benchmarking objective \cite{folkerts2012benchmarking}. Therefore, it is important to first establish a set of goals for the benchmarking before designing or implementing it. In this section, we will introduce the benchmarking object vices, related to the main research question \ref{rq-2}.


To design a benchmark that evaluates key characteristics of a system, we guide the process by following industry best practices for measuring and monitoring distributed systems. Following the best practices and principles of the Google Site Reliability Engineering handbook \cite{google-sre}, we can establish a base set of metrics and signals that are important to measure in distributed systems. More specifically, we can follow the guidelines for \textit{white-box monitoring} and implement the  \textit{Four Golden Signals} in our benchmarking instrument.

The following list presents the objectives that the benchmarking instrument has.

\begin{enumerate}[label=\textbf{O\arabic*}, leftmargin=3\parindent]
    \item \textbf{Support System Resource Measurements}
    \label{o-1}
    
    The first objective of the benchmarking instrument is to measure the amount of system resources a \gls{sm} system uses.
    
    \item \textbf{Support Throughput Measurements}
    \label{o-2}
    
    The second objective is to measure the amount of throughput a system can handle. This is used to see if the \gls{sm} system introduces any bottleneck in terms of throughput.

    \item \textbf{Support Latency Measurements}
    \label{o-3}
    
    The third objective is to measure the amount of latency requests have. This is used to measure how much latency a \gls{sm} system introduces.
    

    \item \textbf{Support Resiliency Measurements}
    \label{o-4}
    The final objective of the instrument is to measure the resiliency of the systems by measuring the error rates within the system.
    
\end{enumerate}

\input{4_system_design/sections/system-under-test}
\section{Requirements Analysis}
\label{sec:system:requirements-analysis}

In this section, we expand on the details and requirements of the benchmarking instrument. To properly define the requirements of the benchmark, we first have to define the stakeholders and their use cases for such an instrument. After that is defined, we can define the actual requirements of the instrument based on those.

In \cref{sec:system:requirements-analysis:stakeholders} we define the stakeholders. Afterwards, in \cref{sec:system:requirements-analysis:use-cases} we define the use cases for the benchmarking instrument. Finally, in \cref{sec:system:requirements-analysis:requirements} we define the actual requirements.


\subsection{Stakeholders}
\label{sec:system:requirements-analysis:stakeholders}
% Who are the stakeholders?
% End users (cluster operators)
% Developers of SM platforms
% Scientific users

To uncover the use cases of this benchmarking instrument, we first have to identify the primary stakeholders. This is because stakeholders can have different applications for such a system, and therefore have different requirements. The following list of stakeholders (\ref{sh-1}-\ref{sh-3}) introduces the identified stakeholders and their concerns.

\begin{enumerate}[label=\textbf{S\arabic*}, leftmargin=3\parindent]
    \item \textbf{DevOps Engineers}
    \label{sh-1}
    
    This group of stakeholders is the primary user of a \gls{sm} system. They manage and control the infrastructure applications and services run on, and therefore can consider implementing a \gls{sm} system. Their primary concern with the benchmarking instrument is to evaluate the performance and resiliency characteristics of a \gls{sm} implementation on existing infrastructure or compare different implementations to make an informed decision to choose between the various implementations. For this group of stakeholders, it is important that the \gls{sm} matches their goals and requirement. This is exemplified by a DevOps engineer bound to predetermined \glspl{sla} that define contracts on service latencies.
    
    \item \textbf{Service Mesh Engineers}
    \label{sh-2}
    
    The second group of stakeholders consists of the engineers that develop the \gls{sm} systems itself. For these engineers, it is important to track the performance of their system to make sure that it meets the predetermined performance requirements. Furthermore, it can enable the engineers to compare their system to other offerings in the \gls{sm} landscape, which can help this group to set expectations and requirements.
    
    \item \textbf{Scientific Researchers}
    \label{sh-3}
    
    The final group of stakeholders represents researchers from academic institutions. This group uses the benchmarking instrument to conduct scientific research on the \gls{sm} systems, and their workloads. For this group of stakeholders, it is important that the benchmark instrument reports detailed information on both the system and the workloads running on it so that they can reason on the results.  Furthermore, it is important that the results the instrument produces are reproducible, meaning that experiments conducted with similar inputs produce similar outputs (bearing in mind the inherent variability in distributed environments). Finally, it is important that the instrument can run different workloads so that they can evaluate different situations and environments.
\end{enumerate}


\subsection{Use cases}
\label{sec:system:requirements-analysis:use-cases}
% What can this system be used for?
% https://www.similarweb.com/corp/blog/research/business-benchmarking/benchmarking-types/ -> Business benchmark translated to software
%  Competitive benchmarking –> See how one system compares to another
% Strategic benchmarking -> To look at the best performing system (in a category) to learn from
% Performance benchmarking -> Set targets of a system and evaluate over time
%  Validate Perforamnce Optimizations

Previously, we have identified the stakeholders of a benchmarking instrument, (\ref{sh-1}-\ref{sh-3}) as it is important to know the users and their concerns. This helps us to understand \textit{who} the users of the benchmark are, and \textit{why} they would use such an instrument.

To understand \textit{how} such a benchmark is used, we need to identify the use cases. Drawn from the concerns of the stakeholders, we present a list of use cases that (\ref{u-1}-\ref{u-3}) this benchmarking instrument can fulfil.

\begin{enumerate}[label=\textbf{UC\arabic*}, leftmargin=3\parindent]
    \item \textbf{Competitive Benchmarking}, \textit{concerns:} \ref{sh-1}, \ref{sh-2}, \ref{sh-3}
    \label{u-1}
    % Basic performance evaluation on multiple systems
    % Compare systems -> Which system is the best
    % Applies to: All
    
    The first and most common use case of the benchmarking instrument is to perform competitive benchmarking. The benchmarking instrument produces quantitative data on the performance characteristics of a \gls{sm} system.  The results of the benchmark can then be used to compare \gls{sm} implementations and decide on the best performing system in a domain.
    
    \item \textbf{Strategic Benchmarking}, \textit{concerns:} \ref{sh-2}, \ref{sh-3}
    \label{u-2}
    % Look at best system in category x
    % Look at the implementation details of that system (regarding category x)
    % Learn from it -> Apply
    % Applies to: Engineers, Academia
    
    \todo{Bench use case: mb rename strategic?}
    
    A second use case of the benchmarking instrument is to enable critical insights into the best performing systems. A benchmarking instrument gives insights into the performance of various characteristics of a system. This allows users to analyse the best performing systems in certain individual areas. Analysing those systems in further detail allows users to develop an understanding on how these systems achieved their results and allows them to learn from it. These critical insights can then be translated into other systems applying these best practices.
    
    \item \textbf{Evaluate Periodic Performance}, \textit{concerns:} \ref{sh-1}, \ref{sh-2}, \ref{sh-3}
    \label{u-3}
    % Compare the performance of a single system over time
    % Applies to: All
    
    A third use case for the benchmarking instrument is to capture the performance of a system over time. When conducting a benchmark, you evaluate a system and capture the performance of that system at a specific time. However, it can also be used to monitor and evaluate the performance of a system over a larger time frame to determine if, and by how much, the performance characteristics alter over time.
    
    
    \item \textbf{Validation of Performance Optimizations}, \textit{concerns:} \ref{sh-2}, \ref{sh-3}
    \label{u-4}
    % Developers -> Test changes to a sm system
    % See if perforamnce differs over versions
    % Applies to: Engineers, Academia

    Another use case for this benchmarking instrument is to analyse the impact of performance optimizations applied to \gls{sm} systems. When such an optimization is made, users can evaluate the system pre- and post-optimization to measure the impact of the optimization.
    

    \item \textbf{Validate System Requirements}, \textit{concerns:} \ref{sh-1}
    \label{u-5}
    % Evaluate if a sm implementation could be used in the environment oof the end user
    % e.g. SLA (min latency
    % Applies to: DevOps (end user)
    
    A final use case for this benchmarking instrument, is to evaluate if a \gls{sm} meets the requirements of the end user. End users can have different requirements, such as performance and resiliency requirements. This benchmark will enable those users to evaluate a \gls{sm} on their infrastructure to see if the system meets their requirements.
    
\end{enumerate}

\subsection{Requirements}
\label{sec:system:requirements-analysis:requirements}
% Introduce the requirements of the system
% Based on Folkters et al. three types of requirements
% - General Requirements
% - Implementation Requirements
% - Workload requirements

To design a benchmarking instrument that resembles any value, we satisfy the concerns and use cases of the identified stakeholders. To ensure that we satisfy this, we formulate a set of requirements, which we present in the following sections.

\subsubsection{Functional Requirements}
\label{sec:system:requirements-analysis:functional}

Functional requirements are requirements that the system \textit{must} implement, to function correctly. The following list (\ref{system:fr-1} - \ref{system:fr-7}) introduces the functional requirements of the benchmarking instrument and to which use case they relate.

\begin{enumerate}[label=\textbf{FR\arabic*}, leftmargin=3\parindent]
    \item \textbf{Support the most popular \gls{sm} implementations.}, \textit{related to:} \ref{u-1}, \ref{u-3}
    \label{system:fr-1}
    
    The benchmarking instrument must support the most popular \gls{sm} implementations. During the systems survey (\cref{chap:survey}), we identified that two of the identified \gls{sm} implementations currently dominate the landscape in terms of usage \cite{cncf-survey-2021}. Therefore, the benchmarking instrument has to support both \textit{Istio} and \textit{Linkerd} to be relevant for most users. 
    
    \item \textbf{Support systems using different data plane architectures}, \textit{related to:} \ref{u3}
    \label{system:fr-2}
    
    The benchmarking instrument must support all the identified data plane architectures (\cref{sec:survey:analysis:architectures}). During the systems survey, we identified four different data plane architectures with various performance implications. To analyse this, the benchmark must support at least one of the identified systems for each identified data plane architecture. The system already has to support systems adhering to the \textit{per-service} through the first established functional requirement (\ref{system:fr-1}). Since the other architectures only relate to a single \gls{sm} implementation, it means that the system must additionally support  \textit{Cilium} and \textit{Traefik}.
    
    \item \textbf{Implement and Report the Four Golden Signals}, \textit{related to:} \ref{u-1}, \ref{u-2}, \ref{u-3} 
    \label{system:fr-3}
    
    The benchmarking instrument must implement the \textit{Four Golden Signals}. To report meaningful data, we have to define meaningful metrics that the benchmarking instrument will report. To achieve this, we define the metrics by following industry best practices as established in the Google SRE handbook \cite{google-sre} and implement the \textit{Four Golden Signals}. This means that the benchmarking instrument must implement relevant metrics to report the \textit{Latency}, \textit{Traffic}, \textit{Errors}, and, \textit{Saturation} of the \gls{sut}.
    
    \item \textbf{Support Varying Workloads}, \textit{related to:} \ref{u-1}, \ref{u-3} 
    \label{system:fr-4}
    
    The benchmarking instrument must support varying workloads. This allows the benchmark to be relevant for the different use cases it can encounter and simulate the different workloads a \gls{sm} can see in the real world. To support this, the benchmark must provide a mechanism to specify the workload that the benchmark will run. Additionally, it has to provide a way in which the user can specify how the benchmark can extract relevant metrics. This is done by specifying how the workload, or service, can be consumed (e.g. HTTP endpoints).
    
    \item \textbf{Support Varying Cluster Environments}, \textit{related to:} \ref{u-1}, \ref{u-2}, \ref{u-3} 
    \label{system:fr-5}
    
    The benchmarking instrument must support varying cluster environments. This means that the benchmark must support any valid \gls{k8s} cluster environment. This allows users to evaluate \gls{sm} implementations on within their own environment and enables environment-specific use cases such as \ref{u-5}.
    
    \item \textbf{Report results in a well-known data format}, \textit{related to:} \ref{u-1}, \ref{u-2}, \ref{u-3} 
    \label{system:fr-6}
    
    The benchmarking instrument must report its results in a well-known data format. In order for the results to be useful to the various stakeholders, data must be presented in well-known formats such as \textit{JSON}, \textit{CSV} or \textit{YAML}.
    
    
    \item \textbf{Produce Reproducible Results}, \textit{related to:} \ref{u-1}, \ref{u-2}, \ref{u-3} 
    \label{system:fr-7}
    
    The benchmarking instrument must produce reproducible results. To provide meaningful results, the instrument has to be able to produce similar results when running a single experiment in the same environment. If the instrument is given the same inputs, similar outputs have to be expected.
    
\end{enumerate}


\section{Design of a Service Mesh Benchmarking Instrument}
\label{sec:system:design}

\section{Analysis of System Survey}
\label{sec:survey:analysis}


In the previous section (\cref{sec:survey:results}), we conducted a data synthesis process on the data we obtained during the systems survey. This process resulted in domain-specific characteristics of identified service mesh systems. In this section, we continue with the data and analyse and discuss notable differences we discovered during the data synthesis process.

The remainder of this section is structured as follows. In \cref{sec:survey:analysis:sm-framework}, we provide a qualitative comparison of the service mesh systems and present several key findings from the obtained data. After that, in \cref{sec:survey:analysis:architectures}, we dive into the different architectures and how they can relate to the observed behaviour. Finally, in \cref{sec:survey:analysis:conclusion}, we provide an answer to \ref{rq-1} and conclude the systems survey.


\subsection{Qualitative Comparison of Service Mesh Systems}
\label{sec:survey:analysis:sm-framework}
% Display resulting framework
% Explain per FR, NFR how it is measured
% Introduction to how it might relate to architecture
% Observations
% - Architecture -> Most Per-Service
% - Traefik Mesh -> No MTLS
% - Three systems fully hit all functional requirements
% - Linkerd2 -> Different Focus

In \cref{tab:result-comparison}, we present a qualitative evaluation of state-of-the-art service mesh systems. In this comparison, we present the architectural style the service proxy uses and map the identified requirements and their level of satisfaction (\cref{sec:survey:results:sm-requirements} to each individual system. 

\begin{table}[!t]

\centering
\resizebox{\linewidth}{!}{ %< auto-adjusts font size to fill line
\begin{tabular}{c|cc|cccc|ccccc}
\toprule

% HEADER 1
\multicolumn{1}{c|}{} &
\multicolumn{2}{c|}{\textbf{Data Plane}} &
\multicolumn{4}{c|}{\textbf{Functional Requirements}} &
\multicolumn{5}{c}{\textbf{Non-Functional Requirements}}  \\

% HEADER 2
\multicolumn{1}{c|}{\textbf{Service Mesh}} &
\multicolumn{1}{c}{\textbf{Service Proxy}}  &
\multicolumn{1}{c|}{\textbf{Proxy Architecture}}  &
\multicolumn{1}{c}{\ref{fr-1}}  &
\multicolumn{1}{c}{\ref{fr-2}}  &
\multicolumn{1}{c}{\ref{fr-3}}  &
\multicolumn{1}{c|}{\ref{fr-4}} &

\multicolumn{1}{c}{\ref{nfr-1}} &
\multicolumn{1}{c}{\ref{nfr-2}} &
\multicolumn{1}{c}{\ref{nfr-3}} &
\multicolumn{1}{c}{\ref{nfr-4}} &
\multicolumn{1}{c}{\ref{nfr-5}} \\

\midrule
% CONTENT

AWS App Mesh
& Envoy         % Service Proxy
& Per-Service   % Proxy Architecture
& \circleF      % FR1 Observability
& \circleF      % FR2 Security
& \circleH      % FR3 Resilience
& \circleF      % FR4 Additional Deployment Models

& \circleH      % NFR1 Application Level Protocol Aware
& \circleE      % NFR2 Open-Source
& \circleF      % NFR3 Documentation
& \circleE      % NFR4 CNCF Level
& \circleE      % NFR5 Commmunity Recognition
\\

Cilium
& Cilium eBPF   % Service Proxy
& In Kernel     % Proxy Architecture
& \circleF      % FR1 Observability
& \circleF      % FR2 Security
& \circleF      % FR3 Resilience
& \circleF      % FR4 Additional Deployment Models

& \circleF      % NFR1 Application Level Protocol Aware
& \circleF      % NFR2 Open-Source
& \circleH      % NFR3 Documentation
& \circleH      % NFR4 CNCF Level
& \circleH      % NFR5 Commmunity Recognition
\\

Consul
& Envoy         % Service Proxy
& Per-Service   % Proxy Architecture
& \circleF      % FR1 Observability
& \circleF      % FR2 Security
& \circleH      % FR3 Resilience
& \circleF      % FR4 Additional Deployment Models

& \circleH      % NFR1 Application Level Protocol Aware
& \circleF      % NFR2 Open-Source
& \circleF      % NFR3 Documentation
& \circleE      % NFR4 CNCF Level
& \circleH      % NFR5 Commmunity Recognition
\\

Ease Mesh
& EaseGress     % Service Proxy
& Per-Service   % Proxy Architecture
& \circleF      % FR1 Observability
& \circleE      % FR2 Security
& \circleH      % FR3 Resilience
& \circleE      % FR4 Additional Deployment Models

& \circleH      % NFR1 Application Level Protocol Aware
& \circleF      % NFR2 Open-Source
& \circleE      % NFR3 Documentation
& \circleE      % NFR4 CNCF Level
& \circleE      % NFR5 Commmunity Recognition
\\

Istio
& Envoy         % Service Proxy
& Per-Service   % Proxy Architecture
& \circleF      % FR1 Observability
& \circleF      % FR2 Security
& \circleF      % FR3 Resilience
& \circleF      % FR4 Additional Deployment Models

& \circleF      % NFR1 Application Level Protocol Aware
& \circleF      % NFR2 Open-Source
& \circleF      % NFR3 Documentation
& \circleE      % NFR4 CNCF Level
& \circleF      % NFR5 Commmunity Recognition
\\

Kuma
& Envoy         % Service Proxy
& Per-Service   % Proxy Architecture
& \circleF      % FR1 Observability
& \circleF      % FR2 Security
& \circleF      % FR3 Resilience
& \circleF      % FR4 Additional Deployment Models

& \circleF      % NFR1 Application Level Protocol Aware
& \circleF      % NFR2 Open-Source
& \circleF      % NFR3 Documentation
& \circleH      % NFR4 CNCF Level
& \circleE      % NFR5 Commmunity Recognition
\\

Linkerd2
& Linkerd2 Proxy% Service Proxy
& Per-Service   % Proxy Architecture
& \circleF      % FR1 Observability
& \circleF      % FR2 Security
& \circleH      % FR3 Resilience
& \circleF      % FR4 Additional Deployment Models

& \circleH      % NFR1 Application Level Protocol Aware
& \circleF      % NFR2 Open-Source
& \circleF      % NFR3 Documentation
& \circleF      % NFR4 CNCF Level
& \circleF      % NFR5 Commmunity Recognition
\\

Nginx Service Mesh
& NGINX Plus    % Service Proxy
& Per-Service   % Proxy Architecture
& \circleF      % FR1 Observability
& \circleF      % FR2 Security
& \circleH      % FR3 Resilience
& \circleE      % FR4 Additional Deployment Models

& \circleH      % NFR1 Application Level Protocol Aware
& \circleE      % NFR2 Open-Source
& \circleH      % NFR3 Documentation
& \circleE      % NFR4 CNCF Level
& \circleE      % NFR5 Commmunity Recognition
\\

Open Service Mesh
& Envoy         % Service Proxy
& Per-Service   % Proxy Architecture
& \circleF      % FR1 Observability
& \circleF      % FR2 Security
& \circleE      % FR3 Resilience
& \circleE      % FR4 Additional Deployment Models

& \circleH      % NFR1 Application Level Protocol Aware
& \circleF      % NFR2 Open-Source
& \circleH      % NFR3 Documentation
& \circleH      % NFR4 CNCF Level
& \circleE      % NFR5 Commmunity Recognition
\\

Traefik Mesh
& Traefik Proxy % Service Proxy
& Per-Node      % Proxy Architecture
& \circleF      % FR1 Observability
& \circleE      % FR2 Security
& \circleH      % FR3 Resilience
& \circleE      % FR4 Additional Deployment Models

& \circleH      % NFR1 Application Level Protocol Aware
& \circleF      % NFR2 Open-Source
& \circleH      % NFR3 Documentation
& \circleE      % NFR4 CNCF Level
& \circleE      % NFR5 Commmunity Recognition
\\
\bottomrule
\end{tabular}
} %< \resizebox

\caption[Qualitative comparison between state-of-the-art \gls{sm} systems.]{Qualitative comparison between state-of-the-art \gls{sm} systems. \\
There are three different symbols in the table, each of them represents how well a requirement is satisfied. \circleF: Fully Satisfied, \circleE: Not Satisfied, \circleH: Partially Satisfied.}
\label{tab:result-comparison}

\end{table}

\begin{enumerate}[label=\textbf{F\arabic*}, leftmargin=3\parindent]
    \item \textbf{Most of the identified \gls{sm} systems share a similar data plane architecture.}
    \label{f-1}
    % Most use a per-service proxy
    % Most use envoy
    
    The first finding from the obtained data is that most of the systems use a similar architecture for the data plane of the \gls{sm}. Out of the ten identified \gls{sm} implementations, eight used a per-service architecture for the service proxy. Furthermore, out of those eight using a per-service architecture, half of them used the same service proxy implementation in the form of \textit{Envoy}, an open-source general purpose proxy.
    
    \item \textbf{Per-node data plane architecture prevents support for common security features.}
    \label{f-2}
    % Traefik uses per-node
    % Is the only one that does not support the security requirements
    
    The second finding is that only a single identified system used a service proxy architecture, where the service proxy was implemented on a per-node granularity. \textit{Traefik Mesh} is the identified service mesh that used such an architecture, and most notably it was the only implementation that did not support any of the security capabilities listed (\ref{fr-2}). In particular, it does not support automatic mutual TLS encryption between services, a killer feature to enhance the security in any environment. To further elaborate on this, we dive deeper into the characteristics of data plane architectures in the next section (\cref{sec:survey:analysis:architectures}).

    \item \textbf{Most identified systems do not support all functional requirements.}
    \label{f-3}
    % 3/10 implementations support all requirements
    % Cilium, Istio and Kuma
    % Cilium -> Network focus
    % Istio -> Feature focus + Mature + Complex
    % Kuma -> Feature focus, Extensibility, Support for different environments
    
    The third finding is that out of all ten identified service mesh systems, just three systems fully satisfy all the functional requirements (\ref{fr-1}-\ref{fr-4}), namely, \textit{Cilium}, \textit{Istio} and \textit{Kuma}. This finding can be explained by taking a closer look at the maturity levels and the goals that the individual \gls{sm} implementations have. The three identified services that achieve all functional requirements are mature projects or heavily emphasize their feature set and extensibility. \textit{Istio} is the most mature platform, with large backing and is the most used \gls{sm} implementation out of all identified systems according to the \gls{cncf} survey conducted in 2021 \cite{cncf-survey-2021}. It also supports the most features and therefore hits all the functional requirements. \textit{Kuma} on the other hand, is a much smaller project. However, its focus lies on having a broad feature set with lots of extensibility options as well, thus achieving all functional requirements as well. \textit{Cilium} on the other hand, is a new entry in the \gls{sm} landscape by looking at the non-functional attributes (\cref{tab:result-nfa}). However, this story also does not paint the entirety of the picture as it builds upon the \textit{Cilium} networking solution, a mature networking solution that facilitates most of the functional requirements.
    

    \item \textbf{Maturity does not translate to support of requirements.}
    \label{f-4}
    % Linkerd2 Does not hit all requirements
    % Different project focus
    % - Simple
    % - Fast
    % - Tailored proxy, less features
    
    A fourth finding is that maturity does not necessarily translate to the support of functional and non-functional requirements. This refers to \textit{Linkerd2} in particular, which does not  not fully satisfy the resiliency requirements \ref{fr-3} and also lacking in the additional application level protocol support \ref{nfr-1}. Despite its maturity, large-scale usage \cite{cncf-survey-2021} in production environments and graduated \gls{cncf} project status, the system seems to be lacking in features. This, however, can be attributed to the goals of the project. Whereas \textit{Istio} has the most features of any system, it is also dubbed as a complex system. \textit{Linkerd2}, however, aims to optimize for simplicity and speed. Instead of using a generic proxy with numerous features and capabilities, such as \textit{Envoy} or \textit{NGINX}, it uses a custom proxy specifically tailored to the service mesh, optimizing for simplicity, security and speed \cite{linkerd-no-envoy}.
    
    \item \textbf{Lack of additional layer 7 protocol support.}
    \label{f-5}
    % 4/10 systems support 'additional' layer 7 protocols
    % Why it is useful
    % External project (Aeraki) -> Tries to solve this
    
    A fifth finding is that just a few of the identified systems support additional application level protocols (\ref{nfr-1}).  Since \gls{sm} systems can greatly benefit from application aware proxies, it can be a promising area. It can enable application level aware traffic management, rich metric collection or can lead to advanced authorization policies. \textit{Cilium} and \textit{Kuma} for example, provide support for the Apache Kafka protocol and therefore can target specific messages when dealing with events of the platform.  Although we did identify only a few implementations that provided support for this, we did identify an external project in the \gls{cncf} landscape that tries to enable this. \textit{Aeraki} \footnote{\url{https://www.aeraki.net/}}, attempts to close this gap by creating an extensible platform that enables layer 7 support for many protocols such as \textit{Redis}, \textit{Kafka} and \textit{ZooKeeper}. This project, however, only provides support for \textit{Istio} and therefore currently does not support any other \gls{sm} implementation.
    
    \item \textbf{New technologies enable additional data plane architectures.}
    \label{f-6}
    % Single sm uses kernel based proxy (Cilium)
    % - Does not have linear dependency on proxy/service
    % - Relatively new techonology
    
    A final finding is that only a single project uses a kernel-based proxy. \textit{Cilium} uses a kernel-based proxy approach by utilizing \gls{ebpf} applications to implement the service proxy. Together with \textit{Traefik proxy}, it is the only \gls{sm} that does not use a per-service based data plane architecture. We can attribute this finding to the fact that \gls{ebpf} is a relatively new technology, and that the \textit{Cilium} networking project has been banking on that technology before extending its networking solution to include the properties of a service mesh \cite{cilium-mesh} in late 2021. Although this is a fairly new concept, it has been adopted by one of the leading \gls{k8s} service provider in the form of \textit{Google Kubernetes Engine} \cite{google-cilium-ebpf}. Furthermore, other \gls{sm} implementations also experiment with the technology to enable new functionality and features, such as reported for both \textit{Istio} \cite{istio-merbridge} and certain \textit{NGINX Service Mesh} beta features \cite{nginx-service-mesh-arch}. 
\end{enumerate}


% The first observation we can make from the obtained data is that most of the systems use a data plane with a per-service proxy. However, some \gls{sm} systems use a different approach, such as \textit{Traefik Mesh}, which uses a per-node proxy and \textit{Cilium}, which uses a kernel-based proxy. This can be attributed to the fact that half of the implementations use \textit{Envoy} as proxy, which is an open-source proxy. 

% The second observation that we can make is that \textit{Traefik Mesh} does not support any of the security capabilities listed (\ref{fr-2}). Notably, it does not support automatic mutual TLS encryption between services, a killer feature to enhance the security in any environment. This result, however, is unsurprising if we relate it to the first observation. Due to its per-node proxy, all service-to-service communications travel first from a service to the proxy in a potentially unencrypted state. This allows for potential attackers to intercept this, and thus to create a zero-trust environment application developers have to implement mutual TLS connections manually.

% The third observation is that there are three systems that fully satisfy all the functional requirements (\ref{fr-1}-\ref{fr-4}), namely, \textit{Cilium}, \textit{Istio} and \textit{Kuma}. The first of those systems has a heavy focus on networking and features, as it originates as a networking solution and transcended into the service mesh space since December 2021 \cite{cilium-ebpf-mesh}. The second of those systems can be considered the most mature \gls{sm} system, which sees the most amount of production use according to the \gls{cncf} 2021 survey \cite{cncf-survey-2021} and has enterprise backing. Kuma, on the other hand, is a project with much smaller backing and usage, however it has a focus on extensibility and features, even allowing the system in non-\gls{k8s} or vm-based environments. 

% A fourth observation is that \textit{Linkerd2} does not hit all the functional requirements as it does not fully satisfy the resiliency requirements \ref{fr-3} and also lacking in the additional application level protocol support \ref{nfr-1}. Despite its maturity, large-scale usage in production environments and graduated \gls{cncf} project status, the system seems to be lacking in features. This, however, can be attributed to the goals of the project. Whereas \textit{Istio} has the most features of any system, it is also dubbed as a complex system. \textit{Linkerd2}, however, aims to optimize for simplicity and speed. Instead of using a generic proxy with numerous features and capabilities, such as \textit{Envoy} or \textit{NGINX}, it uses a custom proxy specifically tailored to the service mesh, optimizing for simplicity, security and speed \cite{linkerd-no-envoy}.

% A fifth observation is that additional application level protocol support \ref{nfr-1} exists, however, not many systems have support for it out of the box. Since \gls{sm} systems can greatly benefit from application aware proxies, it can be a promising area. It can enable application level aware traffic management, rich metric collection or can lead to advanced authorization policies. Some proxies like \textit{Cilium} and \textit{Kuma} provide support for protocols such as \textit{Kafka}, in which they can target specific messages. Another project in the \gls{cncf} landscape under the name of \textit{Aeraki} \footnote{\url{https://www.aeraki.net/}}, attempts to close this gap by creating an extensible platform that enables layer 7 support for many protocols such as \textit{Redis}, \textit{Kafka} and \textit{ZooKeeper}. This project, however, only provides support for \textit{Istio} and therefore currently does not support any other \gls{sm} implementation.

% A final observation is that only a single project uses a kernel-based proxy. \textit{Cilium} uses a kernel-based proxy approach by utilizing \gls{ebpf} technology. Together with \textit{Traefik proxy}, it is the only \gls{sm} that does not use a per-service based data plane architecture. The reason for this is that \gls{ebpf} the technology is relatively new, and \textit{Cilium} has been banking on that technology before extending its networking solution to include the properties of a service mesh \cite{cilium-mesh}. Although this is a fairly new concept, it has been adopted by one of the leading \gls{k8s} service provider in the form of \textit{Google Kubernetes Engine} \cite{google-cilium-ebpf}. Furthermore, it is also used for a bleeding edge \textit{Istio} feature \cite{istio-merbridge} and in certain \textit{NGINX Service Mesh} beta features \cite{nginx-service-mesh-arch}.


\subsection{Analysis of Common Service Mesh Architectures}
\label{sec:survey:analysis:architectures}
% Introduce section
% Per-Service Arch
% Per-Node Arch
% Kernel

In this section, we analyse the identified \gls{sm} architectures in more detail. For every identified architecture, we present a reference topology and discuss the advantages and disadvantages one such implementation has. First, in \cref{sec:survey:analysis:architectures:per-service}, we discuss the most common identified approach, a \gls{sm} using a per-service proxy. Next, in \cref{sec:survey:analysis:architectures:per-node}, we discuss architectures leveraging a per-node proxy. Finally, in \cref{sec:survey:analysis:architectures:ebpf}, we discuss architectures using a kernel-based \gls{ebpf} approach.


\subsubsection{Per-Service Proxy}
\label{sec:survey:analysis:architectures:per-service}
% - Simplest architecture
% - Overhead, many proxies
% - Resource Usage
% - Latency / Hops

During the systems survey, we identified ten different \gls{sm} systems. Out of those ten systems, eight of those shared similar architectural characteristics. For all of these systems, we identified that the mesh network was established by introducing a proxy for every individual software service present. This architectural style is so common within the \gls{k8s} ecosystem, that the pattern is often referred to as the \textit{sidecar pattern}.


In \cref{fig:sm-arch-per-service}, we present a reference architecture of this design pattern for service meshes in a \gls{k8s} cluster. This shows the data path of a packet throughout its lifespan in a system that implements this architecture. To explain the data path of a packet, we first introduce the individual components depicted, as most of them are present in the other architectures as well. 

\begin{figure}[!t]
    \centering
    
    \scalebox{.8}{
    \includegraphics[width=\linewidth]{3_systems_survey/figures/sm-arch-per-service.pdf}
    }

    \caption{A service mesh using a service proxy per-service.}
    \label{fig:sm-arch-per-service}
\end{figure}

% Basic components introduction
First off, we have the node (\designref{1}), this represents a worker machine in \gls{k8s}, which can run workloads. Next up, we have a \gls{pod} (\designref{1}), this is the smallest unit of deployment within \gls{k8s} and consists of one or more application containers. The last \gls{k8s} related concept is the Kubernetes Networking Model (\designref{3}) \cite{kubernetes-cluster-networking}, which can be implemented in various ways, but has as goal to connect the nodes and pods within the cluster. Within the \gls{pod}, lives the actual application (\designref{4}), which can for example be a web server. The \textit{Linux Network Stack} (\designref{5}) is abstracted in this design as one single component, but consists of routing and filtering tables used to manage packets within the system. An important note, is that both the node and the \gls{pod} implement a separate networking stack, which isolates the pod network from the host network through network namespaces \cite{man-network-namespaces}. The traffic flows to physical and virtual network interfaces (\designref{6}) connected through one another, to reach its destination. Note that in actual deployments, more complex models are often present, such as bridge networks to support multiple pod networks or tunnel interfaces to create overlay networks, for instance. However, for this example, the added complexity does not help to illustrate the architectural differences at hand.


% Explain datapath
With the basic components of the reference architecture explained, we can take a closer look at the data path of a service request. Whenever a request to a service has been made that lives within the application pod (\designref{4}) a packet first enters the node. The inbound packet enters through the network interface (\designref{6}), has to pass through the kernel's networking stack (\designref{5}) to reach the \gls{pod}'s network namespace. After this, the traffic will reach the service proxy (\designref{P}) because it is configured to intercept all network traffic inside the pod. From there onwards, it has to travel through a loopback interface to finally reach the application. Replies from the application will traverse a similar, but inverse, data path.

% Advantages/Disadvantages
One of the advantages of such an architecture is that you can have fine-grained control over the traffic, as it is intercepted the moment it enters the namespace of the \gls{pod}. This characteristic allows the security features discussed during this systems review (\ref{fr-2}), and can for example enable encrypted connections between all traffic leaving a \gls{pod}. Furthermore, the \textit{sidecar pattern} allows for a very generic implementation, and does not require much if any modifications from the user's perspective. This enables a user-friendly user experience, while also allowing support for general purpose proxies such as \textit{Envoy} or \textit{HAProxy}. These advantages, make it so that this is the most common architectural approach identified. A disadvantage, however, is that this design introduces additional overhead. First off, it introduces an overhead in system resources, as there now is a network proxy for every software service. Furthermore, it introduces additional latency for the traffic, as for every service request, the packets travel through the proxy twice (ingress/egress).


\subsubsection{Per-Node Proxy}
\label{sec:survey:analysis:architectures:per-node}
% - Less overhead and hops
% - No MTLS

Another identified \gls{sm} architecture makes use of the per-node proxy. This architectural style is used by a single identified \gls{sm} system, \textit{Traefik Mesh}. We present this form of \gls{sm} architecture in \cref{fig:sm-arch-per-node}. Although many of the components in this system are similar and previously explained in \cref{sec:survey:analysis:architectures:per-service}, some differences change several characteristics drastically.

\begin{figure}[!t]
    \centering
    
    \scalebox{.8}{
    \includegraphics[width=\linewidth]{3_systems_survey/figures/sm-arch-per-node.pdf}
    }
    
    \caption{A service mesh using a service proxy per-node.}
    \label{fig:sm-arch-per-node}
\end{figure}

The most significant difference compared to the per-service architectural style is that service proxy (\designref{P}) is now not embedded within the \gls{pod} but lives within the node, thus makes use of the host network namespace. The reason for this is that this architectural style tries to limit the additional overheads caused by the introduces proxies of a \gls{sm}. This style of service proxies results in a single proxy per node, and thus does not scale linearly with the number of services on a node. This leads to the advantage that it consumes less system resources in the form of CPU cycles and memory. An additional benefit of this approach is that the traffic in the data path now has to traverse the proxy once in a service-to-service request. This halves the amount of service proxy hops compared to the previous solution. 

However, this architectural pattern comes at a cost. Since the traffic is not passing through the service proxies from within the pod, traffic enters the host network in a potentially compromised state. To elaborate this, the per-service approach allowed the traffic between proxies to be automatically encrypted using a mutual TLS connection, a characteristic evaluated during the system's review (\ref{fr-2}). This is, however, not possible as the data path outside the \glspl{pod} is not fully controlled and therefore has to be implemented manually. A downside is that when this is ignored, service-to-service data flows unencrypted through the node and therefore breaches well established security best practices such as the notion of zero trust networks \cite{zero-trust-network}.




\subsubsection{eBPF Proxy}
\label{sec:survey:analysis:architectures:ebpf}
% - New area
% - Faster, less hops
% - Potential security risks?
% -- Programmable Sandbox in kernel space

The last of the identified \gls{sm} architectures uses a vastly different approach. The approach used by \textit{Cilium} uses \gls{ebpf} to establish a mesh network between the services. What started out as a \gls{k8s} networking solution and evolved to become a fully fledged \gls{sm} implementation \cite{cilium-mesh}. 

% Shortened data path
In \cref{fig:sm-arch-ebpf}, we present a simplified data path of a \gls{sm} implemented using \textit{Cilium}.  Many of the components a re shared with the other architectures and are explained in \cref{sec:survey:analysis:architectures:per-service}. However, compared to the other identified architectures, the data path as depicted here is much shorter. This is because the routing and proxying is done in the kernel through \gls{ebpf} (\designref{P}). 

% Explain EBPF
To explain the shortened data path, we first briefly have to introduce the technology powering it. \gls{ebpf} is a Linux Kernel technology that allows users to run sandboxed programs in the operating system kernel \cite{ebpf}. Furthermore, it is event-driven and allows these programs to execute on certain hooks. This is used throughout \textit{Cilium} to implement the observability, security and networking functionalities of the \gls{sm}. \gls{ebpf} allows for hooks throughout the entire lifecycle of a network packet. This combined with the programmable sandbox environments allows it to replace the role of service proxies in the other architectural styles. Since the \gls{ebpf} programs are aware of the \gls{k8s} endpoints and services, packets can be routed straight from the kernel, enabling a much more direct route.

% Advantages/Disadvantages
This architectural style has several advantages compared to the previously introduced per-service and per-node architectures. First, it has potential to decrease latency in its data path. It does this by not requiring additional hops in service-to-service communications. Additionally, it allows for greater programmability of the mesh network, as certain programs can be compiled and then run in a sandboxed environment. However, it also can have some disadvantages. First off, the technology and this implementation of it in particular is fairly new, even in a beta phase at the time of writing. This means that the \gls{sm} implementation could be unstable and not suitable for production usage. Furthermore, using user programs in kernel space could introduce potential security risks. 




\textit{Cilium} started out 
\begin{figure}[!t]
    \centering
    
    \scalebox{.8}{
    \includegraphics[width=\linewidth]{3_systems_survey/figures/sm-arch-ebpf.pdf}
    }
    
    \caption{A service mesh using a service proxy per-node.}
    \label{fig:sm-arch-ebpf}
\end{figure}


\subsection{Conclusion}
\label{sec:survey:analysis:conclusion}

To conclude the systems survey, we return to the initial research question (\ref{rq-1}), \textit{How do existing \gls{sm} implementations compare?}. The answer to this question is provided in various levels of detail throughout the systems survey. The most detailed and objective answer is provided  within the data synthesis and results section (\cref{sec:survey:results}) in which we identified domain-specific characteristics and features and compared the systems based on that. A summarized result is presented in the form of established functional and non-functional requirements (\cref{sec:survey:analysis}), in which we combine results from the data synthesis to provide a more generic, but more useful result. Finally, in \cref{sec:survey:analysis:architectures}, we take a deep-dive into the architectural differences of identified systems in which we answer the research question by exploring key characteristics in more detail.

Throughout this systems survey, we observed that \gls{sm} systems can have different approaches and goals and that systems within this the area are constantly changing. With many of the systems adopting different strategies and bleeding edge technologies, the area is exciting in many ways. 
\section{Implementation}
\label{sec:system:implementation}

