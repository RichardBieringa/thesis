\begin{table*}[t]
\centering
% \resizebox{\linewidth}{!}{ %< auto-adjusts font size to fill line

\begin{tabularx}{\textwidth}{ll|cc|cc|c}
\toprule

\multicolumn{2}{c|}{\textbf{Workload Generator}} &
\multicolumn{2}{c|}{\textbf{Workload Support}} &
\multicolumn{2}{c|}{\textbf{Configurability}} &
\multicolumn{1}{c}{\textbf{Other}}  \\

\textbf{ID} &
\textbf{Name} &
\textbf{HTTP/2} &
\textbf{GRPC} &
\textbf{Payload} & 
\textbf{Output} & 
\textbf{Server Process} \\
\midrule

\ref{wg-1} &
\textbf{wrk2} &
\circleE &
\circleE &
\circleH &
\circleH &
\circleE  \\

\ref{wg-2} &
\textbf{k6} &
\circleH &
\circleH &
\circleH &
\circleH &
\circleH \\

\ref{wg-3} &
\textbf{hey} &
\circleF &
\circleE &
\circleF &
\circleF &
\circleE \\

\ref{wg-4} &
\textbf{ghz} &
\circleE &
\circleF &
\circleF &
\circleF & 
\circleE \\

\ref{wg-5} &
\textbf{fortio} & 
\circleF &
\circleF &
\circleF & 
\circleF & 
\circleF \\


\bottomrule
\end{tabularx}

% } %< \resizebox

\caption[Comparing workload generator systems.]{Comparing workload generator systems on relevant properties and features. There are three different symbols in the table, each of them represents how well a requirement is satisfied. \circleF: Fully Satisfied, \circleE: Not Satisfied, \circleH: Partially Satisfied.}
\label{tab:system:implementation:workload-generators}
\end{table*}

