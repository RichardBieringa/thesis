\section{Cloud Native Computing Foundation}
\label{sec:background:cncf}

% Explain the CNCF
% - What are the goals of the CNCF
% - Well established  entity and governing body in the field

In this section, we briefly introduce the \glsfirst{cncf}, an entity that is often mentioned throughout this thesis as they are a governing body in the field. 

The \gls{cncf} is part of the \textit{Linux Foundation}\footnote{The Linux Foundation  is a non-profit technology consortium that aims to standardize Linux, support its growth, and promote its commercial adoption.} and was founded in 2015 at the same time the first version of \gls{k8s} was introduced. The goal of this foundation is to be an open-source, vendor-neutral hub of \textit{cloud native computing} \cite{cncf-who-are-we}. Where cloud native technologies are defined as technologies that \say{...empower organizations to build and run scalable applications in modern, dynamic environments such as public, private, and hybrid clouds. Containers, service meshes, microservices, immutable infrastructure, and declarative APIs exemplify this approach}.

The \gls{cncf} governs many open-source software projects, \gls{k8s} is one of such projects as it was donated to the foundation upon release. There are currently 1119 projects governed by the foundation at the time of writing, which form the \textit{CNCF landscape}. This landscape (\cref{appendix:cncf-landscape}) has an estimated market cap of \$20.3 trillion and a total funding of \$51.6 billion\footnote{\url{https://landscape.cncf.io/}}. Projects are categorized in three levels of maturity\footnote{\url{https://www.cncf.io/projects/}}, ranging from (in ascending order) \textit{Sandbox}, to \textit{Incubated} to \textit{Graduated}.

Throughout the rest of this thesis, we use several projects that are governed by the \gls{cncf}, as many of the \gls{sm} systems are part of the landscape. In addition to that, we use \gls{k8s} as the resource manager to construct our experimental environment (see \cref{sec:experiments:design:environment:cluster}) and use various 
 other projects in our design and implementation of \textit{Mesh Bench} (see \cref{sec:system:implementation}). In addition to the projects, we use a lot of the data the foundation gathers through their annual surveys, which provide key insights into the field of cloud native, and serverless technologies.
 