\section{Kubernetes}
\label{sec:background:kubernetes}

In the previous sections we discussed the changes in deployment models (\cref{sec:background:containers}) and how this caused a shift in design based on decoupling business logic (\cref{sec:background:soa}). In this section, we introduce \gls{k8s}, a container-centric resource manager, that is the de facto standard within the field. With over 96\% of  organizations using or evaluating \gls{k8s} according to a recent survey \cite{cncf-survey-2021} conducted by the the \gls{cncf}, the industry-leading foundation for container technologies, it is clear that the technology had a large impact on the industry and technological developments around it. Throughout the remainder of the work presented in this thesis we use \gls{k8s}, as most of the systems we examine and evaluate, rely or build on top of \gls{k8s}. We also briefly introduce some relevant \gls{k8s} related concepts, which are used throughout this thesis, such as the architectural diagrams discussed in \cref{sec:survey:analysis:architectures}.

The \gls{k8s} project was initially conceived by Google and started in June 2014  and launched in early 2015 at OSCON \cite{kubernetes-launch}. It was designed to be a container-centric resource manager, and its design was heavily influenced by the internal tooling used within Google \cite{burns2016borg}. With the container as unit of compute, \gls{k8s} provides a streamlined approach to declaratively manage workloads. From \textit{service discovery} and \textit{load balancing} to \textit{storage orchestration}, \gls{k8s} aims to solve the many challenges that appear when managing large amounts of containers in production environments. \todo{Wikipedia ref?} Based on concepts of \textit{Control Theory} \cite{control-theory}, the system uses a declarative approach which tries to keep the ecosystem in a desired state at any given time.

\Gls{k8s} consists of a set of components\footnote{\url{https://kubernetes.io/docs/concepts/overview/components/}} that have to be installed on several \textit{nodes}\footnote{\url{https://kubernetes.io/docs/concepts/architecture/nodes/}} (physical or \glspl{vm}) to form a compute \textit{cluster}. The core idea of \gls{k8s} is that is controlled through a centralized \textit{kube-apiserver}, which controls all components and acts as an interface for operators to control the cluster and manage workloads running on it.

\subsection{The Kubernetes Networking Model}
\label{sec:background:kubernetes:networking-model}

\Gls{k8s} relies on a specific networking model, which makes it easier for the \textit{cluster operator}\footnote{The term cluster operator is used throughout this thesis to indicate the person operating a Kubernetes cluster, i.e. deploying workloads, managing networking and security etc.} to manage and direct network traffic within the cluster. The networking model required essentially comes down to the following rules \cite{container-networking-from-scratch}.

\begin{enumerate}
  \item All containers can communicate with all other containers without \textit{NAT}
  \item All nodes can communicate with all containers (and vice-versa) without \textit{NAT}
  \item The IP that a container sees itself as is the same IP that others see it as
\end{enumerate}

\Gls{k8s} does not come with a solution that implements this networking model, however, it requires it for operation. There are many ways to implement the \gls{k8s} networking model, third-party solutions (often referred to as a \textit{Container Networking Interface}) such as \textit{Calico} or \textit{Flannel}. Another option commonly used to make use of \textit{Virtual Private Cloud} solutions offered by various cloud providers such as \gls{aws} or \gls{gcp}, which allow the user to emulate this flat networking model without having to install third-party software solutions.

\subsection{Pod}
\label{sec:background:kubernetes:pod}

Although \gls{k8s} is a container-centric resource manager, it does not use the container abstraction to manage the workloads within the ecosystem. Instead, it uses an abstraction on top of containers, namely the \textit{Pod}. Pods are the smallest deployable units of computing that you can create and manage in Kubernetes\footnote{\url{https://kubernetes.io/docs/concepts/workloads/pods/}}. 

It consists of one or more containers that emulate a logical \textit{host}, much like a \gls{vm} would. This means that each pod has its own isolated \textit{networking stack}\footnote{The networking stack, in the context of this thesis, refers to the Linux networking stack and consists of all the layers a packet has to traverse on a Linux-based system.} through \textit{namespaces} (\cref{sec:background:containers}), and that storage volumes are shared between containers inside a pod. Networking between containers within the same pod can be achieved through the \textit{loopback device} and reached through \textit{localhost} and there cannot be multiple processes that listen on the same port, much like a regular host. This is important to note, as this concept and network isolation paradigm is used by the most common \gls{sm} architecture discussed in this thesis (see \cref{sec:survey:analysis:architectures:per-service}).


\subsection{Service}
\label{sec:background:kubernetes:service}

The final \gls{k8s} concept we introduce is the notion of a \textit{service}\footnote{\url{https://kubernetes.io/docs/concepts/services-networking/service/}}. In \gls{k8s} the service abstraction provides a mechanism to expose an application as a network service. Briefly explained, it provides a stable IP address that is used to reach a set of pods allowing the cluster operator to expose those applications and provide a stable interface to reach them. 

Although we do use the \gls{k8s} service abstraction in this thesis during our experimental evaluation \cref{chap:experimental-evaluation}, it can cause some confusion since it conflicts with the definition of a logical service as explained in \cref{sec:background:soa}. Therefore, it is important to note that the term service refers to a \textit{logical service} in the context of this thesis and not to the \gls{k8s} concept with a similar name, unless specifically stated otherwise.