% Setup
\graphicspath{./figures}

\chapter{Background}
\label{chap:background}

% Introduction
% - Did not exist 10 years ago
% - Why now

A decade ago, the \gls{sm} architecture and systems implementing it did not exist, at least not in its current form. It is a by-product of the evolution in application architectures and the trends in cloud-native application design \cite{balalaie2016microservices}. The shift to service-based architectural styles became popular when the \gls{devops} movement gained traction and became standard in the industry \cite{microservices-trends} Although this shift improved the speed and agility of software development \cite{amaral2015performance} it came at the cost of additional operational complexity and overheads which \glspl{sm} try to solve. 


The remainder of this chapter introduces several concepts and related topics relevant to this thesis. The topics are introduced in a specific order, which paints a general picture of the evolution and progression of technologies in the landscape. First, we introduce the concept of container technologies and elaborate on the mass adoption caused by \gls{docker} (\cref{sec:background:containers}). Secondly, we introduce the concept of \glspl{micro-service} as a natural evolution of the \gls{soa} architecture paradigm, and how it is further enabled by the adoption of container technologies (\cref{sec:background:soa}). Furthermore, we introduce \gls{k8s}, the de facto standard in container orchestration and which problems it tries to solve (\cref{sec:background:kubernetes}). Afterwards, we go into the details of \gls{sm} technology where we introduce the core characteristics that define such a system and what problems this additional layer of networking abstraction tries to solve (\cref{sec:background:service-mesh}). Thereafter, we introduce the \glsfirst{cncf}, a governing body in the field which is frequently mentioned throughout this thesis. Finally, in \cref{sec:background:related-work}, we introduce the related work.

% Sections
\section{Containers}
\label{sec:background:containers}


% General introduction Container
A \gls{container} is a unit of executable software that is packaged with all of its dependencies\footnote{\url{https://www.docker.com/resources/what-container}}. By packaging everything in a single unit it makes it easy to run the application on different environments, from desktops and laptops to cloud environments. \Glspl{container} isolate software from their environment and ensure that it works uniformly on all devices regardless of operating systems used or the hardware powering it. 


% How it works
\Glspl{container} leverage a set of Linux technologies, such as \textit{cgroups}\footnote{\url{https://man7.org/linux/man-pages/man7/cgroups.7.html}} and \textit{namespaces}\footnote{\url{https://man7.org/linux/man-pages/man7/namespaces.7.htm}}. The former is a mechanism to organize processes in hierarchical groups whose usage of various system resources can then be limited and monitored. The latter is a technology to wrap a system resource in an abstraction to make it appear to a process as if they had their own isolated instance of that resource. These combined with the fact that a container has a layered file system that contains the application code and operating system dependencies makes it so that it is a fast and lightweight unit of compute. 

A comparison with \glspl{vm} is commonly made because of their isolating properties and abstractions. However, a \gls{vm} is a software-based virtualization of an entire computer system, this includes the hardware and the entire operating system that runs on top of it. A \gls{container} on the other hand is an abstraction at the application level, where it is just another process living in user space. A comparison of the different deployment models can be seen in \cref{fig:vm-vs-container}.


\begin{figure}[!t]
    \centering
    
    \includegraphics[width=.9\linewidth]{2_background/figures/vm-vs-container.pdf}

    \caption[Container and virtual machine-based deployment models]{Comparison of container and virtual machine-based deployment models.}
    \label{fig:vm-vs-container}
\end{figure}

% Docker 
\Gls{docker} was introduced in Santa Clara at PyCon in 2013 \cite{pycon2013}. It jump started the revolution of \glspl{container} and was for many the first introduction to \gls{container} technologies. It was loved by developers for its portability and speed compared to traditional \gls{vm} deployments that were commonly used. Due to its immense popularity and praise, which it has managed to uphold according to recent surveys as conducted by stack overflow \cite{stack-overflow-survey-2021}, \gls{docker} became the de facto standard in \gls{container} technologies. The similarly named company decided in 2015 to establish an open-source initiative that governs \gls{container} related standards\footnote{\url{https://opencontainers.org/}}. Along with this they donated their container image specification format\footnote{\url{https://github.com/opencontainers/image-spec}} to this initiative as an open-source contribution. With this image specification standard in place, others could develop compatible container runtimes which were able to run the container images produced by said specification.

\section{A Shift in System Design}
\label{sec:background:soa}

% Small introduction an how it relates to prev chapter.
% Docker ->  Small deployment cost -> next level soa
In the previous section, we introduced \glspl{container} technology and \Gls{docker} (\cref{sec:background:containers}). These technologies reduced the cost of deployment by introducing a small, standardized unit of software. The adoption of container technology led to a shift in architectural design and led to an increase in usage of \glsfirst{soa} design patterns. Due to the little overhead in terms of system resources caused by the \gls{container} technology, it was an ideal candidate for independent self-contained components. This section will introduce the concept of a logical \textit{service} in distributed systems and documents the general trends caused by the shift in architectural paradigms.

% Before SOA, monoliths
\subsection{The Software Monolith}
\label{sec:background:soa:monolith}

\begin{figure}[!t]
    \centering
    
    \includegraphics[width=0.3\linewidth]{2_background/figures/monolith-architecture.pdf}

    \caption[A monolithic software architecture]{A monolithic architecture. In this form of software design, all the software is contained into a single, self-contained software program depicted by the dashed line. }
    \label{fig:monolithic-architecture}
\end{figure}


Before the adoption of \gls{soa} principles and the notion of services in general, companies, and organizations alike used to create so-called software \glspl{monolith}. A monolithic approach in software design produces a self-contained software program where all of its dependencies, data access patterns and user interfacing components are combined (as depicted in \cref{fig:monolithic-architecture}). This makes monolithic architectures difficult to use in distributed systems without ad-hoc solutions or frameworks \cite{dragoni2017microservices}. A software \gls{monolith} can have several advantages, it produces a single binary, all code is colocated together,  and it is a battle-tested architecture. However, it can also have several disadvantages, it can be hard to maintain, codebases can become gigantic over time and lead to accumulated technical debt that makes the product unmaintainable with reasonable efforts \cite{fritzsch2018monolith}. This architectural pattern also forces the developers of the application to stick to their technical architecture, such as the choice of programming language and frameworks used. Furthermore, it can suffer from dependency hell, where updating or adding libraries can break existing systems  \cite{merkel2014docker}. Finally, the architecture does not scale very well, by scaling the application every single aspect or module has to be duplicated which is inefficient if only a subset of the application is put under load.



\subsection{A Service-Oriented Approach}
\label{sec:background:soa:service-oriented}


\begin{figure}[!t]
    \centering
    
    \includegraphics[width=.8\linewidth]{2_background/figures/service-oriented-architecture.pdf}

    \caption[A service-oriented software architecture]{A service-oriented approach. This architectural approach allows engineers to split applications into separate business components which enables them to scale and operate individually.}
    \label{fig:soa-architecture}
\end{figure}


% From monolith to SOA
% - Benefits
% - Microservices
As previously stated, the \gls{monolith}ic architecture was not well suited for use in distributed systems. This led to a newer style of software architectures that decomposed the business logic into  logical \textit{services}, where functionalities are encapsulated and abstracted from context \cite{perrey2003service}. This architectural paradigm meant that applications have to be decomposed into several self-contained units, which are then exposed via a \textit{service interface}. The service interface utilizes common communication standards so that it can be incorporated in new applications without much hassle \cite{ibm-soa}. Each service contains the code and data integrations required to execute a discrete business function. The core idea behind the \gls{soa} paradigm is that it promotes reusability and component sharing. This then translates into several benefits, such as an increase in scalability because it allows for scaling at a service level instead of having to scale an entire monolith. Furthermore, it allows software developers to use multiple technologies and frameworks, making it easier to pick the right tools for the job. This is enabled by the fact that services can exchange data over common protocols and agreed upon data representations, such as \gls{json} for example.


\subsection{Microservices}
\label{sec:background:soa:microservices}

% Intrdouce Microservices
% Small recap of why now

\begin{figure}[!t]
    \centering
    
    \includegraphics[width=.8\linewidth]{2_background/figures/microservices-vs-soa.pdf}

    \caption[The granularity of a microservices architecture]{Service-Oriented Architecture vs Microservices Architecture both  depicting an authentication service. Note: The granularity of the service definition depicts the most significant difference in the evolution of design.}
    \label{fig:soa-vs-microservices}
\end{figure}

The \textit{microservices} architecture (\cref{fig:soa-vs-microservices}) is an architectural style that emerged from the \gls{soa}. This evolvement was pushed by the  \gls{devops} movement and enabled by the decrease in deployment costs from the emerging \gls{container} technology \cite{amaral2015performance}.  The term \textit{microservice} dates in the context of distributed systems dates at least as far back as 2013 \cite{fowler-microservices}. It realizes its distinct itself from the \gls{soa}  paradigm by having a strong focus on its degree of independence regarding development and operation \cite{ibm-soa-vs-microservices}. The services that make up a \gls{soa}, can range from small, specialized services to enterprise-wide services, whereas a microservice consists of a highly specialized service, designed to do one thing well. This finer granularity allows for individual teams focussing on select subsets of an application and increases agility and development speed. However, this amplifies the problems that were mentioned above as it introduces more individual units in a complex system. Large companies such as Netflix that use such a microservices' architecture for example have reportedly managed to accumulate over 1000 microservices as of 2021 \cite{design-example-microservices, netflix-microservices-cost}.



\subsection{A New Set of Challenges}
\label{sec:background:soa:challenges}

% Down sides of a service oriented approach
% - Complexity
% - Network failures
% - Load balancing
% - Health cheecking
% - Serbice discovery
% Solutions to these complexitis
% - Fat clients
% - Enterprise Service Bus


\begin{figure}[!t]
    \centering
    
    \includegraphics[width=.7\linewidth]{2_background/figures/software-lib-approach.pdf}

    \caption[Software library approach for solving networking challenges.]{Software library approach for solving the challenges introduces by a service-oriented design. In this design, the software library implements a uniform client and server API, designed to  handle fault tolerance, load balancing and latency optimizations to construct high-concurrency services.}
    \label{fig:software-lib-approach}
\end{figure}

However, by splitting the application up into several components, we introduce additional complexities. First off, the coarse granularity of the service-oriented design means that testing and validating every combination and condition may be very complex or even impossible \cite{mahmood2007service}. Furthermore, the loosely coupled services might be an architect's dream; however, it introduces additional complexities for a software developer \cite{fowler2012patterns}. Additionally, the service-to-service communications can now introduce failures, especially if they are carried out over unreliable networks such as the internet. Finally, the interoperability of services introduces additional challenges. How and where can we reach the services? Which version of the service is running and is it still compatible with everything? If there are multiple instances of this service, which one should be targetted to prevent overloading and ensure equal load? 



One approach used to solve these problems was through software libraries to handle all service-to-service communications (\cref{fig:software-lib-approach}) in a uniform way \cite{service-mesh-history}. Companies like Google, Netflix, and Twitter developed custom software libraries for this, such as \textit{Stubby} \cite{stubby}, \textit{Hystrix} \cite{hystrix} and \textit{Finagle} \cite{finagle} respectively. These libraries would perform load balancing, implement retry mechanisms, routing, and telemetry. A downside of this approach was that this meant that the libraries were usually written in a single programming language, locking the developers in, or resulting in having to support multiple libraries. Furthermore, it resulted in a scenario where updating the library meant that every service implementing it also required an update. 

% \todo{double check ESB bit} Another approach used was to utilize another design pattern that implements an \gls{esb}, an open standards, message-based, distributed integration infrastructure that provides routing, invocation and mediation services to facilitate the interactions of disparate distributed applications and services in a secure and reliable manner \cite{menge2007enterprise}. This dedicated piece of infrastructure combines  Message-Oriented Middleware (MOM), web services, transformation and routing intelligence as a backbone for Service-Oriented Architecture.



\section{Kubernetes}
\label{sec:background:kubernetes}

In the previous sections we discussed the changes in deployment models (\cref{sec:background:containers}) and how this caused a shift in design based on decoupling business logic (\cref{sec:background:soa}). In this section, we introduce \gls{k8s}, a container-centric resource manager, that is the de facto standard within the field. With over 96\% of  organizations using or evaluating \gls{k8s} according to a recent survey \cite{cncf-survey-2021} conducted by the the \gls{cncf}, the industry-leading foundation for container technologies, it is clear that the technology had a large impact on the industry and technological developments around it. Throughout the remainder of the work presented in this thesis we use \gls{k8s}, as most of the systems we examine and evaluate, rely or build on top of \gls{k8s}. We also briefly introduce some relevant \gls{k8s} related concepts, which are used throughout this thesis, such as the architectural diagrams discussed in \cref{sec:survey:analysis:architectures}.

The \gls{k8s} project was initially conceived by Google and started in June 2014  and launched in early 2015 at OSCON \cite{kubernetes-launch}. It was designed to be a container-centric resource manager, and its design was heavily influenced by the internal tooling used within Google \cite{burns2016borg}. With the container as unit of compute, \gls{k8s} provides a streamlined approach to declaratively manage workloads. From \textit{service discovery} and \textit{load balancing} to \textit{storage orchestration}, \gls{k8s} aims to solve the many challenges that appear when managing large amounts of containers in production environments. \todo{Wikipedia ref?} Based on concepts of \textit{Control Theory} \cite{control-theory}, the system uses a declarative approach which tries to keep the ecosystem in a desired state at any given time.

\Gls{k8s} consists of a set of components\footnote{\url{https://kubernetes.io/docs/concepts/overview/components/}} that have to be installed on several \textit{nodes}\footnote{\url{https://kubernetes.io/docs/concepts/architecture/nodes/}} (physical or \glspl{vm}) to form a compute \textit{cluster}. The core idea of \gls{k8s} is that is controlled through a centralized \textit{kube-apiserver}, which controls all components and acts as an interface for operators to control the cluster and manage workloads running on it.

\subsection{The Kubernetes Networking Model}
\label{sec:background:kubernetes:networking-model}

\Gls{k8s} relies on a specific networking model, which makes it easier for the \textit{cluster operator}\footnote{The term cluster operator is used throughout this thesis to indicate the person operating a Kubernetes cluster, i.e. deploying workloads, managing networking and security etc.} to manage and direct network traffic within the cluster. The networking model required essentially comes down to the following rules \cite{container-networking-from-scratch}.

\begin{enumerate}
  \item All containers can communicate with all other containers without \textit{NAT}
  \item All nodes can communicate with all containers (and vice-versa) without \textit{NAT}
  \item The IP that a container sees itself as is the same IP that others see it as
\end{enumerate}

\Gls{k8s} does not come with a solution that implements this networking model, however, it requires it for operation. There are many ways to implement the \gls{k8s} networking model, third-party solutions (often referred to as a \textit{Container Networking Interface}) such as \textit{Calico} or \textit{Flannel}. Another option commonly used to make use of \textit{Virtual Private Cloud} solutions offered by various cloud providers such as \gls{aws} or \gls{gcp}, which allow the user to emulate this flat networking model without having to install third-party software solutions.

\subsection{Pod}
\label{sec:background:kubernetes:pod}

Although \gls{k8s} is a container-centric resource manager, it does not use the container abstraction to manage the workloads within the ecosystem. Instead, it uses an abstraction on top of containers, namely the \textit{Pod}. Pods are the smallest deployable units of computing that you can create and manage in Kubernetes\footnote{\url{https://kubernetes.io/docs/concepts/workloads/pods/}}. 

It consists of one or more containers that emulate a logical \textit{host}, much like a \gls{vm} would. This means that each pod has its own isolated \textit{networking stack}\footnote{The networking stack, in the context of this thesis, refers to the Linux networking stack and consists of all the layers a packet has to traverse on a Linux-based system.} through \textit{namespaces} (\cref{sec:background:containers}), and that storage volumes are shared between containers inside a pod. Networking between containers within the same pod can be achieved through the \textit{loopback device} and reached through \textit{localhost} and there cannot be multiple processes that listen on the same port, much like a regular host. This is important to note, as this concept and network isolation paradigm is used by the most common \gls{sm} architecture discussed in this thesis (see \cref{sec:survey:analysis:architectures:per-service}).


\subsection{Service}
\label{sec:background:kubernetes:service}

The final \gls{k8s} concept we introduce is the notion of a \textit{service}\footnote{\url{https://kubernetes.io/docs/concepts/services-networking/service/}}. In \gls{k8s} the service abstraction provides a mechanism to expose an application as a network service. Briefly explained, it provides a stable IP address that is used to reach a set of pods allowing the cluster operator to expose those applications and provide a stable interface to reach them. 

Although we do use the \gls{k8s} service abstraction in this thesis during our experimental evaluation \cref{chap:experimental-evaluation}, it can cause some confusion since it conflicts with the definition of a logical service as explained in \cref{sec:background:soa}. Therefore, it is important to note that the term service refers to a \textit{logical service} in the context of this thesis and not to the \gls{k8s} concept with a similar name, unless specifically stated otherwise.
\section{Service Mesh}
\label{sec:background:service-mesh}

In the previous sections we discussed the evolution in deployment models (\cref{sec:background:containers}) and how this caused a shift in system design, based on decoupling business logic into logical services (\cref{sec:background:soa}). We then went on to briefly introduce \gls{k8s} and several related concepts (\cref{sec:background:soa}). In this section, we build on top of the topics discussed in the background chapter and introduce the \gls{sm} architecture which is at the core of the work presented in this thesis. A style of architecture that builds on top of the \gls{k8s} ecosystem which aims to solve the communicative challenges caused by service-oriented design.

% This section introduces a service mesh to the user
% What is a service mesh
% What does a service mesh try to solve
% Why is it becoming a thing now
% Is it a overhyped technology




% Context
% Short intro service mesh, how it can add to latency/overhead
A \gls{sm} is a dedicated layer of networking infrastructure that sits between logical software services and aims to solve some communicative challenges introduces by a \gls{soa}. The terminology was introduced in the last decade \cite{service-mesh-hype}, and several systems adhering to this architecture have seen the light of day since. The premise of introducing additional observability in the system, increasing the reliability of it while also adding security best practices to little or no additional lines of application code made this technology a hot topic in the field. However, this technology does not come in the form of a zero cost offering. The \gls{sm} architecture introduces additional machinery to achieve its goals. The components that make up a \gls{sm} can be divided into two  categories. A set of management components that constitute the \textit{Control Plane} and a set of components that handle the network traffic referred to as the \textit{Data Plane}.


Most notably, it adds a network proxy in front of logical services which intercepts the network traffic of a service. By introducing these components into the system, an increase in system resources is to be expected. More importantly, the introduction of a network proxy leads to an additional hop in the data path for each packet. This leads to an increase in request latency for each request destined to the software service. 

% Effects of introduces latency / system  resources
The introduction of additional request latencies or the increase in system resources can be a make or break scenario in many aspects. Companies often set certain \glspl{kpi} based on the end-to-end response times of services they provide. This is because of the effects that request latencies can have on the performance of business offerings. Amazon, for example, found that for every 100ms of additional end-to-end latency, it would lose out on roughly 1\% of total sales revenue \cite{amazon-latency-conversion-drop}. Google, indicated back in 2009 \cite{google-latency-bounce-rate} that latency has a direct relation with service usage and has been actively prioritizing performance of services since then. However, Google goes a step further, as they now actively measure the performance of websites and use these results to implement their search ranking \cite{google-search-index-perf}. 

% Cost/Benefit discussion
% Media-Attention
This naturally brings us to the following questions; Do the benefits of a \gls{sm} system outweigh the performance implications it introduces? Is this technology just another fad originating from the \gls{k8s} community? Or is the hype  \cite{hashicorp-sm-hype} and attention it has been receiving within countless blog posts and talks justified? Can it even be considered a "necessity" \cite{sm-kubernetes-necessity} in modern-day infrastructure or is it limited to certain use cases?

% Lack of research into the performance overeads
% Related work (kinvolk)
The technology has been in the spotlight, and many implementations of the \gls{sm} architecture have been released in the last couple of years, some of them even receiving significant support and backing from enterprise players in the field such as Google and Microsoft \cite{cncf-sm-landscape}. Additionally, the technology has seen a steady increase in its  its adoption rate \cite{cncf-survey-2020, rh-survey} and implementations have matured. However, there appears to be a sharp contrast in amount of attention the technology has received from both the industry and academia. In particular, there have been very few studies that evaluate the performance characteristics of these systems. Previous attempts originating from the industry have established the groundwork by introducing simple benchmarking instruments \cite{kinvolk-bench} and by presenting their preliminary results \cite{bench-istio-linkerd-2019, bench-istio-linkerd-2021}. However, the results originating from these benchmarks are primitive and compare a subset of \gls{sm} implementations in the field.

\section{Cloud Native Computing Foundation}
\label{sec:background:cncf}

% Explain the CNCF
% - What are the goals of the CNCF
% - Well established  entity and governing body in the field

In this section, we briefly introduce the \glsfirst{cncf}, an entity that is often mentioned throughout this thesis as they are a governing body in the field. 

The \gls{cncf} is part of the \textit{Linux Foundation}\footnote{The Linux Foundation  is a non-profit technology consortium that aims to standardize Linux, support its growth, and promote its commercial adoption.} and was founded in 2015 at the same time the first version of \gls{k8s} was introduced. The goal of this foundation is to be an open-source, vendor-neutral hub of \textit{cloud native computing} \cite{cncf-charter}. Where cloud native technologies are defined as technologies that \say{...empower organizations to build and run scalable applications in modern, dynamic environments such as public, private, and hybrid clouds. Containers, service meshes, microservices, immutable infrastructure, and declarative APIs exemplify this approach}.

The \gls{cncf} governs many open-source software projects, \gls{k8s} is one of such projects as it was donated to the foundation upon release. There are currently 1119 projects governed by the foundation at the time of writing, which together, form the \textit{CNCF landscape}. This landscape has an estimated market cap of \$20.3 trillion and a total funding of \$51.6 billion\footnote{\url{https://landscape.cncf.io/}}. Projects are categorized in three levels of maturity\footnote{\url{https://www.cncf.io/projects/}}, ranging from (in ascending order) \textit{Sandbox}, to \textit{Incubated} to \textit{Graduated}.

Throughout the rest of this thesis, we use several projects that are governed by the \gls{cncf}, as many of the \gls{sm} systems are part of the landscape. In addition to that, we use \gls{k8s} as the resource manager to construct our experimental environment (see \cref{sec:experiments:design:environment:cluster}) and use various other projects in our implementation of \textit{Mesh Bench} (see \cref{sec:system:implementation}). In addition to the projects, we use a lot of the data the foundation gathers through their annual surveys, which provide key insights into the field of cloud native, and serverless technologies.
 
\section{Related Work}
\label{sec:background:related-work}

In this section, we discuss the identified related work in the field. The core work in this thesis is divided into three primary chapters, the systems survey (\cref{chap:survey}), the benchmark design (\cref{chap:system-design}), and the experimental evaluation (\cref{chap:experimental-evaluation}). For each of these chapters we introduce a section with related work from both academia and the industry.

\subsection{System Survey}
\label{sec:background:related-work:survey}


\subsection{Benchmark}
\label{sec:background:related-work:survey}


\subsection{Experimental Evaluation}
\label{sec:background:related-work:survey}


