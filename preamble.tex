% VU Page layout settings
\textwidth 15cm
\textheight 22cm
\parindent 10pt
\oddsidemargin 0.85cm
\evensidemargin 0.37cm

% VU overflow settings
% https://www.overleaf.com/learn/latex/%5Chbadness?
% https://www.overleaf.com/learn/latex/%5Chfuzz
\hbadness=10000
\hfuzz=50pt

% VU line spacing settings
% https://www.overleaf.com/learn/latex/Paragraph_formatting#Line_spacing
\renewcommand\baselinestretch{1.2}
\baselineskip=18pt plus1pt


% VU font encoding default
% https://www.overleaf.com/learn/latex/Font_typefaces
\usepackage[T1]{fontenc}

% UTF8 File Encoding
% https://www.overleaf.com/learn/how-to/What_file_encodings_and_line_endings_should_I_use%3F
\usepackage[utf8]{inputenc}

% Makes text pretty
\usepackage[
    activate={true,nocompatibility},
    final,
    tracking=false,
    kerning=true,
    spacing=false,
    factor=1100,
    stretch=10,
    shrink=10,
]{microtype}

% Defines acronyms used throughout the thesis
% See: glossaries.tex
\usepackage[abbreviations]{glossaries-extra}
\setabbreviationstyle[acronym]{long-short}

% GLOSSARY ENTRIES
% Defines acronyms and descriptions for frequently used
% terminology present in this thesis.
% http://tug.ctan.org/macros/latex/contrib/glossaries/glossariesbegin.pdf

% Define a term like this:
% \newglossaryentry{⟨label⟩}
% {
%     name={⟨name⟩},
%     description={⟨description⟩},
%     ⟨other options⟩
% }

% Acronym Example
% \newabbreviation{svm}{svm}{support vector machine}


% --------------------------------------------------------------------------

\makeglossaries


% --------------------------- ABBREVIATIONS --------------------------------

\newabbreviation{soa}{SOA}{Service Oriented Architecture}
\newabbreviation{iot}{IoT}{Internet of Things}
\newabbreviation{qos}{QoS}{Quality of Service}
\newabbreviation{kpi}{KPI}{Key Performance Indicator}
\newabbreviation{json}{JSON}{JavaScript Object Notation}
\newabbreviation{aip}{AIP}{Article Information Parser}
\newabbreviation{mcs}{MCS}{Massivizing Computer Systems}
\newabbreviation{cncf}{CNCF}{Cloud Native Computing Foundation}




%-------------------------- END ABBREVIATIONS ------------------------------




% ----------------------------- GLOSSARIES ---------------------------------

\newglossaryentry{sm}
{
    name={service mesh},
    plural={service meshes},
    description={A service mesh is a dedicated infrastructure layer that that abstracts the network layer between (micro)-services, to introduce distributed systems' best practices such as observability, traffic management, and security in a uniform way.}
}

\newglossaryentry{edge-computing}
{
    name={edge eomputing},
    description={A distributed computing paradigm that brings computation and data storage closer to the sources of data.}
}

\newglossaryentry{smi}
{
    name={Service Mesh Interface},
    description={A standard interface for service meshes on Kubernetes, which covers the most common service mesh capabilities such as traffic policies, traffic telemetry and traffic management.}
}

\newglossaryentry{smp}
{
    name={Service Mesh Performance},
    description={A common format to capture and describe configurations and results of service mesh performance experiments.}
}

\newglossaryentry{fr}
{
    name={Functional Requirement},
    description={A a requirement that specifies criteria that can be used to judge the operation of a system, rather than specific behaviors.}
}


\newglossaryentry{nfr}
{
    name={Non-Functional Requirement},
    description={A a requirement that specifies criteria that can be used to judge the operation of a system, rather than specific behaviors.}
}

% \newglossaryentry{cncf}
% {
%     name={Cloud Native Computing Foundation},
%     description={A Linux Foundation project that was founded in 2015 to help advance container technology and align the tech industry around its evolution in a vendor neutral fashion.}
% }

\newglossaryentry{pl}
{
    name={Primary Literature},
    description={The immediate results of research activities by scientists.}
}

\newglossaryentry{gl}
{
    name={Grey Literature},
    description={The materials and research produced by organizations outside of the traditional commercial or academic publishing.}
}

\newglossaryentry{slr}
{
    name={Systematic Literature Review},
    description={A type of literature review that uses repeatable analytical methods to collect and analyze data.}
}

\newglossaryentry{mlr}
{
    name={Multivocal Literature Review},
    description={A form of systematic literature reviews that includes \gls{gl} in addition to the formal \gls{pl}.}
}

\newglossaryentry{backward-snowballing}
{
    name={Backward Snowballing},
    description={A form of systematic literature reviews that includes \gls{gl} in addition to the formal \gls{pl}.}
}

\newglossaryentry{devops}
{
    name={DevOps},
    description={A set of software development (dev) and it operations (ops) practices which aim to shorten the systems development life cycle and provide continuous delivery with high software quality.}
}

\newglossaryentry{container}
{
    name={container},
    description={A unit of software packaging that combines the software and all of its dependencies in a standard format.}
}

\newglossaryentry{docker}
{
    name={Docker},
    description={A software platform used to create, run and manage software applications packaged in \glspl{container}.}
}

\newglossaryentry{vm}
{
    name={virtual machine},
    description={A software based, virtualization of a computer system.}
}

\newglossaryentry{micro-service}
{
    name={micro-service},
    description={A variant of the \gls{soa} architectural style with fine-grained services and ligthweight protocols.}
}

\newglossaryentry{k8s}
{
    name={Kubernetes},
    description={An open-source \gls{container} orchestration system for automating software deployment, scaling, and management.}
}

\newglossaryentry{monolith}
{
    name={monolith},
    description={A software architecture that is self-contained and where the user interface and data access paterns are combined into a single program.}
}

\newglossaryentry{esb}
{
    name={Enterprise Service Bus},
    description={An Enterprise Service Bus is an open standards, message-based, distributed integration infrastructure that provides routing, invocation and mediation services to facilitate the interactions of disparate distributed applications and services in a secure and reliable manner \cite{menge2007enterprise}.}
}

\newglossaryentry{ebpf}
{
    name={eBPF},
    description={eBPF (extended Berkeley Packet Filter), is a technology which allows to run sandboxed programs in the Linux Kernel. It can do so without changing the kernel source code or loading kernel modules.}
}

\newglossaryentry{linkerd}
{
    name={Linkerd},
    description={A service mesh implementation by Buoyant.}
}

\newglossaryentry{grpc}
{
    name={gRPC},
    description={A a modern open source high performance Remote Procedure Call (RPC) framework.}
}

\newglossaryentry{pod}
{
    name={Pod},
    description={The smallest deployable units of computing that you can create and manage in Kubernetes, consists of one or more application containers.}
}

\newglossaryentry{sut}
{
    name={System Under Test},
    description={A system that is being tested for correct operation}
}

\newglossaryentry{gcp}
{
    name={Google Cloud Platform},
    description={Google's offering for various cloud compute services}
}

\newglossaryentry{gke}
{
    name={Google Kubernetes Engine},
    description={A managed kubernetes service offered by Google on their Google Cloud Platform service.}
}

\newglossaryentry{aws}
{
    name={Amazon Web Services},
    description={Amazon's offering for various cloud compute services}
}




% \newglossaryentry{soa}
% {
%     name={Service Oriented Architecture},
%     description={A software development approach that takes advantages of decoupling software components into independent and reusable services.}
% }

\newglossaryentry{sla}
{
    name={Service-Level Agreement},
    plural={Service-Level Agreements},
    description={A commitment between a service provider and a client.}
}

\newglossaryentry{turnip}
{
    name={turnip},
    description={round pale root vegetable}
}

\newglossaryentry{carrot}
{
    name={carrot},
    description={orange root}
}

% --------------------------- END GLOSSARIES -------------------------------


\usepackage{amsmath, amsthm, amssymb} % American Math Society packages
\usepackage{pifont} % Symbols etc.
\usepackage{pdfpages} % Insert PDF pages into latex documents
\usepackage{tikz} % Package for creating graphics
\usepackage{booktabs} % Nicer tables
\usepackage{tabularx} % Extends tabular* through the addition of an X-column
\usepackage{multirow} % Complicated Tables
\usepackage{xcolor} % Enables coloured text
\usepackage{minted} % Code listings
\usepackage[noabbrev, capitalise]{cleveref} % Uniform (smart) references
\usepackage{todonotes} % Enables visual todonotes
\usepackage{enumitem} % Spacing in enumerations
\usepackage{url} % Clickable and formatted urls
\usepackage{listings} % Source code printer
\usepackage{svg} % Allows svg images with includesvg
\usepackage{graphicx} % Allows pdf image
\usepackage{pdfpages} % Allows pdf pages
\usepackage{fancyvrb } % Fancy inline verbatim (code snippets etc.)
\usepackage{dirtytalk} % Quotations
\usepackage{makecell} % Newlines in table cells etc.


% Listings styling 
% https://stackoverflow.com/questions/741985/latex-source-code-listing-like-in-professional-books
\lstset{
    basicstyle=\footnotesize\ttfamily, % Default font
    % numbers=left,              % Location of line numbers
    numberstyle=\tiny,          % Style of line numbers
    % stepnumber=2,              % Margin between line numbers
    numbersep=5pt,              % Margin between line numbers and text
    tabsize=2,                  % Size of tabs
    extendedchars=true,
    breaklines=true,            % Lines will be wrapped
    keywordstyle=\color{red},
    frame=b,
    % keywordstyle=[1]\textbf,
    % keywordstyle=[2]\textbf,
    % keywordstyle=[3]\textbf,
    % keywordstyle=[4]\textbf,   \sqrt{\sqrt{}}
    stringstyle=\color{white}\ttfamily, % Color of strings
    showspaces=false,
    showtabs=false,
    xleftmargin=17pt,
    framexleftmargin=17pt,
    framexrightmargin=5pt,
    framexbottommargin=4pt,
    % backgroundcolor=\color{lightgray},
    showstringspaces=false
}

% Circles for the qualitative comparison

% Empty
\newcommand*\circleE[1][1ex]{\tikz\draw (0,0) circle (#1);}

% Half filled
\newcommand*\circleH[1][1ex]{%
  \begin{tikzpicture}
  \draw[fill] (0,0)-- (90:#1) arc (90:270:#1) -- cycle ;
  \draw (0,0) circle (#1);
  \end{tikzpicture}}
  
% Fully filled
\newcommand*\circleF[1][1ex]{\tikz\fill (0,0) circle (#1);} 


% Checkmark and X
\newcommand{\cmark}{\ding{51}}%
\newcommand{\xmark}{\ding{55}}%

% designref icon

\newcommand{\designref}[1]{
 \begin{tikzpicture}[baseline=(char.base)]
   \node[draw, circle, inner sep=0.5pt, fill=black, text=white] (char){\small #1};
 \end{tikzpicture}
}


\setuptodonotes{
    inline,
    color=red!40,
}





