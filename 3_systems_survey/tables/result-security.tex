\begin{table*}[t]
\centering
% \resizebox{\linewidth}{!}{ %< auto-adjusts font size to fill line

\begin{tabular}{ll|cc}


\toprule

% HEADER 1
\multicolumn{2}{c|} {\textbf{Service Mesh}} &
\multicolumn{2}{c} {\textbf{Security Capbilities}}  \\

% HEADER 2
ID &
Name &
Mutual TLS &
S2S Auth. Policies \\


\midrule


\textbf{SM1} &
AWS App Mesh  & 
\cmark  &
\cmark  \\

\textbf{SM2} &
Cilium  & 
\cmark  & 
\cmark \\


\textbf{SM3} &
Consul  & 
\cmark  & 
\cmark   \\


\textbf{SM4}&
Ease Mesh & 
\cmark  & 
\cmark \\


\textbf{SM5} &
Istio & 
\cmark  & 
\cmark   \\


\textbf{SM6} &
Kuma & 
\cmark  & 
\cmark   \\


\textbf{SM7} &
Linkerd2 & 
\cmark  & 
\cmark \\


\textbf{SM8} &
Nginx Service Mesh & 
\cmark & 
\cmark \\


\textbf{SM9} &
Open Service Mesh  & 
\cmark  & 
\cmark \\


\textbf{SM10} &
Traefik Mesh &
\xmark  & 
\xmark \\


\bottomrule
\end{tabular}
% } %< \resizebox

\caption[Comparing service mesh security capabilities]
{Comparing the security capabilities of identified service mesh systems.

\textit{Mutual TLS} indicates if the \gls{sm} system supports TLS authentications, which enables encrypted communications between service proxies. \\
\textit{S2S Auth. Policies} (Service-to-service authorization policies) indicate whether the \gls{sm} system supports advanced  authorization policies which enables the operator to grant or deny permissions to services to communicate with given entities (such as other services). \\

\cmark: Indicates that the feature is supported. \\
\xmark:  Indicates that the feature is no supported. }
\label{tab:result-security}
\end{table*}

