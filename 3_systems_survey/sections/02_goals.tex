\section{Goals}
\label{sec:survey:goals}
% What are we trying to achieve in this survey
% - Bridge the gap between academia and industry
% - Gain insights into the current state-of-the-art-systems
% - Obtain knowledge about relative metrics and workloads 
% - ^> Is a stepping stone to the next chapter

% - Tie it all to the research question defined in the introduction

Before conducting a systematic systems survey, we first establish a set of goals which we will try to achieve in this work. This allows us to structure the research and adjust its methodologies based on the goals. The following list represents the formulated goals.

\begin{enumerate}[label=\textbf{G\arabic*}, leftmargin=3\parindent]
    \item \textbf{Bridge the gap between academia and industry.}
    \label{g-1}
    
    In the previous section (\cref{sec:survey:related-work}), we identified that there is a large gap between interest from academia and the industry. We identified that there is a lack of formal publications in the field, whereas preliminary exploration showed that this is in sharp contrast to the interest as expressed from the industry. This leads to this first goal, which is to close this gap and provide formal research in this emerging field.
    
    \item \textbf{Gain insight into the current state-of-the-art \gls{sm} systems.}
    \label{g-2}
    
    The second goal of this survey is to gain a clear understanding of the field and the current state-of-the-art \gls{sm} systems. We want to identify the systems that currently exist and identify their defining properties and characteristics.

    \item \textbf{Obtain knowledge about relevant performance metrics.}
    \label{g-3}
    
    Finally, we want to identify metrics that properly capture the performance implications of these systems. By examining these systems in detail, we want to obtain the metrics which are relevant in the application domain. 

\end{enumerate}

% - Tie it all to the research question defined in the introduction
% - Tie it to the general flow of the thesis as this provides a stepping stone to further chapters
The goals and their respective order in which they are  presented are closely tied to the general flow of research conducted within this thesis. We first established the gap between industry and academia in our previous work and formulated this thesis around the idea to close this \ref{g-1}. Furthermore, we apply best practices by conducting a survey of the field. This helps us to understand the field and allows us to examine the current state-of-the-art systems in detail \ref{g-2}. The data synthesis of this survey allows us to establish a framework to compare the existing and future \gls{sm} systems which might emerge. The results of this will be used to provide the answer to the research question \ref{rq-1} during our analysis. Additionally, the survey and resulting framework allows us to determine relevant metrics \ref{g-3}. This provides a stepping stone towards the following chapter, in which we design an instrument to compare the performance characteristics of these systems (\cref{chap:system-design}).