\section{Motivation}
\label{sec:survey:introduction}
\label{sec:survey:motivation}


% % Discuss identified related work
% % - Service Mesh survey
% % - Our previously conducted survey on the k8s ecosystem
% Before conducting a survey on the field, we examined relevant and related work on the topic. We have identified secondary literature in which authors have conducted a literature review on the state-of-the-art on \gls{sm} technology \cite{service-mesh-survey}. This review that was conducted in 2019 noted the fact that relevant, and related formal literature on the topic was lacking. They further stated that their work was the first work to formally synthesize the data in the field. Previous efforts by us to systematically synthesize the data concerning the formal literature on the \gls{k8s} ecosystem found similar results. In our previous work, we presented a topical framework on research conducted in the Kubernetes ecosystem and found that most of the formal literature was related to the resource management aspects of the ecosystem. With most of the research dedicated to scheduling and scaling algorithms, several of the layers as presented in the topical framework had relatively little attention. One of the identified technologies that was related to the Kubernetes ecosystem, in the form of \glspl{sm}, had very little formal work conducted on it, which ultimately led to the motivation behind the research conducted in this thesis.

% % Reason the need for such a survey
% % - Previous attempt gave general introduction
% % - Not a sytems survey
% % - Comparison of only a few (Istio, Linkerd, App Mesh, Synapse) sm implementations
% % - Landscape changed, new technolgies, definition might change?
% Although there has been a previous effort to survey the field \cite{service-mesh-survey}, we identified several shortcomings for our goals. First, the survey conducted by Li et al. provide a generic introduction to the concept of a service mesh, its features, and the challenges and opportunities in the field. Our goal is to provide a comparison of \gls{sm} implementations through their defining characteristics and performance related properties. The work conducted by Li et al. do provide some form of comparison, but do so with a focus on generic and subjective characteristics such as the maturity of the product, whether it is open-source and actively worked on and what the major advantage and disadvantage is of each platform. This provides a sharp contrast in the type of comparison that we try to conduct, which heavily emphasizes on characteristics typically associated with  \gls{sm} technologies, those that define the control and data plane components of such a system. Furthermore, the survey was conducted in 2019, however the landscape of \gls{sm} technologies has changed drastically since then. New \gls{sm} systems have emerged, while existing systems have evolved. Where Li et al. \cite{service-mesh-survey} have identified four different systems back in 2009, the \gls{cncf} landscape \cite{cncf-landscape}\footnote{\url{https://landscape.cncf.io/card-mode?category=service-mesh&grouping=category}} contains many more systems as is depicted in \cref{fig:cncf-landscape-sm}. This landscape only presents formally accepted projects and might not cover all the systems which are out there. Finally, recent industry-based efforts altered some ideas and principles of \gls{sm} systems, changing the traditional proxy-based approaches to include  \gls{ebpf}\footnote{\url{https://ebpf.io/}} based solutions, drastically changing the data plane by including revolutionary Linux kernel technologies \cite{istio-merbridge, cilium-mesh}.


% Conclude that there is a need for the review
% - Lack of formal (reiterate)
% - Industry interest
% - New approaches (proxyless, ebpf)



\begin{figure*}[t]
    \centering
    \includegraphics[width=0.9\linewidth]{3_systems_survey/figures/cncf-landscape-service-mesh}
    \caption[CNCF landscape of \gls{sm} technologies]{\gls{cncf} landscape of \gls{sm} technologies as of March 2022.}
    \label{fig:cncf-landscape-sm}
\end{figure*}


To justify a formal systematic survey we first evaluate the existing related work in the field. In \cref{sec:background:related-work:survey}, we discuss the singular secondary study that we identified before executing this system survey. We concluded, similarly to the authors of the paper, that there is a lack of academic literature in this field. Even though the work  from Li et al. \cite{service-mesh-survey} was concluded in 2019, the situation still prevails. This provides a sharp contrast with the amount of attention the field has received from the industry as it contains an abundance of blog posts and talks. The author of one of the most popular \gls{sm} systems even went as far to call it the \say{the world’s most over-hyped technology} \cite{service-mesh-manifesto}. Aside from the attention it has been getting, it has also been shown that \gls{sm} systems are becoming increasingly popular as indicated in surveys conducted by Red Hat \cite{rh-survey-2022} and the \gls{cncf} \cite{cncf-survey-2020}.

In addition to the attention of the industry, the field has seen many developments. New systems have emerged, and existing systems have evolved. With very recent developments  \cite{istio-merbridge, cilium-mesh} (less than two months old at the time of writing) paving the way for fundamentally different approaches, the field is as exciting as it gets.

This all results in a justification for conducting the survey, and that the timing of it manages to capture an interesting, evolutionary shift within the field.

