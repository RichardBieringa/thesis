\section{Survey Objectives}
\label{sec:survey:objectives}
% What are we trying to achieve in this survey
% - Bridge the gap between academia and industry
% - Gain insights into the current state-of-the-art-systems
% - Obtain knowledge about relative metrics and workloads 
% - ^> Is a stepping stone to the next chapter

% - Tie it all to the research question defined in the introduction

Before conducting a systematic systems survey, we first establish a set of objectives which we will try to achieve in this work. This allows us to structure the research and adjust its methodologies based on the objectives. The following list represents the formulated objectives.

\begin{enumerate}[label=\textbf{O\arabic*}, leftmargin=3\parindent]
    \item \textbf{Bridge the gap between academia and industry.}
    \label{obj:survey:1}
    
    In the previous section (\cref{sec:survey:introduction}), we identified that there is a large gap between interest from academia and the industry. We identified that there is a lack of formal publications in the field, whereas preliminary exploration showed that this is in sharp contrast to the interest as expressed from the industry. This leads to this first objective, which is to close this gap and provide formal research in this emerging field.
    
    \item \textbf{Gain insight into the current state-of-the-art \gls{sm} systems.}
    \label{obj:survey:2}
    
    The second objective of this survey is to gain a clear understanding of the field and the current state-of-the-art \gls{sm} systems. We want to identify the systems that currently exist and identify their defining properties and characteristics.

    \item \textbf{Obtain knowledge about relevant performance related intricacies.}
    \label{obj:survey:3}
    
    Finally, we want to identify and analyse the properties and components that capture the performance implications of these systems. By examining these systems in detail, and by taking a closer look at the design of these systems, we can hypothesize on the performance complications of design decisions. This is important for the later parts of our work so that we can analyse the experimental results and relate back to architectural differences uncovered.
    
\end{enumerate}

% - Tie it all to the research question defined in the introduction
% - Tie it to the general flow of the thesis as this provides a stepping stone to further chapters
The objectives and their respective order in which they are  presented are closely tied to the general flow of research conducted within this thesis. We first established the gap between industry and academia in our previous work and formulated this thesis around the idea to close this \ref{obj:survey:1}. Furthermore, we apply best practices by conducting a survey of the field. This helps us to understand the field and allows us to examine the current state-of-the-art systems in detail \ref{obj:survey:2}. The data synthesis of this survey allows us to establish a framework to compare the existing and future \gls{sm} systems which might emerge. The results of this will be used to provide the answer to the research question \ref{rq-1} during our analysis. Additionally, the survey and resulting framework allows us to determine relevant metrics and components to measure \ref{obj:survey:3}. This provides a stepping stone towards the following chapter, in which we design an instrument to compare the performance characteristics of these systems (\cref{chap:system-design}).