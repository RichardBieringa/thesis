\section{A Distributed Systems Approach}
\label{sec:approach}

% Covers methodology of this thesis
In this thesis, we approach the problem statement and their resulting research questions using distributed systems approach, a consolidation of best practices as taught and utilized by the members AtLarge group\footnote{\url{https://atlarge-research.com/}}. This approach follows the vision of \gls{mcs} \cite{iosup2018} in which it guides the thesis and the conceptual, technical and experimental work that it consist of.

First, to address \ref{rq-1} we perform an extensive research into the state-of-the-art \gls{sm} systems. To accomplish this, we perform a systematic systems survey. This will help us to identify various \gls{sm} systems that exist and helps us to understand these systems in more detail. Additionally, it allows us to identify key characteristics of these systems in the form of \glspl{fr} and \glspl{nfr}. Based on these findings we can construct a framework, that represents an overview of the identified \gls{sm} systems and allows us to compare them to one another. This part of the research helps to establish the groundwork, in which we identify the key components and architectures that influence the performance characteristics.

Secondly, we tackle \ref{rq-2} based on the key findings from \ref{rq-1}. With an extensive understanding of \gls{sm} systems, we aim to design a benchmark instrument that is capable of evaluating the performance characteristics. We perform an extensive requirements analysis, in which we establish the stakeholders for such a benchmarking instrument and their respective use cases. Once these requirements are clearly defined we guide our design through established best practices that helps us to capture the most significant metrics in evaluating distributed systems. After this, we implement a prototype and validate the design requirements.


Ultimately, to address \ref{rq-3} we use the instrument as devised from \ref{rq-2}. We design experiments and workloads that evaluate the \gls{sm} systems under various conditions. We construct an experimental environment that aims to minimize external impacts on the results of these studies and measure the variation of our experimental environment by performing various micro-benchmarks. The extensive experimental approach follows the vision of \gls{mcs} and aims to construct and perform reproducible experiments that can answer the performance related impacts of \gls{sm} systems.