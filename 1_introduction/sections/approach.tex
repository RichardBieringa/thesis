\section{A Distributed Systems Approach}
\label{sec:approach}

% Covers methodology of this thesis

In this thesis, we approach the problem statement and their resulting research questions using distributed systems approach, a consolidation of best practices as taught and utilized by the members AtLarge group \footnote{\url{https://atlarge-research.com/}}. This guides the thesis through an approach based on conceptual, technical and experimental work that it consist of.

First, to address \ref{rq-1} we investigate the state-of-the-art in current \gls{sm} implementations. In order to do this, we perform a systematic systems survey by applying common literature survey best practices. This will help us identify key characteristics of these systems and gives us an insight on how they work. Furthermore, it allows us to create an overview of the current landscape of \gls{sm} systems and a framework of what a state-of-the-art system should look like. 

Secondly, we tackle \ref{rq-2} based on the previously identified characteristics of state-of-the-art systems from \ref{rq-1}. With this information in mind, we perform a requirements analysis for a benchmarking instrument capable of evaluating these systems in order to capture their core capabilities. We compare existing solutions to these requirements and either design or extend an instrument based on these requirements and constraints. Finally, we build a prototype of this instrument according to the aforementioned design.

Ultimately, to address \ref{rq-3} we use the instrument as devised from \ref{rq-2}. We design experiments and workloads to capture the performance characteristics and \glspl{nfr} of state-of-the-art \gls{sm} systems. We then use a cloud based environment to conduct these designed experiments on using an automated workflow. Using this methodology, we can systematically conduct real world experiments that which are reproducible. Finally, we analyze the quantitative results obtained from the experiments in order to objectively compare the \gls{sm} systems.