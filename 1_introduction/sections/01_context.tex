\section{Context}
\label{sec:introduction:context}
% Societal impact
% Economic -> Impact for companies (e2e latencies)
% Society at large -> We rely on infra (shift in responsibilies)
% Companies involved



% Field imapct (Intellectual Argument)
% Industry -> Considerable resouces / Relatively little scientific output
% Shift from libary approach -> service meshes
% Emerging field, new tech (ebpf)
% Lots of new systems (OSM, Cilium)


% Why should people care about the  work
We live in the \textit{Age of Computer Ecosystems} \cite{Iosup2018}, where our society as a whole relies on the modern IT infrastructure. From governments to schools and businesses, many use the internet as a gateway to their services and offerings. The design of modern IT systems and infrastructure therefore has a large societal impact, for both the end user, the providing entities, and the every increasing number of engineers \cite{it-employment-worldwide} that are required to develop and manage it.

Businesses that rely on IT infrastructure, can face major economic impact through slight design alterations. Amazon, for example, found that for every 100ms of additional end-to-end latency, it would lose out on roughly 1\% of total sales revenue \cite{amazon-latency-conversion-drop}. And Google, indicated back in 2009 \cite{google-latency-bounce-rate} that latency has a direct relation with service usage and has been actively prioritizing performance of services ever since

In this thesis, we evaluate the \gls{sm} architecture and systems that implement this. These systems introduce an additional layer of machinery that naturally causes a performance overhead. Even though, the performance complications can have major impact, the systems in question have received lots of attention from the industry and increasing number of companies are evaluating and using such solutions in their IT infrastructure \cite{cncf-survey-2021}.  With large corporations, such as Google and Microsoft backing the development of \gls{sm} systems it is clear that this technology has lots of potential. 

The \gls{sm} landscape is a rapidly evolving field, that is in its relative infancy.  There have been numerous new systems that have emerged in the last couple of years and there have been many technological advancements that pave the way for alternative approaches to the problem. It comprises a landscape where most of the systems are in active development and see frequent improvements and architectural changes \cite{istio-merbridge, cilium-mesh}. This grants us the opportunity to evaluate the different approaches these systems take, as they evolve and explore an exciting field of bleeding-edge technologies.

Although \gls{sm} systems have gained a lot of attention and hype from within the industry. It has received relatively little to no attention from academia as observed by our previous works in which we have conducted a topical survey of the field. In this thesis, we aim to close that gap by performing an extensive study on \gls{sm} systems, with a focus on their performance characteristics.

% MCS relation
Throughout this thesis, we align ourselves with the goals and visions expressed in \gls{mcs} \cite{Iosup2018}. The work as presented in this thesis aims to uncover the intricacies of the computer ecosystems discussed, helping to understand them and guide future system designs. \gls{mcs} envisions a domain where everything developed is tested and benchmarked in a reproducible manner. This ideology is exemplified by the experimental approach used  throughout this thesis.




