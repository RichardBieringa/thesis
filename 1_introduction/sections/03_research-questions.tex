\section{Main Research Questions}
\label{sec:introduction:research-questions}

We decompose the challenges as described in the problem statement \cref{sec:introduction:problem-statement} in the following research questions.


% Research Questions
% Map the research questions to the chapters/contributions
% 1 RQ per chapter
% - Systems Survey 
% - Design/implementation of a benchmarking system
% - Real World Experiments
\begin{enumerate}[label=\textbf{RQ\arabic*}, leftmargin=3\parindent]
    \item \textbf{How to compare, and evaluate \gls{sm} systems?}
    \label{rq-1}
    
    Many \gls{sm} implementations have emerged from within the rapidly changing landscape of container and resource management technologies, before the adaptation of any standard to guide their development. It is necessary to evaluate these implementations in a systematic systems survey, where we identify and analyse the characteristics of the current iterations of \gls{sm} systems. This allows us to create a comparison framework, that acts an overview of the state-of-the-art within the \gls{sm} landscape.
    
    \item \textbf{How to design and implement a benchmark that evaluates the performance of \gls{sm} systems?}
    \label{rq-2}
    
    Based on the results of the system survey in \ref{rq-1}, we identify the requirements for a system which that can quantitively evaluate the performance characteristics of a \gls{sm} system. Based on these identified requirements, we should evaluate existing instruments to see if any of these satisfy. We then have to design or extend an instrument guided by benchmarking best practices \cite{folkerts2012benchmarking}. The benchmarking instrument should align with the goals and visions of \gls{mcs} in which we aim to produce systematic and reproducible experiments.
    
    \item \textbf{What are the differences between current \gls{sm} systems in terms of overhead, throughput and latency?}
    \label{rq-3}
    
    Based on the benchmark as designed in \ref{rq-2}, we have to conduct performance oriented experiments on different \gls{sm} systems. We should design and perform experiments that simulate various workload patterns for such a system. The experiments should evaluate \gls{sm} systems while they experience various levels of load in the form of throughput. The experiments should also capture the effects of various payload types and sizes. Finally, the experiments should be reproducible so that anyone can verify the results.
\end{enumerate}

% Old ones
% \subsection{Research Questions}
% \begin{enumerate}
%   \item What does the term 'service mesh' mean in the context of distributed systems?
%   \item What are the defining properties and characteristics that define the service mesh?
%   \item How can we model the performance and reliability characteristics of a service mesh?
%   \item How can we conduct real world experiments on different service mesh implementations to objectively compare them?
% \end{enumerate}

