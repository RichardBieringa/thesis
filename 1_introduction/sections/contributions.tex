\section{Main Contributions}
\label{sec:contributions}

This thesis introduces qualitative, quantitative and software contributions that answer the formulated research questions. Each contribution is linked to a main research question. The contributions are as follows:

\todo{Contributions: Check formulation of contributions}
\todo{Contributions: Fix references to chapters}

% Emphasize why these concepts are challenging
% What is the usefulness of the results?
% Software Product -> Artifact
% Experiments -> Artifact

\begin{enumerate}
    \item (\textit{Conceptual}, \ref{rq-1}) A qualitative survey on the characteristics of current state-of-the-art \gls{sm} systems \cref{chap:survey}.
    
    \item (\textit{Conceptual}, \ref{rq-1}) Analysis of current state-of-the-art \gls{sm} systems and their conformance with current proposed industry standards \cref{chap:survey}.    
    
    
    \item (\textit{Conceptual}, \ref{rq-2}) Requirements analysis and design of a benchmark instrument capable of performing systematic and reproducible real world experiments on \gls{sm} systems \cref{chap:survey}.  
    
    \item (\textit{Technical}, \ref{rq-2}) Technical implementation of a benchmarking instrument based on the former requirements in the form of open-source software. \cref{chap:survey}.


    \item (\textit{Experimental}, \ref{rq-3}) Design execution of real world experiments on cloud environments \cref{chap:survey}.
    
    \item (\textit{Experimental}, \ref{rq-3}) Quantitive results and analysis of state-of-the-art \gls{sm} systems \cref{chap:survey}.
\end{enumerate}


 
