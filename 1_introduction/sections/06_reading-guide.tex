\section{Reading Guide}
\label{sec:introduction:reading-guide}

\todo{Include Figure}

This chapter briefly introduced the impact of system design and how current approaches evolved and why. It then discussed the recent emergence of \gls{sm} systems and the challenges it tries to solve. However, it briefly touches upon many subjects that are part of the bigger picture that is addressed throughout the work in this thesis. It is advisable to read the contents of this thesis from the front to the back. The order of the material discussed builds on top of another and is prefaced by a background chapter that introduces several concepts in greater detail. These concepts are used throughout the entirety of the thesis and the sections in the background chapter provide the reader with enough knowledge to understand the materials discussed.

The following chapter (\cref{chap:background}), introduces some of these concepts in greater detail. In \cref{sec:background:containers} we introduce the notion of a \textit{container} in more detail and how it had a significant impact on cost reductions for deployments in comparison to earlier models. In \cref{sec:background:soa} we discuss several evolutions in architectural design and why it gained traction. We introduce the notion of a service-oriented approach to software design and the challenges that it brings. \cref{sec:background:kubernetes} briefly introduces \gls{k8s} and several related concepts that return in architectural designs (in the system survey analysis \cref{sec:survey:analysis}) and experiments (in the experimental environment \cref{sec:experiments:design}). Following that, in \cref{sec:background:service-mesh} we introduce the \gls{sm} architecture in greater detail. It describes a general \gls{sm} system and the challenges it tries to solve. Furthermore, it gives a comparison to related solutions that aim to solve similar challenges. Thereafter, in \cref{sec:background:cncf} we discuss the \glsfirst{cncf} a governing body in the field. Throughout this thesis, we use several projects maintained by the foundation and use several of their surveys, as key insights. Finally, in \cref{sec:background:related-work} we introduce the related work in this field.