% Setup
\graphicspath{./figures}

\chapter{Introduction}
\label{chap:introduction}
% Why is the topic being studied?
% How is the topic being studied?
% What is being studied?



% Structure
% Hook
% Background
% What is this thesis about/problem
% What is being presented in this thesis

How to design and build the IT infrastructure of the present, and of the future? A difficult, but important question in the era distinguished by globalization and digitalization. More and more of society is relying on the internet, and its wide variety of products and service offerings. From online stores, to productivity tools, streaming services to news websites, modern IT infrastructure empowers many areas and is a staple of the global economy. This was amplified and made abundantly clear in the last couple of years, where the world as a whole, moved online to prevent the further spread of COVID-19.



% Customer experience
% Growing cybersecurity, data privacy threats
% Fierce competition for talent
There are many designs and approaches that are currently in use within the field \cite{design-example-monolith, design-example-four-tier, design-example-microservices, design-example-serverless}. So far, there has not been a single approach, a golden architecture, that answers all the toughest challenges that we are facing today. How to minimize the end-to-end latencies, and improve reliability to improve the seamless end-user experience? How to deal with the growing threats of cybersecurity and the rise of data privacy threats? How to optimize the productivity of developers so that they can focus on business logic instead of infrastructure related concerns? These are just a few of the most pressing challenges \cite{modern-it-challenges}, that should be considered when designing and building a modern, distributed system. 

% Model of deployment
% Microservices architecture
Recent trends and developments in deployment models have paved the way for the currently popular \textit{microservices architecture}. An architectural style characterized by separating applications in many small, independent parts, that have their own responsibilities. This style of architecture is made feasible by the low cost of computational overhead caused by container technologies and emerging workload managers such as \gls{k8s}. With most of the organizations (71\% as observed in 2021 by Statista \cite{microservices-adoption}) adopting or using a microservices architecture in some for, it is important to analyse the benefits and especially the shortcomings of such an approach.

% Communication complexities
The architectural style has many benefits, such as the ability to decompose complex problems into smaller ones and allow teams to work on decoupled services. It also enables better scaling, as individual components can now be replicated instead of entire applications. However, it also introduces a new layer of complexity, that comes in the form of service-to-service communications. Networking brings numerous challenges, it introduces additional latencies and poses security threats. The architectural style has to deal with routing and load balancing and consider the possibility of failing links in inherently unreliable networks.

% Software lib -> sm approach
For many years, companies approached this set of problems by changing the application code responsible for service-to-service communications, often in the form of client libraries that did so in a uniform manner. However, recent trends and community-wide efforts empowered the rise of the \gls{sm} architecture. This architectural paradigm aims to solve similar challenges, without having to modify any application code or having to stick to a single programming language or framework. The architecture works by introducing a dedicated infrastructure layer, that facilitates service-to-service communications, achieved by adding network \textit{proxies} in front of the logical services. These proxies intercept communications from and to these services, and provides a uniform approach of dealing with service-to-service traffic, enabling observability into communications, secure connections and reliability mechanisms. 


% Work presented in this thesis
% Systems Survey -> Qualitative Study
% Mesh Bench -> Benchmark design/implementation
% Experimental Evaluation -> Insights into the performance overheads and promising directions
Within this emerging field of \gls{sm} systems, we focus on the performance implications of these systems. What is the computational overhead of the additional machinery introduced, and what is the impact of latency through the additional network hops? In this thesis, we present an extensive study into the performance characteristics of \gls{sm} systems. First, we conduct a systems survey, in which we identify and analyse the state-of-the-art \gls{sm} systems. After this, we design and introduce \textit{Mesh Bench}, a benchmarking instrument that can evaluate the performance characteristics of \gls{sm} systems. Finally, with a focus on reproducibility, we design, perform and analyse  performance related experiments of \gls{sm} systems.


% Sections 
\section{Context}
\label{sec:introduction:context}
% Societal impact
% Economic -> Impact for companies (e2e latencies)
% Society at large -> We rely on infra (shift in responsibilies)
% Companies involved



% Field imapct (Intellectual Argument)
% Industry -> Considerable resouces / Relatively little scientific output
% Shift from libary approach -> service meshes
% Emerging field, new tech (ebpf)
% Lots of new systems (OSM, Cilium)


% Why should people care about the  work
We live in the \textit{Age of Computer Ecosystems} \cite{Iosup2018}, where our society as a whole relies on the modern IT infrastructure. From governments to schools and businesses, many use the internet as a gateway to their services and offerings. The design of modern IT systems and infrastructure therefore has a large societal impact, for both the end user, the providing entities, and the every increasing number of engineers \cite{it-employment-worldwide} that are required to develop and manage it.

Businesses that rely on IT infrastructure, can face major economic impact through slight design alterations. Amazon, for example, found that for every 100ms of additional end-to-end latency, it would lose out on roughly 1\% of total sales revenue \cite{amazon-latency-conversion-drop}. And Google, indicated back in 2009 \cite{google-latency-bounce-rate} that latency has a direct relation with service usage and has been actively prioritizing performance of services ever since

In this thesis, we evaluate the \gls{sm} architecture and systems that implement this. These systems introduce an additional layer of machinery that naturally causes a performance overhead. Even though, the performance complications can have major impact, the systems in question have received lots of attention from the industry and increasing number of companies are evaluating and using such solutions in their IT infrastructure \cite{cncf-survey-2021}.  With large corporations, such as Google and Microsoft backing the development of \gls{sm} systems it is clear that this technology has lots of potential. 

The \gls{sm} landscape is a rapidly evolving field, that is in its relative infancy.  There have been numerous new systems that have emerged in the last couple of years and there have been many technological advancements that pave the way for alternative approaches to the problem. It comprises a landscape where most of the systems are in active development and see frequent improvements and architectural changes \cite{istio-merbridge, cilium-mesh}. This grants us the opportunity to evaluate the different approaches these systems take, as they evolve and explore an exciting field of bleeding-edge technologies.

Although \gls{sm} systems have gained a lot of attention and hype from within the industry. It has received relatively little to no attention from academia as observed by our previous works in which we have conducted a topical survey of the field. In this thesis, we aim to close that gap by performing an extensive study on \gls{sm} systems, with a focus on their performance characteristics.

% MCS relation
Throughout this thesis, we align ourselves with the goals and visions expressed in \gls{mcs} \cite{Iosup2018}. The work as presented in this thesis aims to uncover the intricacies of the computer ecosystems discussed, helping to understand them and guide future system designs. \gls{mcs} envisions a domain where everything developed is tested and benchmarked in a reproducible manner. This ideology is exemplified by the experimental approach used  throughout this thesis.2





\section{Problem Statement}
\label{sec:introduction:problem-statement}

% Intro problem 
% - Fast growing, but immature
% - Operational & performance challenges
% - - SMI / SMP
% - Non Functional Requirements
% - How to compare systems
% - Balance between performance/reliability


\Gls{sm} systems form a relatively new and exciting field, that is rapidly evolving. However, it is also a field where there is very limited output from the academic world. Specifically, there is very little known about the performance implications of these systems. The goal of this thesis is to answer this unsolved challenge, and perform an extensive study into performance implications of \gls{sm} systems. 

Although there has not been much attention from academia, prior efforts originating from the industry have resulted in a tool \cite{kinvolk-bench} that can capture performance overheads in \gls{sm} systems. This tool lacks in depth and scope, but has laid some groundwork by producing early performance results \cite{bench-istio-linkerd-2019, bench-istio-linkerd-2021}. 


To fully understand the performance implications of \gls{sm} systems, it is important to know how these systems function under the hood and how they differ from one another. Once we understand these systems in detail, we have to design a benchmark that captures the performance characteristics of \gls{sm} systems. With a benchmark system capable of doing that, we have to design performance oriented experiments that evaluate \gls{sm} systems under various synthetic loads, that emulate common real-world scenarios.



% The \gls{sm} architecture provides modern, distributed systems best practices and tries to solve and minimize the complexities of application developers and system operators alike. It is an area of interest that is actively being worked on in the industry in a rapidly evolving landscape. Although it has gained a lot of interest in the last decade, the technology is in its relative infancy and is facing several challenges from both operational and performance related standpoints. Since it implements a layer of networking abstraction, it creates a potential for a programmable interface, which in turn allows for application and service aware traffic management capabilities. This can differentiate different service mesh implementations based on their capabilities and \glspl{nfr}, where some implementations can support advanced feature sets like circuit breaking behaviour while others do not. Collaborative efforts have are established to create a standard for service meshes \cite{service-mesh-interface, service-mesh-interface-spec}, resulting in the \gls{smi}, a standard interface for service mesh implementations on Kubernetes that describes the most common service mesh capabilities, such as traffic policies, traffic telemetry and traffic management. However, the proposed standard is relatively new, primitive and imposes the question if any existing implementation even accounts for this interface as preliminary research done by us suggests that it does not. Another aspect that is of importance is the performance characteristics of these implementations. How can we characterize the performance characteristics of a \gls{sm} so that we can objectively compare implementations. Through joint efforts, a \gls{cncf} project on a performance specification dubbed \gls{smp} has also been established \cite{service-mesh-performance, service-mesh-interface-spec}. This specification, however, is even newer than the \gls{smi} standard and at the time of writing consists of only the most basic and generic measurements available.

% % Challenges (three chapters/RQ/contributions)
% % - Systems Survey -> Characteristics/SMI spec comparison
% % - Design/Extend perforamnce SMP Spec
% % - Conduct real world experiments
% The goal of this master thesis is to explore the performance characteristics of \gls{sm} implementations using a distributed systems approach. Although, as mentioned, the community has made initial efforts to solve these challenges, the solutions are far from complete nor capture the core elements to fully understand the performance implications of these systems. Also, since both aforementioned collaborative efforts are merely standards and do not carry an official implementation, it can merely act as a guidance in this research. The first challenge is to evaluate existing \gls{sm} implementations according to their characteristics and compare them to their proposed standard. The second challenge is to evaluate the existing performance standard, and if required, design or extend it to capture the core capabilities of a \gls{sm}. Given the many use cases for \gls{sm} systems and the different environments that one such system can run in, it is important to capture all the elements of such a system so that an objective comparison can be made. The final challenge will be to conduct actual experiments that capture these objective results reliably. Given the very dynamic nature of distributed systems, creating such an environment to evaluate this can be a challenge of its own. All of these challenges will guide the research in this thesis in order to evaluate existing \gls{sm} systems.
\section{Main Research Questions}
\label{sec:research-questions}

We decompose the challenges as described in the problem statement \cref{sec:problem-statement} in the following research questions.


% Research Questions
% Map the research questions to the chapters/contributions
% 1 RQ per chapter
% - Systems Survey 
% - Design/implementation of a benchmarking system
% - Real World Experiments

\todo{Research Questions: Formulation}

\begin{enumerate}[label=\textbf{RQ\arabic*}, leftmargin=3\parindent]
    \item \textbf{How to compare, and evaluate \gls{sm} systems?}
    \label{rq-1}
    
    Many \gls{sm} implementations have emerged from within the rapidly changing landscape of container and resource management technologies, before the adaptation of any standard to guide their development. It is necessary to evaluate these implementations in a structured systems survey, where we identify and analyse the characteristics of the current iterations of \gls{sm} systems. This allows us to create an overview of the state-of-the-art within the \gls{sm} landscape.
    
    \item \textbf{How to design and implement a benchmark that evaluates the performance of \gls{sm} systems?}
    \label{rq-2}
    
    Based on the results of the system survey in \ref{rq-1}, we will identify the requirements for a system which can evaluate a service mesh. Based on these identified requirements, we should evaluate existing instruments to see if any of these satisfy. We then have to design or extend an instrument based on best practices \cite{folkerts2012benchmarking} which can support multiple experiment configuration, workloads and can produce reproducible real-world experiments. The goal of this instrument is to capture the relevant system metrics, performance metrics and application domain specific \glspl{nfr} which results in a generic and objective comparison between said systems.

    \item \textbf{What are the differences between the different \gls{sm} systems in terms of overhead, throughput and latency?}
    \label{rq-3}
    
    Based on the benchmark as designed in \ref{rq-2}, we have to conduct real-world experiments on different \gls{sm} systems. The benchmark should compare popular \gls{sm} implementations and implementations that vary based on their architectural design. The benchmark should compare different application workloads, highlighting various application-specific aspects of the system where different implementations can vary in performance or capabilities. Finally, the benchmark should be as systematic as possible, meaning that anyone can verify the results by reproducing the experiments bearing in mind the inherent nature of variance within distributed systems.

\end{enumerate}

% Old ones
% \subsection{Research Questions}
% \begin{enumerate}
%   \item What does the term 'service mesh' mean in the context of distributed systems?
%   \item What are the defining properties and characteristics that define the service mesh?
%   \item How can we model the performance and reliability characteristics of a service mesh?
%   \item How can we conduct real world experiments on different service mesh implementations to objectively compare them?
% \end{enumerate}


\section{A Distributed Systems Approach}
\label{sec:approach}

% Covers methodology of this thesis
In this thesis, we approach the problem statement and their resulting research questions using distributed systems approach, a consolidation of best practices as taught and utilized by the members AtLarge group\footnote{\url{https://atlarge-research.com/}}. This approach follows the vision of \gls{mcs} \cite{iosup2018} in which it guides the thesis and the conceptual, technical and experimental work that it consist of.

First, to address \ref{rq-1} we perform an extensive research into the state-of-the-art \gls{sm} systems. To accomplish this, we perform a systematic systems survey. This will help us to identify various \gls{sm} systems that exist and helps us to understand these systems in more detail. Additionally, it allows us to identify key characteristics of these systems in the form of \glspl{fr} and \glspl{nfr}. Based on these findings we can construct a framework, that represents an overview of the identified \gls{sm} systems and allows us to compare them to one another. This part of the research helps to establish the groundwork, in which we identify the key components and architectures that influence the performance characteristics.

Secondly, we tackle \ref{rq-2} based on the key findings from \ref{rq-1}. With an extensive understanding of \gls{sm} systems, we aim to design a benchmark instrument that is capable of evaluating the performance characteristics. We perform an extensive requirements analysis, in which we establish the stakeholders for such a benchmarking instrument and their respective use cases. Once these requirements are clearly defined we guide our design through established best practices that helps us to capture the most significant metrics in evaluating distributed systems. After this, we implement a prototype and validate the design requirements.


Ultimately, to address \ref{rq-3} we use the instrument as devised from \ref{rq-2}. We design experiments and workloads that evaluate the \gls{sm} systems under various conditions. We construct an experimental environment that aims to minimize external impacts on the results of these studies and measure the variation of our experimental environment by performing various micro-benchmarks. The extensive experimental approach follows the vision of \gls{mcs} and aims to construct and perform reproducible experiments that can answer the performance related impacts of \gls{sm} systems.
\section{Main Contributions}
\label{sec:introduction:contributions}

This thesis introduces several qualitative, quantitative and technical contributions that assist in answering the formulated research questions. Each contribution is linked to a main research question. The contributions are as follows:


% Emphasize why these concepts are challenging
% What is the usefulness of the results?
% Software Product -> Artifact
% Experiments -> Artifact

\begin{enumerate}
    \item (\textit{Conceptual}, \ref{rq-1}) A qualitative systems survey on the characteristics of state-of-the-art \gls{sm} systems \cref{chap:survey}.
    
    \item (\textit{Conceptual}, \ref{rq-1}) Extensive analysis on the design and architectural approach of state-of-the-art \gls{sm} systems \cref{chap:survey}.    
    
    \item (\textit{Conceptual}, \ref{rq-2}) Requirements analysis and design of a \gls{sm} benchmarking instrument  \cref{chap:system-design}.  
    
    \item (\textit{Technical}, \ref{rq-2}) Prototype implementation  of the designed benchmarking instrument, \textit{Mesh Bench} \cref{chap:system-design}.

    \item (\textit{Experimental}, \ref{rq-3}) Design and deployment of performance oriented experiments that evaluate \gls{sm} systems under various environments \cref{chap:experimental-evaluation}.
    
    \item (\textit{Experimental}, \ref{rq-3}) Quantitative performance results and extensive analysis of \gls{sm} systems \cref{chap:experimental-evaluation}.
\end{enumerate}


 

\section{Reading Guide}
\label{sec:introduction:reading-guide}

\todo{Include Figure}

This chapter briefly introduced the impact of system design and how current approaches evolved and why. It then discussed the recent emergence of \gls{sm} systems and the challenges it tries to solve. However, it briefly touches upon many subjects that are part of the bigger picture that is addressed throughout the work in this thesis. It is advisable to read the contents of this thesis from the front to the back. The order of the material discussed builds on top of another and is prefaced by a background chapter that introduces several concepts in greater detail. These concepts are used throughout the entirety of the thesis and the sections in the background chapter provide the reader with enough knowledge to understand the materials discussed.

The following chapter (\cref{chap:background}), introduces some of these concepts in greater detail. In \cref{sec:background:containers} we introduce the notion of a \textit{container} in more detail and how it had a significant impact on cost reductions for deployments in comparison to earlier models. In \cref{sec:background:soa} we discuss several evolutions in architectural design and why it gained traction. We introduce the notion of a service-oriented approach to software design and the challenges that it brings. \cref{sec:background:kubernetes} briefly introduces \gls{k8s} and several related concepts that return in architectural designs (in the system survey analysis \cref{sec:survey:analysis}) and experiments (in the experimental environment \cref{sec:experiments:design}). Following that, in \cref{sec:background:service-mesh} we introduce the \gls{sm} architecture in greater detail. It describes a general \gls{sm} system and the challenges it tries to solve. Furthermore, it gives a comparison to related solutions that aim to solve similar challenges. Thereafter, in \cref{sec:background:cncf} we discuss the \glsfirst{cncf} a governing body in the field. Throughout this thesis, we use several projects maintained by the foundation and use several of their surveys, as key insights. Finally, in \cref{sec:background:related-work} we introduce the related work in this field.
