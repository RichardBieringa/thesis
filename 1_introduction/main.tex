% Setup
\graphicspath{./figures}

\chapter{Introduction}
\label{chap:introduction}

% Introduction

A \gls{sm} is a dedicated layer of infrastructure that sits between services and aims to provide several distributed systems best practices in a uniform manner. The terminology was introduced in the last decade, and several implementations have seen the light of day since. The premise of introducing additional observability in the system, increasing the reliability of it while also adding security best practices to little or no additional lines of application code made this technology a hot topic in the field. However, this technology does not come in the form of a zero cost offering. Most of the service mesh designs work by introducing additional network proxies and an additional control plane. By introducing these components into the system, an increase in system resources is to be expected. More importantly, the introduction of a proxy between each service leads to additional latencies in every service-to-service communication. 

The introduction of additional latencies in service-to-service communications or the increase in system resources can be a make or break scenario in many aspects. Companies and organizations for example can have \gls{kpi} based on the end-to-end response times of the services they provide. These \gls{qos} based metrics are important and drive their business decisions. Can the introduction of the benefits provided by a \gls{sm} outweigh the additional latency introduced, or more importantly, how much latency will be introduced at all? Another scenario that comes to mind comes in the form of resource constrained environments, which can often occur in \gls{iot} or \gls{edge-computing} topologies. 

Although the technology has been in the spotlight in the last decade and the technology has seen a steady increase in adoption \cite{cncf-survey-2020, rh-survey} and implementations have seen to matured, there has been a lack of performance studies that evaluate these systems in depth. Previous works have established the groundwork by introducing simple benchmarking instruments \cite{kinvolk-bench} and their preliminary results \cite{bench-istio-linkerd-2019, bench-istio-linkerd-2021}. These benchmarks and instruments, however, do not capture the unique characteristics of a \gls{sm} well enough.

\todo{Fix intro last p (END)}
In this thesis, we present x, an automated service mesh evaluation benchmark that makes it easy to conduct reproducible real world experiments on different \gls{sm} implementations. With this instrument, we conduct experiments on existing \gls{sm} implementation to compare them on key performance characteristics.

% Sections 
% Why should people care about the  work


% Societal impact
% Economic -> Impact for companies (e2e latencies)
% Society at large -> We rely on infra (shift in responsibilies)
% Companies involved



% Field imapct (Intellectual Argument)
% Industry -> Considerable resouces / Relatively little scientific output
% Shift from libary approach -> service meshes
% Emerging field, new tech (ebpf)
% Lots of new systems (OSM, Cilium)
\input{1_introduction/sections/problem-statement}
\input{1_introduction/sections/research-questions}
\section{A Distributed Systems Approach}
\label{sec:approach}

% Covers methodology of this thesis

In this thesis, we approach the problem statement and their resulting research questions using distributed systems approach, a consolidation of best practices as taught and utilized by the members AtLarge group \footnote{\url{https://atlarge-research.com/}}. This guides the thesis through an approach based on conceptual, technical and experimental work that it consist of.

First, In order to address \ref{rq-1} we investigate the state-of-the-art in current \gls{sm} implementations. In order to do this, we perform a systematic systems survey by applying common literature survey best practices. This will help us identify key characteristics of these systems and gives us an insight on how they work. Furthermore, it allows us to create an overview of the current landscape of \gls{sm} systems and a framework of what a state-of-the-art system should look like. 

Secondly, we tackle \ref{rq-2} based on the previously identified characteristics of state-of-the-art systems from \ref{rq-1}. With this information in mind, we perform a requirements analysis for a benchmarking instrument capable of evaluating these systems in order to capture their core capabilities. We compare existing solutions to these requirements and either design or extend an instrument based on these requirements and constraints. Finally, we build a prototype of this instrument according to the aforementioned design.

Ultimately, to address \ref{rq-3} we use the instrument as devised from \ref{rq-2}. We design experiments and workloads to capture the performance characteristics and \glspl{nfr} of state-of-the-art \gls{sm} systems. We then use a cloud based environment to conduct these designed experiments on using an automated workflow. Using this methodology, we can systematically conduct real world experiments that which are reproducible. Finally, we analyze the quantitative results obtained from the experiments in order to objectively compare the \gls{sm} systems.
\section{Main Contributions}
\label{sec:contributions}

This thesis introduces qualitative, quantitative and software contributions that answer the formulated research questions. Each contribution is linked to a main research question. The contributions are as follows:

\todo{Contributions: Check formulation of contributions}
\todo{Contributions: Fix references to chapters}

% Emphasize why these concepts are challenging
% What is the usefulness of the results?
% Software Product -> Artifact
% Experiments -> Artifact

\begin{enumerate}
    \item (\textit{Conceptual}, \ref{rq-1}) A qualitative survey on the characteristics of current state-of-the-art \gls{sm} systems \cref{chap:survey}.
    
    \item (\textit{Conceptual}, \ref{rq-1}) Analysis of current state-of-the-art \gls{sm} systems and their conformance with current proposed industry standards \cref{chap:survey}.    
    
    
    \item (\textit{Conceptual}, \ref{rq-2}) Requirements analysis and design of a benchmark instrument capable of performing systematic and reproducible real world experiments on \gls{sm} systems \cref{chap:survey}.  
    
    \item (\textit{Technical}, \ref{rq-2}) Technical implementation of a benchmarking instrument based on the former requirements in the form of open-source software. \cref{chap:survey}.


    \item (\textit{Experimental}, \ref{rq-3}) Design execution of real world experiments on cloud environments \cref{chap:survey}.
    
    \item (\textit{Experimental}, \ref{rq-3}) Quantitive results and analysis of state-of-the-art \gls{sm} systems \cref{chap:survey}.
\end{enumerate}


 

\input{1_introduction/sections/reading-guide}
