% Setup
\graphicspath{./figures}

\chapter{Introduction}
\label{chap:introduction}

% Introduction

A \gls{sm} is a dedicated layer of infrastructure that sits between services and aims to provide several distributed systems best practices in a uniform manner. The terminology was introduced in the last decade, and several implementations have seen the light of day since. The premise of introducing additional observability in the system, increasing the reliability of it while also adding security best practices to little or no additional lines of application code made this technology a hot topic in the field. However, this technology does not come in the form of a zero cost offering. Most of the service mesh designs work by introducing additional network proxies and an additional control plane. By introducing these components into the system, an increase in system resources is to be expected. More importantly, the introduction of a proxy between each service leads to additional latencies in every service-to-service communication. 

The introduction of additional latencies in service-to-service communications or the increase in system resources can be a make or break scenario in many aspects. Companies and organizations for example can have \gls{kpi} based on the end-to-end response times of the services they provide. These \gls{qos} based metrics are important and drive their business decisions. Can the introduction of the benefits provided by a \gls{sm} outweigh the additional latency introduced, or more importantly, how much latency will be introduced at all? Another scenario that comes to mind comes in the form of resource constrained environments, which can often occur in \gls{iot} or \gls{edge-computing} topologies. 

Although the technology has been in the spotlight in the last decade and the technology has seen a steady increase in adoption \cite{cncf-survey-2020, rh-survey} and implementations have seen to matured, there has been a lack of performance studies that evaluate these systems in depth. Previous works have established the groundwork by introducing simple benchmarking instruments \cite{kinvolk-bench} and their preliminary results \cite{bench-istio-linkerd-2019, bench-istio-linkerd-2021}. These benchmarks and instruments, however, do not capture the unique characteristics of a \gls{sm} well enough.

\todo{Fix intro last p (END)}
In this thesis, we present x, an automated service mesh evaluation benchmark that makes it easy to conduct reproducible real world experiments on different \gls{sm} implementations. With this instrument, we conduct experiments on existing \gls{sm} implementation to compare them on key performance characteristics.

% Sections 
% Why should people care about the  work


% Societal impact
% Economic -> Impact for companies (e2e latencies)
% Society at large -> We rely on infra (shift in responsibilies)
% Companies involved



% Field imapct (Intellectual Argument)
% Industry -> Considerable resouces / Relatively little scientific output
% Shift from libary approach -> service meshes
% Emerging field, new tech (ebpf)
% Lots of new systems (OSM, Cilium)
\section{Problem Statement}
\label{sec:problem-statement}

% Intro problem 
% - Fast growing, but immature
% - Operational & performance challenges
% - - SMI / SMP
% - Non Functional Requirements
% - How to compare systems
% - Balance between performance/reliability
The service mesh abstraction provides modern distributed systems best practices and tries to solve and minimize the complexities of application developers and system operators alike. It is an area of interest that is actively being worked on in the industry in a rapidly evolving landscape. Although it has gained a lot of interest in the last decade, the technology is in its relative infancy and is facing several challenges from both operational and performance related standpoints. Since it implements a layer of networking abstraction, it creates a potential for a programmable interface, which in turn allows for application and service aware traffic management capabilities. This can differentiate different service mesh implementations based on their capabilities and \glspl{nfr}, where some implementations can support advanced feature sets like circuit breaking behaviour while others do not. Collaborative efforts have are established to create a standard for service meshes \cite{service-mesh-interface, service-mesh-interface-spec}, resulting in the \gls{smi}, a standard interface for service mesh implementations on Kubernetes that describes the most common service mesh capabilities, such as traffic policies, traffic telemetry and traffic management. However, the proposed standard is relatively new, primitive and imposes the question if any existing implementation even accounts for this interface as preliminary research done by us suggests that it does not. Another aspect that is of importance is the performance characteristics of these implementations. How can we characterize the performance characteristics of a \gls{sm} so that we can objectively compare implementations. Through joint efforts, a \gls{cncf} project on a performance specification dubbed \gls{smp} has also been established \cite{service-mesh-performance, service-mesh-interface-spec}. This specification, however, is even newer than the \gls{smi} standard and at the time of writing consists of only the most basic and generic measurements available.

% Challenges (three chapters/RQ/contributions)
% - Systems Survey -> Characteristics/SMI spec comparison
% - Design/Extend perforamnce SMP Spec
% - Conduct real world experiments
The goal of this master thesis is to explore the performance characteristics of \gls{sm} implementations using a distributed systems approach. Although, as mentioned, the community has made initial efforts to solve these challenges, the solutions are far from complete nor capture the core elements to fully understand the performance implications of these systems. Also, since both aforementioned collaborative efforts are merely standards and do not carry an official implementation, it can merely act as a guidance in this research. The first challenge is to evaluate existing \gls{sm} implementations according to their characteristics and compare them to their proposed standard. The second challenge is to evaluate the existing performance standard, and if required, design or extend it to capture the core capabilities of a \gls{sm}. Given the many use cases for \gls{sm} systems and the different environments that one such system can run in, it is important to capture all the elements of such a system so that an objective comparison can be made. The final challenge will be to conduct actual experiments that capture these objective results reliably. Given the very dynamic nature of distributed systems, creating such an environment to evaluate this can be a challenge of its own. All of these challenges will guide the research in this thesis in order to evaluate existing \gls{sm} systems.
\section{Main Research Questions}
\label{sec:research-questions}

We decompose the challenges as described in the problem statement \cref{sec:problem-statement} in the following research questions.


% Research Questions
% Map the research questions to the chapters/contributions
% 1 RQ per chapter
% - Systems Survey 
% - Design/implementation of a benchmarking system
% - Real World Experiments

\todo{Research Questions: Formulation, double check RQ with Erwin/Alex, I suck at formulating these}

\begin{enumerate}[label=\textbf{RQ\arabic*}, leftmargin=3\parindent]
    \item \textbf{How do existing \gls{sm} implementations compare?}
    \label{rq-1}
    
    Many \gls{sm} implementations have emerged from within the rapidly changing landscape of container and resource management technologies, prior to the adaptation of any standard to guide their development. It is necessary to evaluate these implementations in a structured systems survey, where we identify and analyse the characteristics of the current iterations of \gls{sm} systems. This allows us to create an overview of the state-of-the-art within the \gls{sm} landscape.
    
    \item \textbf{How can we map design a system to evaluate \gls{sm} systems?}
    \label{rq-2}
    
    Based on the results of the system survey in \ref{rq-1}, we will identify the requirements for a system which can evaluate a service mesh. Based on these identified requirements, we should evaluate existing instruments to see if any of these satisfy. We then have to design or extend an instrument based on best practices \cite{folkerts2012benchmarking} which can support multiple experiment configuration, workloads and is able to produce reproducible real world experiments. The goal of this instrument is to capture the relevant system metrics, performance metrics and application domain specific \glspl{nfr} which results in a generic and objective comparison between said systems.

    \item \textbf{How can we conduct real world experiments on different service mesh implementations to objectively compare them?}
    \label{rq-3}
    
    Based on the system as designed in \ref{rq-2}, we have to conduct real world experiments on different \gls{sm} implementations. The benchmark should compare popular \gls{sm} implementations and implementations that vary based on their architectural design. The benchmark should compare different application workloads, highlighting various application specific aspects of the system where different implementations can vary in performance or capabilities. Finally, the benchmark should be as systematic as possible, meaning that anyone can verify the results by reproducing the experiments bearing in mind the inherent nature of variance within distributed systems.

\end{enumerate}

% Old ones
% \subsection{Research Questions}
% \begin{enumerate}
%   \item What does the term 'service mesh' mean in the context of distributed systems?
%   \item What are the defining properties and characteristics that define the service mesh?
%   \item How can we model the performance and reliability characteristics of a service mesh?
%   \item How can we conduct real world experiments on different service mesh implementations to objectively compare them?
% \end{enumerate}


\section{A Distributed Systems Approach}
\label{sec:approach}

% Covers methodology of this thesis

In this thesis, we approach the problem statement and their resulting research questions using distributed systems approach, a consolidation of best practices as taught and utilized by the members AtLarge group \footnote{\url{https://atlarge-research.com/}}. This guides the thesis through an approach based on conceptual, technical and experimental work that it consist of.

First, to address \ref{rq-1} we investigate the state-of-the-art in current \gls{sm} implementations. In order to do this, we perform a systematic systems survey by applying common literature survey best practices. This will help us identify key characteristics of these systems and gives us an insight on how they work. Furthermore, it allows us to create an overview of the current landscape of \gls{sm} systems and a framework of what a state-of-the-art system should look like. 

Secondly, we tackle \ref{rq-2} based on the previously identified characteristics of state-of-the-art systems from \ref{rq-1}. With this information in mind, we perform a requirements analysis for a benchmarking instrument capable of evaluating these systems in order to capture their core capabilities. We compare existing solutions to these requirements and either design or extend an instrument based on these requirements and constraints. Finally, we build a prototype of this instrument according to the aforementioned design.

Ultimately, to address \ref{rq-3} we use the instrument as devised from \ref{rq-2}. We design experiments and workloads to capture the performance characteristics and \glspl{nfr} of state-of-the-art \gls{sm} systems. We then use a cloud based environment to conduct these designed experiments on using an automated workflow. Using this methodology, we can systematically conduct real world experiments that which are reproducible. Finally, we analyze the quantitative results obtained from the experiments in order to objectively compare the \gls{sm} systems.
\section{Main Contributions}
\label{sec:contributions}

This thesis introduces qualitative, quantitative and software contributions that answer the formulated research questions. Each contribution is linked to a main research question. The contributions are as follows:

\todo{Contributions: Check formulation of contributions}
\todo{Contributions: Fix references to chapters}

% Emphasize why these concepts are challenging
% What is the usefulness of the results?
% Software Product -> Artifact
% Experiments -> Artifact

\begin{enumerate}
    \item (\textit{Conceptual}, \ref{rq-1}) A qualitative survey on the characteristics of current state-of-the-art \gls{sm} systems \cref{chap:survey}.
    
    \item (\textit{Conceptual}, \ref{rq-1}) Analysis of current state-of-the-art \gls{sm} systems and their conformance with current proposed industry standards \cref{chap:survey}.    
    
    
    \item (\textit{Conceptual}, \ref{rq-2}) Requirements analysis and design of a benchmark instrument capable of performing systematic and reproducible real world experiments on \gls{sm} systems \cref{chap:survey}.  
    
    \item (\textit{Technical}, \ref{rq-2}) Technical implementation of a benchmarking instrument based on the former requirements in the form of open-source software. \cref{chap:survey}.


    \item (\textit{Experimental}, \ref{rq-3}) Design execution of real world experiments on cloud environments \cref{chap:survey}.
    
    \item (\textit{Experimental}, \ref{rq-3}) Quantitive results and analysis of state-of-the-art \gls{sm} systems \cref{chap:survey}.
\end{enumerate}


 

\input{1_introduction/sections/reading-guide}
