% GLOSSARY ENTRIES
% Defines acronyms and descriptions for frequently used
% terminology present in this thesis.
% http://tug.ctan.org/macros/latex/contrib/glossaries/glossariesbegin.pdf

% Define a term like this:
% \newglossaryentry{⟨label⟩}
% {
%     name={⟨name⟩},
%     description={⟨description⟩},
%     ⟨other options⟩
% }

% Acronym Example
% \newabbreviation{svm}{svm}{support vector machine}


% --------------------------------------------------------------------------

\makeglossaries


% --------------------------- ABBREVIATIONS --------------------------------

\newabbreviation{soa}{SOA}{Service Oriented Architecture}
\newabbreviation{iot}{IoT}{Internet of Things}
\newabbreviation{qos}{QoS}{Quality of Service}
\newabbreviation{kpi}{KPI}{Key Performance Indicator}
\newabbreviation{json}{JSON}{JavaScript Object Notation}
\newabbreviation{aip}{AIP}{Article Information Parser}




%-------------------------- END ABBREVIATIONS ------------------------------




% ----------------------------- GLOSSARIES ---------------------------------

\newglossaryentry{sm}
{
    name={Service Mesh},
    plural={Service Meshes},
    description={A service mesh is a dedicated infrastructure layer that that abstracts the network layer between (micro)-services, to introduce distributed systems' best practices such as observability, traffic management, and security in a uniform way.}
}

\newglossaryentry{edge-computing}
{
    name={edge eomputing},
    description={A distributed computing paradigm that brings computation and data storage closer to the sources of data.}
}

\newglossaryentry{smi}
{
    name={Service Mesh Interface},
    description={A standard interface for service meshes on Kubernetes, which covers the most common service mesh capabilities such as traffic policies, traffic telemetry and traffic management.}
}

\newglossaryentry{smp}
{
    name={Service Mesh Performance},
    description={A common format to capture and describe configurations and results of service mesh performance experiments.}
}



\newglossaryentry{nfr}
{
    name={Non-Functiona Requirement},
    description={A a requirement that specifies criteria that can be used to judge the operation of a system, rather than specific behaviors.}
}

\newglossaryentry{cncf}
{
    name={Cloud Native Computing Foundation},
    description={A Linux Foundation project that was founded in 2015 to help advance container technology and align the tech industry around its evolution in a vendor neutral fashion.}
}

\newglossaryentry{pl}
{
    name={Primary Literature},
    description={The immediate results of research activities by scientists.}
}

\newglossaryentry{gl}
{
    name={Grey Literature},
    description={The materials and research produced by organizations outside of the traditional commercial or academic publishing.}
}

\newglossaryentry{slr}
{
    name={Systematic Literature Review},
    description={A type of literature review that uses repeatable analytical methods to collect and analyze data.}
}

\newglossaryentry{mlr}
{
    name={Multivocal Literature Review},
    description={A form of systematic literature reviews that includes \gls{gl} in addition to the formal \gls{pl}.}
}

\newglossaryentry{backward-snowballing}
{
    name={Backward Snowballing},
    description={A form of systematic literature reviews that includes \gls{gl} in addition to the formal \gls{pl}.}
}

\newglossaryentry{devops}
{
    name={DevOps},
    description={A set of software development (dev) and it operations (ops) practices which aim to shorten the systems development life cycle and provide continuous delivery with high software quality.}
}

\newglossaryentry{container}
{
    name={container},
    description={A unit of software packaging that combines the software and all of its dependencies in a standard format.}
}

\newglossaryentry{docker}
{
    name={Docker},
    description={A software platform used to create, run and manage software applications packaged in \glspl{container}.}
}

\newglossaryentry{vm}
{
    name={virtual machine},
    description={A software based, virtualization of a computer system.}
}

\newglossaryentry{micro-service}
{
    name={micro-service},
    description={A variant of the \gls{soa} architectural style with fine-grained services and ligthweight protocols.}
}

\newglossaryentry{k8s}
{
    name={Kubernetes},
    description={An open-source \gls{container} orchestration system for automating software deployment, scaling, and management.}
}

\newglossaryentry{monolith}
{
    name={monolith},
    description={A software architecture that is self-contained and where the user interface and data access paterns are combined into a single program.}
}

\newglossaryentry{esb}
{
    name={Enterprise Service Bus},
    description={An Enterprise Service Bus is an open standards, message-based, distributed integration infrastructure that provides routing, invocation and mediation services to facilitate the interactions of disparate distributed applications and services in a secure and reliable manner \cite{menge2007enterprise}.}
}

\newglossaryentry{ebpf}
{
    name={eBPF},
    description={eBPF (extended Berkeley Packet Filter), is a technology which allows to run sandboxed programs in the Linux Kernel. It can do so without changing the kernel source code or loading kernel modules.}
}

\newglossaryentry{linkerd}
{
    name={Linkerd},
    description={A service mesh implementation by Buoyant.}
}





% \newglossaryentry{soa}
% {
%     name={Service Oriented Architecture},
%     description={A software development approach that takes advantages of decoupling software components into independent and reusable services.}
% }

\newglossaryentry{cabbage}
{
    name={cabbage},
    description={vegetable with thick green or purple leaves}
}

\newglossaryentry{turnip}
{
    name={turnip},
    description={round pale root vegetable}
}

\newglossaryentry{carrot}
{
    name={carrot},
    description={orange root}
}

% --------------------------- END GLOSSARIES -------------------------------
