
% Thesis Abstract -----------------------------------------------------


%\begin{abstractslong}    %uncommenting this line, gives a different abstract heading
\begin{abstracts}        %this creates the heading for the abstract page
Service-oriented computing, an architectural approach of decomposing software into logical services, has seen an increase in popularity over the past decades. Later forms, such as the widely popular microservices-based architecture is an evolution of service-oriented computing made possible by the reductions in deployment costs and resource overhead. These architectures offer advantageous characteristics, such as an increase in flexibility and scalability. However, due to its design, it also introduces an additional layer of complexity due to the overhead in communication and the inherent reliability issues from networks. The service mesh architecture tries aims to solve these issues whilst providing a uniform way of handling security, observability and reliability concerns without modifying application code. In this thesis, we dissect the state-of-the service mesh systems and identify three different service mesh architectures. Furthermore, we design \textit{Mesh Bench} a benchmark used to evaluate the performance characteristics of service mesh systems. We implement a prototype of the benchmark and take an experimental approach to evaluating the most popular service mesh implementations in the form of \textit{Istio} and \textit{Linkerd}, and implementations using vastly different architectures in the form of \textit{Cilium} and \textit{Traefik mesh}. Based on the results of the performance experiments we uncover the performance overheads of service mesh systems and identify promising, in-kernel, networking solutions.

\end{abstracts}
%\end{abstractlongs}


% ---------------------------------------------------------------------- 
